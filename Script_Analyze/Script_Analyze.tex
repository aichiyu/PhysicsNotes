\documentclass[UTF8,9pt]{ctexart}
\usepackage{../template/Notes/notes}
\title{Script Analyze}
\begin{document} 
\maketitle
\se{概念}
\begin{itemize} 
\item 事件\\
    故事里的事件必须有意义、有价值、有冲突。如果一个场景从开始到结束并没有带来任何变化,这个场景就是一个无意义事件,我们叫它“非事件”,与“事件”相对。许多编剧会用“非事件”来解释有关人物或影片世界的信息,但更好的方式是把需要传达的信息分散添加到其他场景当中。
\item 场景/Scene\\
    场景:在某一相对连续的时空中,通过冲突表现出来的一段动作,这段动作至少在一个重要程度可以感知的价值层面上,使人物生活中负荷着价值的情境发生转折。故事的事件就是场景,只要这个事件符合以下条件:有意义、有价值、有冲突、包含转折。在一部常规的90-120分钟的电影里,通常会有40-60个场景。\\
    地点:内景INT./外景EXT.\\
    时间:日景DAY/夜景NIGHT
\item 序列/Sequences\\
    几个连续的场景如果互有联系、其内在冲突呈递增趋势、并在最后到达顶峰,这几个场景就构成了序列。通常,构成一个序列的有2-5个场景。
\item 幕/Act\\
    幕是一系列序列的组合,以一个高潮场景为顶点,导致价值发生重大转折,其冲击力比之前的序列或场景更加强劲。一个电影通常有三个幕。\\
    事件$\rightarrow$场景$\rightarrow$序列$\rightarrow$幕$\rightarrow$故事
\item 母题/基序/Motif\\
    母题是叙事中指在故事中重复出现、具有象征意义的元素。 一个叙事母题可以是图像、写作结构、 语言和其他叙述性元素。
\item 人物/Character\\
    有行动或台词的角色。
\item 动作/Action\\
    简洁,直接,明确。只写WHAT,不写HOW。
\item 对白/Dialogue\\
    画外音 V.O.: 画面外的声音。包括旁白,内心独白,电话传来的声音。\\
    镜外音 O.S.: 说话人在场景内镜头外,场景内的人能听到。
\item 转场/Transition\\
    淡入: /Fade in: 画面渐渐出现,由黑转入场景。通常用于剧本开头。所有转场中,只有“淡入:”写在一行的右侧,其他都靠右写。\\
    淡出/Fade out 与“淡入:”相反,画面渐渐隐去,由场景转入黑屏。用于剧本结尾。几乎所有剧本都会以“淡出”为最后一行文字。\\
    切至: /Cut to: 画面从上一个场景直接切入至下一个场景,不带什么过渡。“切至:”通常用于突发的场景切换,用于表现两个相邻场景之间的关联与冲突。\\
    叠化: /Dissolve to: 最常用的场景切换方式之一,上一个场景正在淡出时,下一个场景渐渐淡入,形成两个场景的交叉。常,“叠化:”具有暗示两个场景之间时光飞逝的作用。\\
    匹配剪辑至: /Match cut to: 使用这一转场方式的前提是:两个场景有同一个人物或道具、且电影需要表现两个场景之间的联系。\\
    淡出只黑屏: /Fade to Black: 与“淡出。”一样,这一方式的结果也是画面由场景渐渐转入黑屏。但“淡出至黑屏”只用于剧本当中、而不是剧本结尾。\\
\item 说明/Parenthetical\\
    剧本中写在角色姓名与其对白之间的文字,提示演员说对白时的语气或神情。比如“深情地”、“低声地”、“点头”、“摇头”、“耸肩”等等。与转场和镜头相似,编剧尽量少用说明,除非十分必要,让导演和演员去自由发挥。
\item 音乐/Music\\
    剧本中的音乐信息(不是电影配乐的信息)。
\item 声音/Sound\\
    如果故事中有镜头拍摄不出的声音,而这个声音对故事来说非常重要,编剧就需要在剧本中写下这一声音信息。比如:敲门声,雷声,等等。
\item 中心问题\\
    中心问题通常出现在第一幕结束时,因此成为第一幕结束的标志。第二幕和第三幕的所有内容均源于中心问题、牢牢围绕它,对中心问题的叙述必须跌宕起伏、峰回路转,不能平铺直叙。中心问题由三点构成:\\
    1. 主角的行动目标:这是主角在整部电影中最主要的使命,他为了这一目标而采取了行动。\\
    2. 主角的情感目标:这是对主角以及其周围的人来说意义重大的目标。从字面意思就可以明白,这一目标主要涉及情感,即爱情、亲情、友情。\\
    3. 主角的精神目标:这是主角埋藏于内心深处的愿望或心结(无论他是否意识到),他与这个愿望或心结抗衡了很久。\\
    {\bf 当你的中心问题得到了令人满意的答案,你的故事便可以结束了。}你对中心问题的三个回答也必须要有节奏感。当这些答案的间隔越近,观众所获得的满足感就越强。如果你能够将这三部分联系起来,你的故事就会相当有力且吸引人。
\item 结构\\
    故事的结构就是对人物生活中一系列事件的选择,这种选择将事件组合成一个具有战略意义的序列,以激发特定而具体的情感、并表达一种特定而具体的人生观。于一部电影长片来说,要构建一个有效的故事结构,我们必须以“目标”为参照来选择事件。与“目标”息息相关的,留下;与“目标”没什么关系的,删掉。\\
    故事结构就是以“目标”为明确目的、从故事全局出发,对事件进行筛选与排列。在现实生活中,事件往往会碰巧发生,但在电影故事里,事件绝不能随机排列。
\item 情节\\
    情节就是编剧对事件的选择以及对其在时间中的设计。你的故事应该包含哪些事件?应该删除哪些事件?应该合并哪些事件?什么事件先发生?什么事件后发生?
\end{itemize}
\se{要素}
\begin{itemize}
\item 一页剧本代表一分钟电影
\item 以场景为单位,场景下包含动作、人物、对白等其他要素。
\item 三幕六阶段结构\\
    第一幕  0\%-25\%:\\
    介绍角色,启动矛盾问题或冲突(1\%-10\%)。 $|$ 激励事件发生后,主角将做出反应(10\%-25\%)。\\
    第二幕 25\%-75\%: \\
    冲突加强。主角在搜寻解决问题的线索(25\%-50\%),并在中点处确定真正的目标与对手(50\%)。 $|$ 故事发展(50\%-75\%)。主角在与对手的交锋中几乎失去一切,主角开始动摇(75\%)。此时,他必须于人生的谷底站起来、奋起反击,并作出决定——一个终极决定、一个能为影片带来结局的决定。在做出这个决定之后,主角要么胜利、要么彻底失败,再也没有回头路可走。\\
    第三幕 75\%-100\%: \\
    主角做出的决定带来冲突高潮(75\%-95\%)。主角奋起反击,分出胜负(90\%-99\%)。 $|$ 尾声,矛盾问题得到解答(95\%-100\%)。
\item 一个合格的银幕故事\\
    一个{\bf 吸引观众}的角色克服一系列{\bf 难度不断增强的}困难和一系列{\bf 看似不可逾越的}障碍,最终实现其{\bf 宏伟}目标。
\item 银幕故事类型\\
    四种银幕故事类型:大情节故事、小情节故事、反情节故事、以及非情节故事。我们平时最常见的是大情节故事,其次是小情节故事。\\
    大情节故事:围绕一个主动主人公所构建的故事。主人公为了追求自己的欲望,与主要来自外界的对抗力量进行抗争,通过连续的时间、在一个连贯而具有因果关联的虚构现实里,到达一个表达绝对而变化不可逆转的闭合式结局。\\
    这是绝大部分电影所采用的故事类型,因此我们也叫它“经典设计”。“大情节”的深层含义就是这一模式的使用频率远高于其他故事类型。我们平时在影院观看的绝大部分电影讲述的都是大情节故事。\\
    小情节故事:以“大情节”为基础,对“经典设计”的成分进行一定程度的削减(对大情节的突出特性进行精炼、浓缩、删节或修剪),但在简约精炼的前提下仍能保持经典的精华。\\
    这类电影包括《撞车》、《爱情是狗娘》、《疯狂的石头》和《心迷宫》。\\
    反情节故事:将大情节完全颠倒过来,否认传统形式,反其道而行之。不再基于实际生活,反而成为想象生活的比喻。\\
    反情节故事反应的不是现实,而是电影创作者的唯我论,观众能否进入这一世界则需听从艺术家的调遣。这类电影并不多见,比如《重庆森林》。\\
    非情节故事:故事保持静止状态,没有任何变化。人物在片头和片尾没有什么变化。(无意义)
\item 结局\\
    闭合式结局/开放式结局\\
    所谓闭合式结局,就是:一个表达绝对而不可逆转变化的故事高潮,回答了故事讲述过程中所提出的所有问题并满足了所有观众情感。\\
    所谓开放式结局,就是:一个故事高潮留下了一两个未解答的问题和一些没被满足的情感。\\
    大情节故事通常拥有闭合式结局,小情节故事多用开放式结局。
\item 冲突\\
    外在冲突/内在冲突\\
    所谓外在冲突,就是来自于外界的冲突,比如主角要打败敌人,故事冲突来自于敌人。\\
    所谓内在冲突,就是来自于角色内部的冲突,比如内心的挣扎。\\
    大情节故事强调外在冲突,主角虽然也会有内心斗争,但他更大的敌对势力来自于外界的人、社会或自然界。小情节故事聚焦内在冲突,主角虽然也会与家庭、社会或自然界发生激烈冲突,但故事重点却是他自己激烈的思想斗争。
\item 主角\\
    单一主人公/多重主人公\\
    大情节故事只有一个主人公。小情节故事则常常把故事拆解为若干个次情节,每个次情节有一个主人公,此时,整个故事就有多重主人公,小情节故事也因此常被称为多情节故事。\\
    主动主人公/被动主人公\\
    主动主人公在追求欲望的时候主动采取行动、并与他周围的人和世界发生直接冲突。\\
    被动主人公往往表面消极被动,在内心追求欲望时与其自身性格的方方面面发生冲突。\\
    在大情节故事中,主人公几乎都是主动的,在不断升级的冲突、不断升级的障碍中意志坚定地追求目标。比如主角被动遭难后,会主动作出回应、主动发起反击去实现自己的愿望。在小情节故事中,主人公相对是被动的,具有强烈的内心斗争。
\item 时间\\
    线性时间/非线性时间\\
    所谓线性时间,就是:一个故事的事件被安排成一个观众能够跟踪的时间顺序,尽管有时候其中会穿插一些闪回。\\
    所谓非线性时间,就是:一个故事在时间中随意跳跃,从而模糊了时间的连续性,以致于观众无法判断什么事件发生在前什么事件发生在后。\\
    大情节与小情节故事采用线性时间,开始于时间中的某一点、在大略连贯的时间中运行、并终结于某一个晚些的时日。即使使用闪回,大情节与小情节故事也会让观众将故事的事件正确地置于其时间顺序当中。反情节故事采用非线性时间,将时间顺序打乱或拆解,观众很难或不可能将事件置于任何线性的时间顺序当中。
\item 因果/巧合\\
    所谓因果驱动的故事,就是:有动机的动作导致结果,这些结果又变成其他结果的原因,从而在导向故事高潮的各个片段的连锁反应中将冲突的各个层面相互连接,表现出现实的相互联系性。\\
    所谓巧合驱动的故事,就是:一个虚构的世界,动机不明的动作触发出不会产生进一步结果的事件,因此将故事拆解为互不关联的片段和一个开放式的结尾,表现出现实存在的互不关联性。\\
    大情节故事通常以因果来驱动故事,即人物的动机导致他的行为、继而导致事件的发生。但反情节故事常常以巧合驱动故事。无论怎样,就记住:生活中有许多巧合,但电影中体现的更多的是因果。
\item 连贯性\\
    连贯现实/非连贯现实\\
    所谓连贯现实,就是:一个内部连贯一致的世界,其本身必须能够自圆其说。即使你创造作故事基于一个虚拟的世界观,你的人物和虚拟世界观也会保持相应的互动模式,这个互动模式在故事中始终保持连贯性。\\
    所谓非连贯现实,就是:混合了多种互动模式的背景,故事情节会不连贯地从一个现实跳向另一个现实,以营造出一种荒诞感。\\
    大情节与小情节故事在连贯现实中展开,每一个虚构现实都确立了期间事件独一无二的发生规律,这些规律决不能被打破——即便这些规律在现实生活中并不存在。反情节故事则恰恰相反,它的唯一规律就是打破规律,因此基于非连贯现实。
\item 故事背景\\
    故事背景由四部分组成:时代、期限、地点、冲突层面。\\
    时代是故事在时间中的位置。(过去/现在/未来)。\\
    期限是故事在时间中的长度。(一天/一年)。\\
    地点是故事在空间中的位置。(城市/国家/街道)。\\
    冲突层面是故事在人类斗争的层级体系中的位置。(政治/经济/意识形态/社会力量/人物内心/人际关系/社会机构或帮派之间的斗争/人类与环境之间的斗争)\\
    故事背景越大,你的知识就越被稀释,你的创作选择也就越少,故事就越容易陈词滥调。故事背景越小,你的知识就越完善,你的创作选择也就越多,故事就越容易新颖独特。


\end{itemize}
\se{故事的关键组成部分}
故事有五个关键的组成部分:{\bf 激励事件,进展纠葛,危机,高潮,结局}
\begin{itemize}
\item 激励事件(Inciting Incident,也有地方叫危机事件)\\
    激励事件是影片中的第一个重大事件,是彻底打破主角生活平衡的事件,主角会对激励事件做出反应。激励事件是影片的重要转折,是第一幕的高潮、也是之后故事的导火索。激励事件必须牢牢勾住观众的心,让观众跟随主角一起做出反应。\\
    条件:\\
    1. 单一事件。\\
    2. 与主角有直接联系(发生在主角身上/与主角有因果关系)。\\
    发生方式:\\
    1. 随机发生,出于巧合发生在主角身上。\\
    2. 事出有因:出于某种决定,决定来自于主角或主角的对手。
\item 进展纠葛\\
    进展纠葛是故事的主体,始于激励事件、终于最后一幕的高潮。所谓进展,就是人物推动故事向前发展。所谓纠葛,就是为人物的生活制造磨难。所以,进展纠葛就是:当人物面对越来越强大的对抗力量时,产生越来越多的冲突,从而创造出一系列逐次发生的事件,经过一个个无法回归的点。\\
    故事的发展源于冲突。没有冲突就没有向前进展的故事。\\
    冲突有三个层面:\\
    1. 内心冲突,比如内心的挣扎。\\
    2. 个人冲突,主角与他人的冲突,比如情敌较量。\\
    3. 个人外冲突,主角与组织、机构、甚至国家的对抗。
\item 危机\\
    人物对事件的应对选择是推动故事的关键。在每个场景中,人物都会做出决定,而{\bf“危机”是 最终极的决定}:人物距离实现自己的目标仅剩最后一步,他必须做出选择,究竟A还是B,它们中的一个会让主角达成欲望、但另一个则会让主角失败,不成功便成仁。主角在此时此刻所做的决定会引发他的{\bf 最后一个行动},之后不会再有第二次机会,要么成功、要么失败。\\
    当主角面对危机做出终极决定之后,他便迈出了通向目标的最后一步,从而引发出全片的高潮。
\item 高潮\\
    高潮是故事不可或缺的重要部分,是整个故事的重大逆转。只有{\bf 绝对而不可逆、且充满意义的高潮}才能令观众印象深刻、大呼过瘾。除此处外所有决定/事件均可以是可逆的。\\
    在进行故事设计时,我们会在尚未完成进展纠葛时便设计出故事高潮,然后在创造出高潮后进行逆向工作,为高潮在故事世界中找到支撑、提供必要的因果依据,由此设计出第二幕的整篇进展纠葛。高潮之前的所有场景都必须以高潮为参照物,与高潮无关的一切场景都没有存在的必要。由于危机与高潮紧密相连,因此两者往往发生在同一场景内。如果你的危机发生在此时此地,而高潮发生在彼时彼地,那么我们必须把两者剪接在一起、让它们在电影时空中融为一体。\\
    高潮的共性:主角甘愿豁出性命去达成目标(绝对而不可逆、且充满意义)。
\item 结局\\
    结局必须不可避免却又出乎意料。如果你的故事有一条次情节,而它在全片高潮结束之后仍然没有迎来自己的高潮,那么你就需要在故事的结局部分为它安排一个场景。如果你的高潮影响巨大、带来阵阵涟漪,那么你需要在故事的结局部分展示高潮效果的影响波及。
\end{itemize}
\se{Ref}
\url{https://www.zhihu.com/people/cagezhujunjie/activities}
\end{document}
