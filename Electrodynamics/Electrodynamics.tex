\documentclass[UTF8,9pt]{ctexart}
\usepackage{../template/Notes/notes}
\title{ElectroDynamics}
\begin{document}
\maketitle
\se{方程} 
真空麦克斯韦方程 
$$\ar[rclrcll]{
    \nabla\cdot\bm{E}&=&\f{\rho}{\epsilon_0}&\qquad \oint_S\bm{E}\d \bm{s}&=&\f{Q}{\epsilon_0}&\text{高斯定律}\\
    \nabla\cdot\bm{B}&=&0&\qquad \oint_S\bm{B}\d \bm{s}&=&0&\text{高斯磁定律}\\
    \nabla\times\bm{E}&=&-\pt{\bm{B}}&\qquad \oint_L\bm{E}\d \bm{l}&=&-\dt{\varphi_B}&\text{法拉第电磁感应定律}\\
    \nabla\times\bm{B}&=&\mu_0\bm{J}+\mu_0\epsilon_0 \pt{\bm{E}}&\qquad \oint_L\bm{B}\d \bm{l}&=&\mu_0I+\mu_0\epsilon_0\dt{\varphi_E}\quad &\text{安培定律}
}$$

物质内麦克斯韦方程
$$\ar[rclrcll]{
    \nabla\cdot\bm{D}&=&\rho_f&\qquad \oint_S\bm{D}\d \bm{s}&=&Q_f&\text{高斯定律}\\
    \nabla\cdot\bm{B}&=&0&\qquad \oint_S\bm{B}\d \bm{s}&=&0&\text{高斯磁定律}\\
    \nabla\times\bm{E}&=&-\pt{\bm{B}}&\qquad\quad\qquad \oint_L\bm{E}\d \bm{l}&=&-\dt{\varphi_B}&\text{法拉第电磁感应定律}\\
    \nabla\times\bm{H}&=&\bm{J}_f+\pt{\bm{D}}&\qquad \oint_L\bm{H}\d \bm{l}&=&I_f+\dt{\varphi_D}\quad\qquad &\text{安培定律}
}$$

镜像法\\
距半径为$R_0$的球的球心距离为$a$处有一点电荷$q$, 则镜像电荷$-\f{R_0}{a}q$距球心$\f{R_0^2}{a}$远,且在靠近$q$方向.

球谐函数解拉普拉斯方程
$$\vp(r,\t)=\sum_{n=0}^\infty(a_nr^n+\f{b_n}{r^{n+1}})P_n(\cos\t)$$
$$\left\{\ar{
    P_0(\cos\t)=&1\\
    P_1(cos\t)=&\cos\t\\
    P_2(\cos\t)=&\ff{2}(3\cos^2\t-1)
}\right.$$
$$\frac{1}{\left| \bm{R} -\bm{a}' \right|} = \frac{1}{\sqrt{R^2+a^{\prime 2} - 2Ra\cos\t}} = \left\{\ar{
\ff{a}\sum_{n=0}^{\infty} (\f{R}{a})^n P_n(\cos \t),\quad (R<a)\\
\ff{R}\sum_{n=0}^{\infty} (\f{a}{R})^n P_n(\cos \t),\quad (R>a)
}\right.$$

格林函数法
$$\nabla^2G(x,x')=-\ff{\ep}\de^3(x-x')$$
(1) 无界空间中
$$G(x,x')=\ca\ff{\abs{r-r'}}$$
(2) 上半平面中
$$G(x,x')=\ca(\ff{\abs{r-r'}}-\ff{\abs{r+r'}})$$
(3) 球外空间 ($R'$为电荷位置, $\a$为场点与电荷位置夹角,$R_0$为球半径).
$$G(x,x')=\ca(\ff{R^2+R'^2-2RR'\cos\a}-\ff{(\f{RR'}{R_0})^2+R_0^2-2RR'\cos\a})$$
给定$\rho(x')$, 第一类边值问题的解为($G$交换了$x,x'$):
$$\vp(x)=\int_VG(x',x)\rho(x')\d V'+\ep_0\oint_S(G(x',x)\pp{\vp}{n'}-\vp(x')\pp{G(x',x)}{n'}\d S')$$
若给定边界$\vp$, 则应使$G$在边界为0, 若给定边界$\pp{\vp}{n}$, 则应使$\pp{G}{n}$在边界为0.

泊松方程 
$${\nabla}^2 \Phi =\div\bm{E}= - {\rho \over \epsilon_0}$$

电多极矩
$$\varphi(\bm{x})=\frac{1}{4 \pi \epsilon_{0}}(\frac{q}{R}+\f{\bm{p}\cdot\bm{R}}{R^3} + \frac{1}{6} \sum_{i, j} \mathfrak{D}_{i j} \frac{\partial^{2}}{\partial x_{i} \partial x_{j}} \frac{1}{R}+\cdots)$$
$$\bm{E}=-\ca(\f{3(\bm{p}\cdot\bm{R})\bm{R}}{R^5}-\f{\bm{p}}{R^3})$$
$$\bm{p}=\iiint_V\rho(x')x'\d^3x'$$
$$\mathfrak{D}=\iiint_{V} 3 \bm{x}^{\prime} \bm{x}^{\prime} \rho(\bm{x}^{\prime}) d^{3} x^{\prime}$$

磁偶极矩
$$\vp = \f{\bm{m}\cdot\bm{R}}{4\pi R^3}$$
$$\bm{m}=\ff{2}\iiint_V\bm{x}'\tm \bm{J}(\bm{x'})\d^3x'$$

保角变换 ($z_1$为原来的点, $a$为夹角出现的位置的横坐标, $\a$为边界夹角).
$$\frac{\d z_{1}}{\d z_{2}}=C_{1} \prod_{i=1}^{n}(z_{2}-a_{i})^{\frac{\alpha_{i}}{\pi}-1}$$

电荷
$$\ar[rl|rl|rl]{
    \text{面电荷}&&\text{电场}&&\text{磁场}\\
    \hline
    \sigma_{polar} =& P\quad&\quad \rho_p=&-\div\bm{P} \quad&\quad \rho_M=&-\mu_0\div\bm{M}\\
    \sigma_{free} =& D \quad&\quad \rho_f=&\div\bm{D} \quad&\quad \rho_f=&0 \\
    \sigma_{total} =& \epsilon_0E \quad&\quad \rho_{tot} =& \ep_0\div\bm{E} \quad&\quad \rho_{tot} = \rho_p =& \mu_0\div\bm{H}\\
    \hline
    &&\ep_0E=&D-P&B=&\mu_0(H+M)
}$$

边界条件
$$\ar[rcl|rcl]{
    \text{磁场}&&&\text{电场}\\
    \hline
    \varphi_{1}&=&\varphi_{2}&\varphi_{1}&=&\varphi_{2} \\ 
    \mu_{1} \frac{\partial \varphi_{1}}{\partial n} &=& \mu_{2} \frac{\partial \varphi_{2}}{\partial n} & \varepsilon_{1} \frac{\partial \varphi_{1}}{\partial n}+\sg_f&=&\varepsilon_{2} \frac{\partial \varphi_{2}}{\partial n}\\
    &&&\sg_{\text{电导}1}\frac{\partial \varphi_{1}}{\partial n}&=&\sg_{\text{电导}2}\frac{\partial \varphi_{2}}{\partial n}\\
    H_{1\parallel}+\bm{\a}_f&=&H_{2\parallel}&E_{1\parallel}&=&E_{2\parallel}\\  
    B_{1\perp}&=&B_{2\perp}&D_{1\perp}+\sg_f&=&D_{2\perp}\\
    M_{1\perp}+\a_M &=& M_{2\perp} & \\
    B_{2\perp}-B_{1\perp}&=& \mu_0(\a_f+\a_M)&E_{2\perp}-E_{1\perp}&=&\,(\sg_f+\sg_p)/\ep_0\\
    B_\perp &=& 0 (\text{超导球}) & D_\perp &=& \sg_f (\text{导体})
}$$

洛伦兹力:$$\ar{
    \bm{F}=&q\bm{E}+q\bm{v}\times\bm{B}\\
    \bm{f}=&\rho\bm{E}+\bm{J}\times\bm{B}
}$$

磁偶极子:
$$ 
\vec{B}=\frac{\mu_{0}}{4 \pi} \frac{3(\vec{m} \cdot \widehat{R}) \widehat{R}-\vec{m}}{R^{3}}
 $$
$$\vp = \f{\bm{m}\cdot\bm{R}}{4\pi R^3}$$

电磁场:
$$\bm{S}=\bm{E}\times\bm{H}$$
$$w=\ff{2}(\bm{E\cdot D}+\bm{H\cdot B})=\ff{2}(\rho\vp+\bm{J}_f\cdot\bm{A})$$

电流:
$$\nabla\cdot J=-\pt{\rho}$$
$$J=\sigma E$$

毕奥——萨伐尔定律  $\bm{B}=\frac{\mu_0}{4\pi}\int\frac{I\d l\times \bm{e}_r}{r^2}$,若$I$为直线,$B=\f{\mu_0Il}{4\pi r^2}$.

电磁波:
$$\ar[rcl|rcl]{
    \bm{\nabla}\times \bm{B}-\ff{c^2}\pt{\bm{E}}&=&0\quad&\quad \bm{\nabla^2 E}-\ff{c^2}\pt[2]{\bm{E}}=\square \bm{E}&=&0\\
    \bm{\nabla}\times \bm{E}+\pt{\bm{B}}&=&0&\bm{\nabla^2 B}-\ff{c^2}\pt[2]{\bm{B}}=\square \bm{B}&=&0
}$$

磁矢势:\\
库仑规范:
$$\ar[rcl]{
    \bm{\nabla}\cdot \bm{A}&=&0\\
    \nabla^2 \varphi&=&-\f{\varphi}{\epsilon_0}\\
    \bm{\square A}&=&-\mu_0\bm{J}+\ff{c^2}\nabla\pt{\varphi}
}$$
洛伦兹规范:
$$\ar[rcl]{
    \bm{\nabla}\cdot\bm{A}+\ff{c^2}\pt{\varphi}&=&0\\
    \square \varphi&=&-\f{\rho}{\epsilon_0}\\
    \bm{\square A}&=&-\mu_0\bm{J}
}$$
\se{数学}
\sub{柱坐标系 $(\rho,\phi,z)$}
$$\nabla \varphi = \hat e_1 \frac { \partial \varphi } { \partial \rho } + \hat e_2 \frac { 1 } { \rho } \frac { \partial \varphi } { \partial \phi } + \hat { e } _ { 3 } \frac { \partial \varphi } { \partial z }$$
$$\nabla \cdot \bm{ A } = \frac { 1 } { \rho } \frac { \partial ( \rho A _ { 1 } ) } { \partial \rho } + \frac { 1 } { \rho } \frac { \partial A _ { 2 } } { \partial \phi } + \frac { \partial A _ { 3 } } { \partial z }$$
$$ { \nabla \times \bm{ A } } =\hat e_1 (\ff{\rho} \frac { \partial A_3} { \partial \phi } - \frac { \partial A _ { 2 } } { \partial z } ) + \hat { e } _ { 2 } ( \frac { \partial A _ { 1 } } { \partial z } - \frac { \partial A _ { 3 } } { \partial \rho } ) + \hat { e } _ { 3 } \frac { 1 } { \rho } ( \frac { \partial ( \rho A _ { 2 } ) } { \partial \rho } - \frac { \partial A _ { 1 } } { \partial \phi } )$$
$$\nabla ^ { 2 } \varphi = \frac { 1 } { \rho } \frac { \partial } { \partial \rho } ( \rho \frac { \partial \varphi } { \partial \rho } ) + \frac { 1 } { \rho ^ { 2 } } \frac { \partial ^ { 2 } \varphi } { \partial \phi ^ { 2 } } + \frac { \partial ^ { 2 } \varphi } { \partial z ^ { 2 } }$$
\sub{球坐标系 $(r,\t,\varphi)$ }
$$\nabla \varphi = \hat { e } _ { 1 } \frac { \partial \varphi } { \partial r } + \hat { e } _ { 2 } \frac { 1 } { r } \frac { \partial \varphi } { \partial \theta } + \hat { e } _ { 3 } \frac { 1 } { r \sin \theta } \frac { \partial \varphi } { \partial \phi }q$$
$$\nabla \cdot \bm{ A } = \ff{r^2} \pp{r^2A_1}{r} + \frac { 1 } { r \sin \theta } \frac { \partial } { \partial \theta } ( \sin \theta A _ { 2 } ) + \frac { 1 } { r \sin \theta } \frac { \partial A _ { 3 } } { \partial \phi }$$
$$\nabla \times \bm{ A }  = 
\hat { e } _ { 1 } \frac { 1 } { r \sin \theta } \left[ \frac { \partial } { \partial \theta } ( \sin \theta A _ { 3 } ) - \frac { \partial A _ { 2 } } { \partial \phi } \right]  
+ \hat { e } _ { 2 } \left[ \frac { 1 } { r \sin \theta } \frac { \partial A _ { 1 } } { \partial \phi } - \frac { 1 } { r } \frac { \partial } { \partial r } ( r A _ { 3 } ) \right] 
+ \hat { e } _ { 3 } \frac { 1 } { r } \left[ \frac { \partial } { \partial r } ( r A _ { 2 } ) - \frac { \partial A _ { 1 } } { \partial \theta } \right] $$
$$\nabla^2\varphi=\ff{r^2\sin\t}\left[\sin\t\pp{}{r}(r^2\pp{\varphi}{r})+\pp{}{\t}(\sin\t\pp{\varphi}{\t})+\ff{\sin\t}\pp[2]{\varphi}{\phi}\right]$$
\sub{矢量变换}
$$\nabla\bm{r}=-\nabla'\bm{r}=\bm{e}_r$$
$$\nabla\ff{\bm{r}}=-\nabla'\ff{\bm{r}}=-\ff{r^2}\bm{e}_r$$
$$\curl\ff{\bm{r}^2}=\div\ff{\bm{r}}=0$$
$$\div\vp\bm{A}=\vp\div\bm{A}+\bm{A}\cdot\grad\vp$$
$$\curl\vp\bm{A}=\vp\curl\bm{A}+\grad\vp\times\bm{A}$$
$$\div(\bm{A}\cdot\bm{B}) = (\bm{A}\cdot\nabla)\bm{B}+(\bm{B}\cdot\nabla)\bm{A}+\bm{A}\tm(\curl\bm{B})+\bm{B}\tm(\curl\bm{A})$$
$$\nabla\cdot(\bm{F} \times \bm{G})=(\nabla\times \bm{F})\cdot \bm{G}-\bm{F}\cdot(\nabla\times \bm{G})$$
$$\bm{A}\times(\bm{B}\times\bm{C})=(\bm{A}\cdot\bm{C})\bm{B}-(\bm{A}\cdot\bm{B})\bm{C}$$
$$( \bm{ A } \times \bm{ B } ) \times ( \bm{ C } \times \bm{ D } ) = [ \bm{ A } \cdot ( \bm{ B } \times \bm{ D } ) ] \bm{ C } - [ \bm{ A } \cdot ( \bm{ B } \times \bm{ C } ) ] \bm{ D }$$
$$( \bm{ A } \times \bm{ B } ) \cdot ( \bm{ C } \times \bm{ D } ) = \abs{\ar{\bm{A}\cdot\bm{C}&\ \ \bm{A}\cdot\bm{D}\\
\bm{B}\cdot\bm{C}&\ \ \bm{B}\cdot\bm{D}}}$$
$$\bm{ A } \times ( \bm{ B } \times \bm{ C } ) + \bm{ B } \times ( \bm{ C } \times \bm{ A } ) + \bm{ C } \times ( \bm{ A } \times \bm{ B } ) = 0$$
\end{document}