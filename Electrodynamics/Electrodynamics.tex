\documentclass[UTF8,9pt]{ctexart}
\usepackage{../template/Notes/notes}
\title{ElectroDynamics}
\begin{document}
\maketitle
\se{方程} 
真空麦克斯韦方程 
$$\ar[rclrcll]{
    \nabla\cdot\bm{E}&=&\f{\rho}{\epsilon_0}&\qquad \oint_S\bm{E}\d \bm{s}&=&\f{Q}{\epsilon_0}&\text{高斯定律}\\
    \nabla\cdot\bm{B}&=&0&\qquad \oint_S\bm{B}\d \bm{s}&=&0&\text{高斯磁定律}\\
    \nabla\times\bm{E}&=&-\pt{\bm{B}}&\qquad \oint_L\bm{E}\d \bm{l}&=&-\dt{\varphi_B}&\text{法拉第电磁感应定律}\\
    \nabla\times\bm{B}&=&\mu_0\bm{J}+\mu_0\epsilon_0 \pt{\bm{E}}&\qquad \oint_L\bm{B}\d \bm{l}&=&\mu_0I+\mu_0\epsilon_0\dt{\varphi_E}\quad &\text{安培定律}
}$$
物质内麦克斯韦方程
$$\ar[rclrcll]{
    \nabla\cdot\bm{D}&=&\rho_f&\qquad \oint_S\bm{D}\d \bm{s}&=&Q_f&\text{高斯定律}\\
    \nabla\cdot\bm{B}&=&0&\qquad \oint_S\bm{B}\d \bm{s}&=&0&\text{高斯磁定律}\\
    \nabla\times\bm{E}&=&-\pt{\bm{B}}&\qquad\quad\qquad \oint_L\bm{E}\d \bm{l}&=&-\dt{\varphi_B}&\text{法拉第电磁感应定律}\\
    \nabla\times\bm{H}&=&\bm{J}_f+\pt{\bm{D}}&\qquad \oint_L\bm{H}\d \bm{l}&=&I_f+\dt{\varphi_D}\quad\qquad &\text{安培定律}
}$$
泊松方程 
$${\nabla}^2 \Phi = - {\rho \over \epsilon_0}$$
电荷
$$\ar{
    \sigma_{total} =& \epsilon_0E\\
    \sigma_{polar} =& P\\
    \sigma_{free} =& D
}$$
边界条件(当无电流和自由电荷)
$$\ar[rcl|rcl]{
    H_{1\parallel}&=&H_{2\parallel}&E_{1\parallel}&=&E_{2\parallel}\\  
    B_{1\perp}&=&B_{2\perp}&D_{1\perp}&=&D_{2\perp}
}$$
洛伦兹力:$$\ar{
    \bm{F}=&q\bm{E}+q\bm{v}\times\bm{B}\\
    \bm{f}=&\rho\bm{E}+\bm{J}\times\bm{B}
}$$
电磁场:
$$\bm{S}=\bm{E}\times\bm{H}$$
$$w=\ff{2}(\bm{E\cdot D}+\bm{H\cdot B})$$
电流:
$$\nabla\cdot J=-\pt{\rho}$$
$$J=\sigma E$$
毕奥——萨伐尔定律  $B=\frac{\mu_0}{4\pi}\int\frac{I\d l\times \bm{e}_r}{r^2}$,若$I$为直线,$B=\f{\mu_0Il}{4\pi r^2}$\\
电磁波:
$$\ar[rcl|rcl]{
    \bm{\nabla}\times \bm{B}-\ff{c^2}\pt{\bm{E}}&=&0\quad&\quad \bm{\nabla^2 E}-\ff{c^2}\pt[2]{\bm{E}}=\square \bm{E}&=&0\\
    \bm{\nabla}\times \bm{E}+\pt{\bm{B}}&=&0&\bm{\nabla^2 B}-\ff{c^2}\pt[2]{\bm{B}}=\square \bm{B}&=&0
}$$
磁矢势:\\
库仑规范:
$$\ar[rcl]{
    \bm{\nabla}\cdot \bm{A}&=&0\\
    \nabla^2 \varphi&=&-\f{\varphi}{\epsilon_0}\\
    \bm{\square A}&=&-\mu_0\bm{J}+\ff{c^2}\nabla\pt{\varphi}
}$$
洛伦兹规范:
$$\ar[rcl]{
    \bm{\nabla}\cdot\bm{A}+\ff{c^2}\pt{\varphi}&=&0\\
    \square \varphi&=&-\f{\rho}{\epsilon_0}\\
    \bm{\square A}&=&-\mu_0\bm{J}
}$$
\se{数学}
\sub{Cylindrical coordinates $(\rho,\phi,z)$}
$$\nabla \varphi = \hat e_1 \frac { \partial \varphi } { \partial \rho } + \hat e_2 \frac { 1 } { \rho } \frac { \partial \varphi } { \partial \phi } + \hat { e } _ { 3 } \frac { \partial \varphi } { \partial z }$$
$$\nabla \cdot \vec { A } = \frac { 1 } { \rho } \frac { \partial ( \rho A _ { 1 } ) } { \partial \rho } + \frac { 1 } { \rho } \frac { \partial A _ { 2 } } { \partial \phi } + \frac { \partial A _ { 3 } } { \partial z }$$
$$ { \nabla \times \vec { A } } =\hat e_1 (\ff{\rho} \frac { \partial A_3} { \partial \phi } - \frac { \partial A _ { 2 } } { \partial z } ) + \hat { e } _ { 2 } ( \frac { \partial A _ { 1 } } { \partial z } - \frac { \partial A _ { 3 } } { \partial \rho } ) + \hat { e } _ { 3 } \frac { 1 } { \rho } ( \frac { \partial ( \rho A _ { 2 } ) } { \partial \rho } - \frac { \partial A _ { 1 } } { \partial \phi } )$$
$$\nabla ^ { 2 } \varphi = \frac { 1 } { \rho } \frac { \partial } { \partial \rho } ( \rho \frac { \partial \varphi } { \partial \rho } ) + \frac { 1 } { \rho ^ { 2 } } \frac { \partial ^ { 2 } \varphi } { \partial \phi ^ { 2 } } + \frac { \partial ^ { 2 } \varphi } { \partial z ^ { 2 } }$$
\sub{Spherical coordinates $(r,\t,\varphi)$ }
$$\nabla \varphi = \hat { e } _ { 1 } \frac { \partial \varphi } { \partial r } + \hat { e } _ { 2 } \frac { 1 } { r } \frac { \partial \varphi } { \partial \theta } + \hat { e } _ { 3 } \frac { 1 } { r \sin \theta } \frac { \partial \varphi } { \partial \phi }q$$
$$\nabla \cdot \vec { A } = \ff{r^2} \pp{r^2A_1}{r} + \frac { 1 } { r \sin \theta } \frac { \partial } { \partial \theta } ( \sin \theta A _ { 2 } ) + \frac { 1 } { r \sin \theta } \frac { \partial A _ { 3 } } { \partial \phi }$$

$$\nabla \times \vec { A }  = \
hat { e } _ { 1 } \frac { 1 } { r \sin \theta } \left[ \frac { \partial } { \partial \theta } ( \sin \theta A _ { 3 } ) - \frac { \partial A _ { 2 } } { \partial \phi } \right]  
+ \hat { e } _ { 2 } \left[ \frac { 1 } { r \sin \theta } \frac { \partial A _ { 1 } } { \partial \phi } - \frac { 1 } { r } \frac { \partial } { \partial r } ( r A _ { 3 } ) \right] 
+ \hat { e } _ { 3 } \frac { 1 } { r } \left[ \frac { \partial } { \partial r } ( r A _ { 2 } ) - \frac { \partial A _ { 1 } } { \partial \theta } \right] $$
$$\nabla^2\varphi=\ff{r^2\sin\t}\left[\sin\t\pp{}{r}(r^2\pp{\varphi}{r})+\pp{}{\t}(\sin\t\pp{\varphi}{\t})+\ff{\sin\t}\pp[2]{\varphi}{\phi}\right]$$
\sub{Vector Trans}
$$\nabla\cdot(F\times G)=(\nabla\times F)\cdot G-F\cdot(\nabla\times G)$$
$$\vec{A}\times(\vec{B}\times\vec{C})=(\vec{A}\cdot\vec{C})\vec{B}-(\vec{A}\cdot\vec{B})\vec{C}$$
$$( \vec { A } \times \vec { B } ) \times ( \vec { C } \times \vec { D } ) = [ \vec { A } \cdot ( \vec { B } \times \vec { D } ) ] \vec { c } - [ \vec { A } \cdot ( \vec { B } \times \vec { C } ) ] \vec { D }$$
$$\vec { A } \times ( \vec { B } \times \vec { C } ) + \vec { B } \times ( \vec { C } \times \vec { A } ) + \vec { C } \times ( \vec { A } \times \vec { B } ) = 0$$
\end{document}