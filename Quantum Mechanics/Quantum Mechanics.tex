\documentclass[UTF8]{ctexart}
\usepackage{../template/Notes/notes}
\title{Quantum Mechanics}
\begin{document}
\maketitle
\se{早期量子现象}
\sub{光电效应} 
\sub{Compton效应}
\sub{氢原子光谱}
\sub{Franck-Hertz实验}
\sub{Stem-Gerlach实验}
\sub{Wien辐射定律}
\sub{Rayleigh-Jeans辐射定律}
\sub{Planck辐射定律}

\se{波粒二象性}
\sub{De Broglie波}
De Broglie假说: 和光子一样, 物质微粒也具有波动性. 
\sub{de Broglie关系}
$$\begin{aligned}
E &= h\nu = \hbar \omega \\
{p} &= \hbar {k}
\end{aligned}$$
\sub{Young双缝实验}
\sub{物质波的衍射}
\sub{波函数}
波函数$\psi(\bm{r},t)$满足的条件:
\rk{
    \item 平方可积, 归一性条件$$\int\abs{\psi(\bm{r},t)}^2\d^3r=1$$
    \item 任意次可微
}
\sub{波函数的统计解释}
\sub{Schrödinger方程}
$$i\hbar\pt{}\psi(\bm{r},t)=-\f{\hbar^2}{2m}\laplace\psi(\bm{r},t)+V(\bm{r},t)\psi(\bm{r},t)$$
\sub{连续性方程}
\sub{叠加原理}
由于Schrödinger方程对$\psi$是线性的, 叠加原理成立, 再加上概率幅的解释, 就能给出波动型的结果. 
\sub{自由粒子}
若粒子在空间各点$V=C$, 则粒子未受力的作用, 我们说它是自由的.
\sub{波包}
\sub{自由波包的时间演化}

\se{力学量的表述}
\sub{线性算符}
线性算符$A$定义
$$\hua{
    \kt{\psi'} &= A\kt{\psi}\\
    A[\l_1\kt{\psi_1}+\l_2\kt{\psi_2}] &= \l_1A\kt{\psi_1}+\l_2A\kt{\psi_2}
}$$
\sub{厄密算符}
厄密算符$A\T$定义: 
$$\kt{\psi'}=A\kt{\psi}\iff \br{\psi}=\br{\psi}A\T$$
推论:
\rk{
    \item 厄米算符$A$的本征值都是实数. 
    \item $A$向左作用: 当$\br{\vp}$是$A$的本征矢, 对于任意$\kt{\psi}$均有
    $$\bkt{\psi|A|\vp} = \l_{\psi}\bkt{\psi|\vp}$$
    \item 厄米算符两个互异本征值的本征矢互相正交. 
}
$$\boxed{
    \ar{
        \br{\psi}A\T\kt{\vp}\x\br{\vp}A\kt{\psi}^*\\
        \br{A\psi}\x\br{\psi}A\T\\
        \bkt{A\T\vp|\psi}\x\bkt{\vp|A\psi}\\
        \of{\kt{u}\br{v}}\T\x\kt{v}\br{u}
}}$$
\sub{位置算符}

\sub{动量算符}
\sub{动能算符}
\sub{角动量算符}
\sub{哈密顿算符}
\sub{量子力学中的基本对易关系}
定义对易子:
$$[A,B]=AB-BA$$
\sub{正则量子化}
\sub{Heisenberg不确定性原理}
\sub{能量-时间不确定性原理}
\sub{位置算符、动量算符、角动量算符的本征值和本征函数}

\se{不含时标量势场中粒子的运动}
\sub{空间变量与时间变量的分离}
\sub{定态}
\sub{定态的叠加}
\sub{一维方势场}

\se{量子力学的数学工具}
\sub{单粒子波函数空间}
$L^2$为所有平方可积函数的集合, 称由$L^2$中充分正规函数 (归一化, 可微等) 构成的波函数集合 (空间) 为$\mathscr{F}$. 
\sub{波函数空间的结构}
$\mathscr{F}$是一个矢量空间. 
\sub{标量积}
定义内积
$$(\vp,\psi)=\int\vp^*(r)\psi(r)\d^3r$$
内积与第二个因子线性, 与第一个因子反线性:
$$\hua{
    (\vp,\psi)\x(\psi,\vp)^*\\
    (\vp,\l_1\psi_1+\l_2\psi_2)\x\l_1(\vp,\psi_1)+\l_2(\vp,\psi_2)\\
    (\l_1\vp_1+\l_2\vp_2,\psi) \x \l_1^*(\vp_1,\psi)+\l_2^*(\vp_2,\psi)\\
}$$
\sub{离散正交归一基底}
正交归一基定义:\\
设可列函数集合$\{u_i(r)\}\in\mathscr{F}$, 当
$$(u_i,u_j)=\de_{ij}$$
且任意函数$\psi(r)\in\mathscr{F}$可按$u_i(r)$展开
$$\psi(r)=\sum c_iu_i(r),\quad c_i=(u_i,\psi)$$
则$\{u_i(r)\}$是一个正交归一基. 
\sub{态空间}
\sub{Dirac符号}
\sub{左矢与右矢}

\sub{表象的定义}
\sub{正交归一关系}
\sub{封闭性关系}
$$\boxed{
    \hua{
        P_{u_i}\x \sum_i\kt{u_i}\br{u_i}\x \mathbbm{1}\\
        P_{w_a}\x \int\kt{w_a}\br{w_a}\d a \x \mathbbm{1}\\
    }
}$$
其含义为将任意$\kt{\psi}$向空间的基投影, 得到其自身. 
\sub{左矢的表示}
\sub{右矢的表示}
\sub{算符的表示}
\sub{表象变换}
变换基$\kt{u_i} \to \kt{t_k}$的变换矩阵为
$$\ar{
    S_{ik}\x \bkt{u_i|t_k}\\
    (S\T)_{ki}\x (S_{ik})^*
}$$
\sub{右矢分量的变换}
由右矢在旧基中的分量得到新基中的分量:
$$\bkt{t_k|\psi}=\sum_i S_{ki}\T\bkt{u_i|\psi}$$
\sub{左矢分量的变换}
由右矢在旧基中的分量得到新基中的分量:
$$\bkt{\psi|t_k}=\sum_i \bkt{\psi|u_i}S_{ik}$$
\sub{算符矩阵元的变换}
$$\bkt{t_k|A|t_l}=\sum_{i,j} \bkt{t_k|u_i}\bkt{u_i|A|u_j}\bkt{u_j|t_l}$$
或写作
$$A_{kl}=\sum_{i,j}S\T_{ki}A_{ij}S_{jl}$$
\sub{可观测量}
可观察量用观察算符描述. 一个厄米算符$A$的互异本征值是正交的, 通过选择, 总可以让每一个相同本征值的子空间的各个本征矢也是正交的. \\
按定义, 如果本征矢的厄米算符$A$的正交归一系构成一个基, 则厄米算符$A$构成一个观察算符. 构成基可用封闭性关系式描述. 
\sub{可观测量完全集}
定理I\\
如果两个算符$A$和$B$是对易的, 且$\kt{\psi}$是$A$的一个本征矢, 则$B\kt{\psi}$也是$A$的本征矢, 即$A$的本征子空间在$B$的作用下不变. 

定理II\\
如果两个观察算符$A,B$是对易的, 且$\kt{\psi_1},\kt{\psi_2}$是$A$的不同本征值的两个本征矢, 那么$\bkt{\psi_1|B|\psi_2}=0$.

定理III(基本定理)\\
如果两个观察算符$A, B$是对易的, 则$A,B$的共同本征矢构成态空间的一个正交归一基. 

ECOC\\
若$A, B, \dots$的共同本征矢构成一个正交归一基, 则$A, B, \dots$构成一个ECOC. 此时, 1. $A, B, \dots$是两两对易的. 2. 给出了全体$A, B, \dots$的本征值的一个数组, 便足以决定唯一的共同本征矢. \\
两个典型的ECOC: $\{X,Y,Z\},\ \{P_x,P_y,P_z\},\ \{X,P_y,P_z\}$. 
\sub{坐标$\{\kt{\bm{r}}\}$表象与动量$\{\kt{\bm{p}}\}$表象}
引入两个特殊的基:
$$\ar{
    \xi_{\bm{r_0}}(\bm{r})\x\delta(\bm{r}-\bm{r}_0)\\
    v_{\bm{p_0}}(\bm{r})\x (2\pi\hbar)^{-3/2}e^{\f{i}{h}\bm{p}_0\cdot \bm{r}}
}$$
利用封闭性关系式, 可以计算得到
$$\ar{
    \bkt{\bm{r}_0|\psi} \x \psi(\bm{r}_0)\\
    \bkt{\bm{p}_0|\psi} \x \bar{\psi}(\bm{p}_0)\\
}$$
变换表象$\{\kt{\bm{r}}\} \to \{\kt{\bm{p}}\}$需要用到下面的数:
$$\bkt{\bm{r}|\bm{p}}=\bkt{\bm{p}|\bm{r}}^*=(2\pi\hbar)^{-3/2}e^{\f{i}{h}\bm{p}_0\cdot \bm{r}}$$
在$\{\kt{\bm{r}}\}$表象中, 
$$\boxed{\bm{P}=\f{\hbar}{i}\nabla}$$
对两个表象的各个分量:
$$\ar{
    [X,P_x] \x i\hbar\\
    [X,Y] \x 0\\
    [P_x,P_y] \x 0
}$$
右矢$\{\kt{\bm{r}}\}$是算符$X, Y, Z$的本征右矢, 对于动量同理. 
\sub{Schwarz不等式}
对于态空间$\mathscr{E}$中的任意右矢, 
$$\bkt{\psi|\psi}=\text{实数}\geq 0$$
这可以导出施瓦兹不等式:
$$\abs{\bkt{\psi_1|\psi_2}}^2\leq\bkt{\psi_1|\psi_1}\bkt{\psi_2|\psi_2}$$
\sub{幺正算符}
幺正算符的定义:
$$U\T U=UU\T=\mathbbm{1}$$
和$U$相联系的幺正变换可以保持$\mathscr{E}$空间中的内积不变. 即
$$\bkt{\wv{\psi}_1|\wv{\psi}_2}=\bkt{\psi_1|U\T U|\psi_2}=\bkt{\psi_1|\psi_2}$$ 
若$A$是厄米算符, 则$e^{iA}$是幺正算符. \\
两个幺正算符的乘积也是幺正的. \\
算符$U$为幺正的充分必要条件: $U$将$\mathscr{E}$中的正交归一基变为另一正交归一基. \\
幺正矩阵: 一列元素与另一列的对应元素的共轭复数的乘积之和为0. \\
幺正算符的本征值模为1, 且互异本征值的两个本征矢是正交的. 

算符的幺正变换:
定义$A$的变换$\wv{A}$是这样一个算符: 它在基$\{\kt{\wv{v}_i}\}$的矩阵元与$A$在$\{\kt{v_i}\}$的矩阵元对应. 即
$$\bkt{\wv{v}_i|\wv{A}|\wv{v}_j} = \bkt{v_i|A|v_j}$$
则
$$\wv{A}=UAU\T$$
$$\ar{
    \of{\wv{A}}\T \x \wv{A\T}\\
    \of{\wv{A}}^n \x \wv{A^n}\\
    \wv{F}(A) \x F\of{\wv{A}}
}$$
\sub{宇称算符}
宇称算符的定义:
$$\Pi\kt{\bm{r}}=\kt{-\bm{r}}$$
$$\bkt{\bm{r}|\Pi|\psi}=\psi(-\bm{r})$$
宇称算符下描述的体系是关于原点与原体系对应的体系. 宇称算符是一个幺正算符\\
性质:\\
$$\Pi=\Pi\m=\Pi\T$$
$\Pi$的本征值只能为$\pm1$, 这两个本征值是简并的. 属于$1$的本征矢是偶性本征矢, 属于$-1$的本征矢是奇性本征矢.\\
利用算符
$$\ar{
    P_+=\ff{2}(\mathbbm{1}+\Pi)\\
    P_-=\ff{2}(\mathbbm{1}-\Pi)
}$$
可以将任意$\kt{\psi}$分解为分别属于偶性$(+)$和奇性$(-)$的本征矢. \\
将任意算符做变换
$$\wv{B}=\Pi B \Pi$$
若$\wv{B}=B$, 则$B$为偶算符, $B_+$与$\Pi$对易, 若$\wv{B}=-B$, 则$B$为奇算符, $B_-$与$\Pi$反对易.\\
奇偶算符的性质:\\
一个偶算符在宇称相反的矢量之间的矩阵元为零:$\bkt{\vp|B_+|\psi}=0$. \\
一个奇算符在宇称相同的矢量之间的矩阵元为零:$\bkt{\vp'|B_-|\psi'}=0$. \\
特别地, 当$\kt{\psi}$具有确定的宇称, $\bkt{\psi|B_-|\psi}=0$.

例子:\rk{
    \item $\bm{R}$是奇算符. 
    \item $\bm{P}$是奇算符.
    \item $\Pi$是偶算符. 
}
\sub{张量积的定义与性质}
\sub{直积空间中的本征值方程}

\se{量子力学的假设}
\sub{体系状态的描述}
第一个假定: 在确定的时刻$t_0$, 一个物理体系的态由态空间$\mathscr{E}$中一个特定的右矢来确定. 
\sub{物理量的描述}
第二个假定: 每一个可以测量的物理量$\mathscr{A}$都可以用在$\mathscr{E}$空间中起作用的一个算符$A$来描述, 这个算符是一个观察算符. 
\sub{物理量的测量}
第三个假定: 每次测量物理量$\mathscr{A}$, 可能得到的结果只能是对应的观察算符$A$的本征值之一. 
第四个假定
\rk{
    \item 非简并的离散谱的情况: 若体系处于已归一化的态$\kt{\psi}$中, 则测量物理量$\mathscr{A}$得到的结果为对应观察算符$A$的非简并本征值$a_n$的概率$\mathscr{P}(a_n)$是:
    $$\mathscr{P}(a_n)=\abs{\bkt{u_n|\psi}}^2$$
    \item 离散谱的情况: 若体系处于已归一化的态$\kt{\psi}$中, 则测量物理量$\mathscr{A}$得到的结果为对应观察算符$A$的本征值$a_n$的概率$\mathscr{P}(a_n)$是:
    $$\mathscr{P}(a_n)=\sum_{i=1}^{g_n}\abs{\bkt{u^i_n|\psi}}^2$$
    \item 非简并连续谱的情况: 测量已处于已归一化的态$\kt{\psi}$的物理量$\mathscr{A}$时, 得到介于$\a$和$\a+\d\a$之间的结果的概率$\d\mathscr{P}(\a)$是
    $$\d\mathscr{P}(\a) = \abs{\bkt{v_\a|\psi}}^2\d\a$$
}
重要后果:\\ \\
\fbox{
\parbox{\textwidth}{
    \begin{center}
        互为比例的两个态矢量表示同一个物理状态, 总的相位因子对于物理预言没有影响, 但展开式中各项系数的相对相位则是有影响的
    \end{center}
}}
第五个假定: 如果对于处在$\kt{\psi}$态的体系测量物理量$\mathscr{A}$得到的结果是$a_n$, 则刚测量之后体系的态是$\kt{\psi}$在属于$a_n$的本征子空间上的归一化投影$\displaystyle \f{P_n\kt{\psi}}{\sqrt{\bkt{\psi|P_n|\psi}}}$.
\sub{体系随时间的演化}
第六个假定: 态矢量$\kt{\psi(t)}$随时间的演变遵从\sch 方程.
$$i\hbar\dt{}\kt{\psi(t)}=H(t)\kt{\psi(t)}$$
\sub{量子化规则}
规则的陈述: \\
首先考虑处在标量势场中的一个无自旋粒子构成的体系, 这时, 我们有下述规则\\
与粒子位置$\bm{r}(x,y,z)$相联系的是观察算符$\bm{R}(X,Y,Z)$. \\
与粒子动量$\bm{p}(p_x,p_y,p_z)$相联系的是观察算符$\bm{P}(P_x,P_y,P_z)$.\\
任意一个物理量$\mathscr{A}$都可以表示为$\bm{r},\bm{p}$的函数$\mathscr{A}(\bm{r},\bm{p},t)$. 要得到观察算符$A$, 将$\bm{r},\bm{p}$替换为$\bm{R},\bm{P}$即可: 
$$A(t)=\mathscr{A}(\bm{r},\bm{p},t)$$
\sub{测量过程}
\sub{给定态中可观测量的平均值}
平均值的定义
$$\bkt{A}_\psi=\bkt{\psi|A|\psi}$$
在实际计算中通常要在一个确定的表象中计算, 例如:
$$\bkt{x}_\psi=\bkt{\psi|X|\psi}=\int\d^3r \psi^*(r)x\psi(r)$$
\sub{方均根偏差}
方均根偏差$\D A$的定义是: 
$$\D A=\sqrt{\bkt{(A-\bkt{A})^2}}=\sqrt{\bkt{A^2}-\bkt{A}^2}$$
可以得到不确定度关系式:
$$\D X\cdot \D P_x \geq \f{\hbar}{2}$$
\sub{可观测量的相容性}
考虑两个对易的观察算符$A,B$, 存在一个由$A,B$的共同本征右矢构成的基$\kt{a_n,b_p,i}$, 对于这样一个态, 测量$A$一定得到$a_n$而测量$B$一定得到$b_p$, 像$A,B$这样可以同时完全确定的可观察量叫做相容的可观察量. \\
如果两个可观察量是相容的, 那么测量顺序是无关紧要的. 而两个不相容的可观察量是不能同时测量的, 且第二次测量会使第一次测量所得信息失去. 
\sub{叠加原理}
由于\sch 方程的解是线性齐次的, 因此它的解是可以线性叠加的. 
\sub{概率守恒}
\rk{
    \item 态矢量的模方保持为常数. 
    \item 局域守恒: 概率密度和概率流
}
\sub{概率幅与干涉效应}
概率幅的概念
\rk{
    \item 量子理论中概率型预言均得自概率幅, 计算时要取它模的平方.
    \item 在一个确定的实验中, 如果没有进行中间状态的测量, 那么我们绝不能根据中间测量可能得到的各种结果的概率, 而应根据它们的概率幅来分析问题.
    \item 一个物理体系的态可以线性叠加, 这意味着一个概率幅往往表现为若干部分幅之和. 因而对应的概率等于若干项之和的模的平方, 而且那些部分幅是彼此相干的. 
}
要计算一个末态的概率, 必须:\\ \\
\fbox{
\parbox{\textwidth}{
    \begin{center}
        将对应同一末态的诸概率幅相加, 然后将对应于正交末态的诸概率相加. 
    \end{center}
}}
\sub{密度算符}
\subsubsection{纯态}
当体系的态是完全确定的, 这是我们说体系处于纯态, 此时用态矢量$\kt\psi$或在态空间起作用的密度算符描述体系是完全等价的. \\
引入密度算符, 其定义是:
$$\r(t)=\kt\psi\br\psi$$
在基$\kt{u}$中, 密度算符是用一个矩阵描述的, 称为密度矩阵, 它的矩阵元是
$$\r_{pn}(t)=c_n^*(t)c_p(t)$$
使用概率算符, 概率守恒变为
$$\Tr\r(t)=1$$
可观察量的平均值变为
$$\bkt{A}(t)=\Tr\{A\rho(t)\}=\Tr\{\rho(t)A\}$$
\sch 方程变为
$$i\hbar\dt{}\r(t)=[H(t),\r(t)]$$
在纯态下, 描述同一物理状态的两个态矢量$\kt{\psi(t)},e^{i\t}\kt{\psi(t)}$对应着同一个密度算符, 可以避免相位因子带来的麻烦, 且上面各式对密度算符是线性的. 最后, 列出密度算符的一些其他性质:
$$\ar{
    \r\T(t)\x\r(t)\\
    \r^2(t)\x\r(t)\\
    \Tr\r^2(t)\x1
}$$
后面两式来源于$\r(t)$是投影算符这一事实, 这两式只在纯态成立.
\subsubsection{统计混合态}
密度算符的定义
考虑这样一个体系, 与它有关的概率可以在满足概率和为一的条件下任意取值, 假设态矢量为$\kt{\psi}$, 则概率为
$$\cm{P}_k(a_n)=\bkt{\psi_k|P_n|\psi}$$ 
要得到所求的概率$\cm{P}(a_n)$, 则要以$p_k$为权重去乘$\cm{P}_k(a_n)$, 再对$k$相加
$$\cm{P}(a_n)=\sum_kp_k\cm{P}_k(a_n)$$

\sub{时间演化算符}
\sub{Schrödinger、Heisenberg、相互作用图像}
\sub{规范不变性}
\sub{Schrödinger方程的传播子}
\sub{不稳定能级}
\sub{寿命}
\sub{任意形状势阱中粒子的束缚态}
\sub{任意形状势阱或势垒中粒子的非束缚态}
\sub{一维周期性结构中粒子的量子性质}

\se{自旋1/2与双能级体系}
\sub{可观测量Sz与自旋态空间}
\sub{其它自旋可观测量}
\sub{各种自旋状态的实际制备}
\sub{自旋测量}
\sub{均匀磁场中自旋1/2的演化}
\sub{耦合对双能级体系定态的影响}
\sub{双能级体系在两个非扰动状态之间的振荡}
\sub{Pauli矩阵}
\sub{矩阵的对角化}
\sub{与双能级体系相联系的虚拟自旋1/2}
\sub{两个自旋1/2粒子的体系}
\sub{自旋1/2密度矩阵}
\sub{静磁场与旋转磁场中的自旋1/2粒子:磁共振}

\se{一维谐振子}
\sub{谐振子在物理中的重要性}
\sub{经典力学中的谐振子}
\sub{量子力学哈密顿量的一般性质}
\sub{能谱的确定}
\sub{本征值的简并性}
\sub{哈密顿量的本征态}
\sub{与定态相联系的波函数}
\sub{基态的性质}
\sub{平均值的时间演化}
\sub{位置表象中的定态研究}
\sub{Hermite多项式}
\sub{用多项式方法求解谐振子的本征值方程}
\sub{动量表象中的定态研究}
\sub{三维各向同性谐振子}

\sub{均匀电场中的带电谐振子}
\sub{无限长耦合谐振子链的振动模}
\sub{谐振子的准经典相干态}
\sub{连续物理体系的振动模}
\sub{处于有限温度下热力学平衡态的谐振子}

\se{量子力学中角动量的一般性质}
\sub{定义与符号}
\sub{和本征值}
\sub{标准表象}
\sub{的本征值}
\sub{球谐函数}
\sub{角动量与转动}
\sub{双原子分子的转动}
\sub{二维谐振子在定态中的角动量}
\sub{磁场中的带电粒子:Landau能级}

\se{中心势场中的粒子 氢原子}
\sub{分离变量}
\sub{中心势场中粒子的定态}
\sub{经典力学中的质心运动与相对运动}
\sub{量子力学中的分离变量}
\sub{氢原子的Bohr模型}
\sub{氢原子的量子理论}

\sub{类氢体系}
\sub{各向同性三维谐振子}
\sub{与氢原子定态相联系的概率流}
\sub{均匀磁场中的氢原子}
\sub{顺磁性与抗磁性}
\sub{Zeeman效应}
\sub{双原子分子的振动-转动能级}

\se{散射的量子理论}
\sub{碰撞现象的重要性}
\sub{势散射}
\sub{散射截面的定义}
\sub{稳定散射态的定义}
\sub{利用概率流计算散射截面}
\sub{散射积分方程}
\sub{Born近似}
\sub{分波法原理}

\sub{自由粒子的定态}
\sub{势场中的分波}
\sub{散射截面的相移表达式}
\sub{自由粒子的具有确定角动量的定态}
\sub{弹性散射截面}
\sub{吸收截面}
\sub{总截面}
\sub{光学定理}
\sub{Yukawa势的Born近似}
\sub{硬球上的低能散射}

\se{电子自旋}
\sub{实验证据}
\sub{量子描述: Pauli理论的假设}
\sub{角动量1/2的特殊性质}
\sub{可观测量与态矢量}
\sub{物理测量概率的计算}
\sub{自旋1/2粒子的转动算符}

\se{角动量的合成}
\sub{经典力学中的总角动量}
\sub{量子力学中总角动量的重要性}
\sub{两个自旋1/2的合成}
\sub{任意两个角动量的合成}
\sub{角动量合成的例子}
\sub{Clebsch-Gordan系数}
\sub{球谐函数的合成}
\sub{Wigner-Eckart定理}

\se{定态微扰理论}
\sub{本征值方程的近似解}
\sub{能量的一级修正}
\sub{本征矢的一级修正}
\sub{能量的二级修正}
\sub{本征矢的二级修正}
\sub{简本能级的微扰}
\sub{van der Waals力}

\sub{体积效应:原子核的空间延展性对原子能级的影响}
\sub{变分方法}
\sub{固体中电子的能带:简单模型}
\sub{化学键的简单例子:离子}

\se{氢原子的精细与超精细结构}
\sub{精细结构哈密顿量}
\sub{与质子自旋相关的磁相互作用:超精细哈密顿量}
\sub{能级的精细结构}
\sub{能级的超精细结构}
\sub{ 基态超精细结构的Zeeman效应}
\sub{氢原子的Stark效应}

\se{与时间有关问题的近似方法}
\sub{表象中的Schrödinger方程}
\sub{微扰方程}
\sub{时刻体系的状态}
\sub{跃迁概率}
\sub{正弦微扰}
\sub{态密度}
\sub{Fermi黄金规则}
\sub{原子与电磁波的相互作用}
\sub{共振微扰作用下体系在两离散态之间的振动}
\sub{与连续终态共振耦合的离散态的衰变}

\se{全同粒子体系}
\sub{全同粒子的定义}
\sub{经典力学中的全同粒子}
\sub{量子力学中的全同粒子}
\sub{两粒子体系}
\sub{任意数目粒子体系}
\sub{对称化假设}
\sub{交换简便的消除}
\sub{物理右矢的构造}
\sub{其它假设的应用}
\sub{Pauli不相容原理}
\sub{独立全同粒子体系的基态}
\sub{量子统计}
\sub{直接与交换过程之间的干涉}
\sub{可忽略对称化假设的情形}
\sub{多电子原子}
\sub{电子结构}
\sub{氦原子的能级}
\sub{电子气的物理性质}
\sub{在固体上的应用}
\end{document}