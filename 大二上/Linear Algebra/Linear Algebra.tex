\documentclass[UTF8,9pt]{ctexart}
\usepackage{../template/Notes/notes}
\renewcommand\arraystretch{1}
\title{Linear Algebra}
\begin{document} 
\maketitle
\section{Words}
\begin{center}
	\begin{supertabular}{rl}
		subtract&减\\
		parenthesis/\,parentheses&括号\\
		reverse&反向\\
		pivot&主元\\  
		span&张成\\ 
		basis&基\\  
		dimension&维数\\
		entry&项\\  
		components&分量\\ 
		elimination&消元\\
		principal diagonal&主对角线\\
		inverse/\,invertible&逆\\
		the Identity&单位矩阵\\
		singular&奇异的,无逆矩阵\\
		determinant&行列式\\
		augmented matrix&增广矩阵\\
		reduced row echelon&简化阶梯形矩阵\\
		transpose&转置\\
		colinear&共线\\
		dependent&线性相关\\
		orthogonal&正交\\
		symmetric&对称\\
		equivalent&等价\\
		construct&求出解\\
		the nullspace&零空间,齐次解构成的空间\\
		factorization-&\\
		of a matrix&矩阵分解\\
		permutation&置换\\
		the origin&原点\\
		pivot column&主列\\
		sovlable&有解\\
		full rank&满秩\\
		Algebra cofactor&代数余子式\\
		general solution&通解\\
		particular solution&特解\\
		orthogonal basis&正交基\\
		orthonormal basis&标准正交基\\

		&\\

		forall&$\forall$\\
		there exists&$\exists$\\
		perpendicular&$\perp$\\
		&\\
		I&Identity,单位矩阵\\
		F&free,自由变量解矩阵\\
		L&lower triangular,下三角矩阵\\
		D&diagonal,对角矩阵\\
		U/R&upper triangular,上三角矩阵\\
		E&elimination,消元矩阵\\
		P&permutation,置换矩阵\\
		P&projection,投影矩阵\\
		R&rectangualr,长方阵\\
		Q&orthogonal,正交矩阵\\
		N&nullspace, 零空间矩阵,\\
		&(Ax=0的通解)$\begin{bmatrix}
				-F \\
				I
			\end{bmatrix}$\\
		R&reff form,简化阶梯形矩阵 $\begin{bmatrix}
				I & F \\
				0 & 0
			\end{bmatrix}$\\
	\end{supertabular}
\end{center}
\section{Matrix}
\begin{itemize}
	\item 矩阵的逆:\\
		\begin{enumerate}
			\item 用可逆矩阵与任意A相乘,不改变A的秩。\\
			\item $(A^T)^{-1}=(A^{-1})^T$\\
			\item 上三角阵的逆仍为上三角阵。\\
			\item 任何方阵可以表示为上三角阵和下三角阵的乘积。\\
		\end{enumerate}
		$$\begin{array}{rcl}
			\begin{bmatrix}
				a & b \\
				c & d
			\end{bmatrix}^{-1}
			& = &
			\dfrac{1}{|A|}\begin{bmatrix}
				d  & -b \\
				-c & a
			\end{bmatrix} \\
			\begin{bmatrix}
				a & b & c \\
				d & e & f \\
				g & h & i
			\end{bmatrix}^{-1}
			& = &
			\dfrac{1}{|A|}\begin{bmatrix}
				ei-hf & -(bi-hc) & bf-ce \\
				fg-id & -(cg-ia) & cd-af \\
				dh-ge & -(ah-gb) & ae-bd
			\end{bmatrix} \\
			\begin{bmatrix}
				1 & 0 \\
				c & 1
			\end{bmatrix}^{-1}
			& = &
			\begin{bmatrix}
				1  & 0 \\
				-c & 1
			\end{bmatrix}               \\
			\begin{bmatrix}
				1 & 0 & 0 \\
				d & 1 & 0 \\
				g & h & 1
			\end{bmatrix}^{-1}
			& = &
			\begin{bmatrix}
				1    & 0  & 0 \\
				-d   & 1  & 0 \\
				dh-g & -h & 1
			\end{bmatrix}              \\
			\begin{bmatrix}
				A_1 & A_3 \\
				0   & A_2
			\end{bmatrix}^{-1}
			& = &
			\begin{bmatrix}
				A_1^{-1} & -A_1^{-1}A_3A_2^{-1} \\
				0        & A_2^{-1}
			\end{bmatrix}
		\end{array}$$
	\item 对于正交矩阵(例如置换矩阵P):
		$$P^{-1}=\f{P^T}{|P|}$$
	\item Schwavz inequality:
		$$|u\cdot v| \leq ||u||\cdot ||v||$$
	\item $x_{complete}=x_{particular}+cx_{special}$
		,解空间由零空间$cx_{special}$平移$x_{particular}$后得到。   \\
	\item 对线性无关的方程组/矩阵$A_{j\times r}=\bm{a}_{j1},\cdots,\bm{a}_{jr}$:
		\begin{align*}
			& \dim \langle \bm{a}_{j1},\cdots,\bm{a}_{jr} \rangle \\
			= & \rank(A_{j\times r})                              \\
			= & \text{最大非零子式阶数}R(A_{j\times r})
		\end{align*}
	\item 想求零空间(x的解),只需要把A变为由I和自由列组成的最简行阶梯形式,然后特解会由I和-F构成。
	\item $AB$可由$A,B$线性表出,且                              \\
		$\rank(AB) \leq \min \{\rank(A),\rank(B)\}$             
	\item Sylvester秩不等式:设$A_{s\times n},\ B_{n\times m}$,则
		$$\rank(AB) \geq \rank(A) + \rank(B) -n$$
	
	\item 分块矩阵\\
		\begin{enumerate}[(i)]
			\item 初等行变换 - 左乘P:
				$\left(\begin{smallmatrix}
					A_1 & A_2\\
					A_3 & A_4
				\end{smallmatrix}\right)
				\Longrightarrow
				\left(\begin{smallmatrix}
					A_1      &      A_2\\
					PA_1+A_3 & PA_2+A_4
				\end{smallmatrix}\right)$\\
			\item 初等列变换 - 右乘P:
				$\left(\begin{smallmatrix}
					A_1  &  A_2\\
					A_3  &  A_4
				\end{smallmatrix}\right)
				\Longrightarrow
				\left(\begin{smallmatrix}
					A_1 & A_1P+A_2\\
					A_3 & A_3P+A_4
				\end{smallmatrix}\right)$
		\end{enumerate}
	\item 对矩阵$A,B,C$,有:
		$$\rank\begin{pmatrix}
				A & C \\
				0 & B
		\end{pmatrix}
		\geq
		\rank\begin{pmatrix}
			A & 0 \\
			0 & B
		\end{pmatrix}
		=\rank(A)+\rank(B)$$
\end{itemize}
\section{Determinant} 
\begin{itemize}
	\item $| A | = a _ { 11 } C _ { 11 } + a _ { 12 } C _ { 12 } + \ldots + a _ { 1 n } C _ { 1 n }$ 
	\item 逆矩阵公式
		$$A ^ { - 1 } = \frac { 1 } { | A | } C ^ { T }$$
	\item 范德蒙行列式:
		$$\left|\begin{matrix}
			1         & 1         & \cdots & 1         \\
			a_1       & a_2       & \cdots & a_n       \\
			a_1^2     & a_2^2     & \cdots & a_n^2     \\
			\vdots    & \vdots    & \ddots & \vdots    \\
			a_1^{n-1} & a_2^{n-1} & \cdots & a_n^{n-1}
		\end{matrix}\right|=\prod_{1\leq j < i \leq n}{(a_i-a_j)}$$
	\item $|AB|=|A|\cdot|B|=|BA|$
	\item Binet-Cauchy公式:设$A=(a_{ij})_{s\times n},B=(b_{ij})_{n\times s}$。\\
		\begin{enumerate}
			\item $s>n$,那么$|AB|=0$\\
			\item $s \leq n$,那么
				$$|AB|=\sum_{1 \leq v_1<\cdots<v_s\leq n}A
				\left(\begin{smallmatrix}
					1   &   2 & \cdots &  s\\
					v_1 & v_2 & \cdots & v_s
				\end{smallmatrix}\right)
				\cdot B
				\left(\begin{smallmatrix}
					v_1 & v_2 & \cdots & v_s\\
					1   &   2 & \cdots &   s
				\end{smallmatrix}\right)$$
		\end{enumerate}
	\item Cramer's rule\\
		求解$Ax=b$时,用$b$代替$A$的第$i$列得到$B_i$,则
		$$x_i=\f{|B_i|}{|A|}$$
\end{itemize}
\section{Vector Space $K^n$}
\begin{itemize}
	\item 替换定理:\\
		如果$\bm{a}_1,\cdots,\bm{a}_s$线性无关,$\bm{c}=b_1\bm{a}_1+\cdots+b_s\bm{a}_s$,使用$\bm{c}$替换$\bm{a}_i$,那么$\bm{a}_1,\cdots,\bm{a}_{i-1},\bm{c},\bm{a}_{i+1},\cdots,\bm{a}_s$也线性无关。\\
		$\bm{a}_1,\cdots,\bm{a}_s$线性无关$\iff \rank\{\bm{a}_1,\cdots,\bm{a}_s\}=s$\\
	\item 设n元齐次线性方程组解空间为W,则:
		$$\dim\ W=n-\rank(A)$$
	\item 若n元非齐次方程式有解,称其解集为U,其导出组的解集为W,则:
		$$U=\{\bm{\gamma}_0+\bm{\eta}|\bm{\eta}\in W\}=\bm{\gamma}_0+\bm{\eta}$$
		$\bm{\gamma}_0$称为$\textbf{特解}$,$\bm{\gamma}_0+\bm{\eta}$称为一个W型的$\textbf{线性流形}$
	\item 子空间维数公式
		$$\dim V_1+\dim V_2= \dim(V_1+V_2)+\dim(V_1 \cap V_2)$$
	\item 矩阵的秩在乘以一个非奇异矩阵时是不变的($\dim N(A)=0 \ip \rank(AB)=\rank (B)$),乘以奇异矩阵可以改变秩,下面的公式准确地显示了发生了多少变化:\\
		If $A$ is $m \times n$ and $B$ is $n \times p$, then
		$$\rank(AB) = \operatorname { rank } (B) -\dim N (A) \cap R (B) $$
		求$N (A) \cap R (B)$的方法:
		$$\begin{array} { rl } 
			& \text { Find a basis } \left\{ x _ { 1 } , x _ { 2 } , \ldots , x _ { r } \right\} \text { for } R ( B) \\
			&  \text { Set } X _ { n \times r } = \left( x _ { 1 } \left| x _ { 2 } \right| \cdots | x _ { r } \right)  \\
			&\text { Find a basis } \left\{ v _ { 1 } , v _ { 2 } , \ldots , v _ { s } \right\} \text { for } N (AX) \\ 
			&\mathcal { B }= \left\{ X v _ { 1 } , X v _ { 2 } , \ldots , X v _ { s } \right\} \text { is a basis for } N (A) \cap R (B)
		\end{array}$$
		当$\dim N (A) \cap R (B)$不易求出,仍有以下推论:
		$$\operatorname { rank } ( \mathbf { A B } ) \leq \min \{ \operatorname { rank } ( \mathbf { A } ) , \operatorname { rank } ( \mathbf { B } ) \}$$
		$$\operatorname { rank } ( \mathbf { A } ) + \operatorname { rank } ( \mathbf { B } ) - n \leq r a n k ( \mathbf { A B } )$$
\end{itemize}
\section{正交矩阵}
\begin{itemize}
	\item 矢量$q_1,q_2, \cdots,q_n$标准正交(orthonormal)如果
	$$q^T_iq_j=\delta_{i,j}$$
	\item 行空间与零空间正交,列空间与左零空间正交
		\putfig{0.7}{2.png}
	\item 对于标准正交矩阵$A$,设$A$的行向量为$\bm{\gamma}_n$,列向量为$\bm{\alpha}_n$,有:\\
		\begin{enumerate}[(i)]
			\item $AA^T=I$
			\item $A^{-1}=A^T$
			\item $\| Qx \| ^2 = \|x\| ^2$
			\item $|A|= \pm1$
			\item $\bm{\gamma}_i\bm{\gamma}_j^T = \delta_{i,j}$
			\item $\bm{\alpha}_i\bm{\alpha}_j^T = \delta_{i,j}$
			\item $A,B$均正交,则$AB$正交
		\end{enumerate}
	\item 投影矩阵:
		$$P=A(A^TA)^{-1}A^T$$
	\item 施密特正交化过程:设线性无关组$\bm{a}_s$, 令
		$$\ar[rcl]{
			\bm{b}_1 &= & \bm{a}_1                                                                   \\
			\bm{b}_2  &= & \bm{a}_2-\frac{\bm{a}_2\bm{b}_1}{\bm{b}_1\bm{b}_1}\bm{b}_1                 \\
					& \vdots &                                                                    \\
			\bm{b}_s & = & \bm{a}_s-\sum_{j=1}^{s-1}\frac{\bm{a}_s\bm{b}_j}{\bm{b}_j\bm{b}_j}\bm{b}_j
		}$$
		则可得到正交组$\bm{b}_s$
	\item 马尔科夫矩阵\\
		1 每个元素非负\\
		2 每列元素和为1\\
		3 特征值的模均小于1
	\item $A$的$QR$分解\\
	Every matrix $A_{m×n}$ with linearly independent columns can be uniquely factored as $A = QR$\\
	$R = Q^TA$ is upper triangular because later $q_i$'s are orthogonal to earlier $a$'s. The lengths of $A, B$ and $C$ are on the diagonals of $R$.\\
	可用QR分解解决最小二乘问题
	$$\mathbf { R } \mathbf { x } = \mathbf { Q } ^ { T } \mathbf { b }$$
	该方法不需要求$A^TA$,但当需要求解$A^TA$时,只需利用$\mathbf { A } ^ { T } \mathbf { A } = \mathbf { R } ^ { T } \mathbf { R }$计算即可(Cholesky factorization)。
\end{itemize}
\section{最小二乘法}
\begin{itemize}
	\item example 1\\
		拟合$n$个点$(a_n,b_n)$到直线$y=k\hat{x}+b$上:\\
		设$\hat{x}=(k,b)^T$\\
		则 $$A=\begin{pmatrix}
			a_1&1\\
			\vdots&\vdots\\
			a_n&1
		\end{pmatrix}=([a_n],[1])$$
		$$b=(b_1,\cdots,b_n)^T$$
		此时有
		$$A^TA\hat{x}=A^Tb$$
		b在A的列向量的投影:
		$$p=A\hat{x}$$
	\item example 2\\
		将矢量$b$投影到$\{a_1,a_2,a_3\}$张成的空间中,设投影得到的向量为$c$,求$c$:\\
		也即,我们考虑存在一个$c=\widehat {x}_{ 1 } \mathbf { a } _ { 1 } +\widehat { x } _ {2} \mathbf { a } _ {2} +\widehat { x _ { n } } \mathbf { a } _ { \mathbf { 3 } }$最接近$\bm{b}$,此时$c$即为$b$的投影.\\
		设$A=\left( a_1,a_2,a_3 \right), c=Ax=\widehat {x}_{ 1 } \mathbf { a } _ { 1 } +\widehat { x } _ {2} \mathbf { a } _ {2} +\widehat { x _ { n } } \mathbf { a } _ { \mathbf { 3 } }$,此时$\hat{x}$为$c$在基$\{a_1,a_2,a_3\}$下的坐标,那么$a_1, a_2,a_3$的线性组合可以被写作$A\hat{x}$。\\
		则
		$$Ax=c=Pb=A(A^TA)^{-1}A^Tb$$ 
		$$A^TAx=A^Tb$$
		注意到,我们所需的$A$矩阵即为以投影空间为列空间的矩阵,所需的$b$则为需要投影的向量,$c$即为目标向量。\\
		从$P$的计算式可以知道:$P$为对称矩阵。
	\item Proof: $x \in N(A) \iff x \in N(A^TA)$\\
		1. Suppose $\mathbf{x}\in N(A).\text{Then}\mathbf{x}\in N\left(A^{T}A\right).\text{ This is because }A\mathbf{x}=0\Rightarrow A^{T}A\mathbf{x}=0$\\
		2. Conversely, let $x\in N\left(A^{T}A\right).$ Then $A^{T}Ax=0 \ip x^{T}A^{T}Ax=0 \ip \|Ax\|^{2}=0.$ since the length of $Ax$ is zero, we have $Ax=0$. Therefore $x\in N(A)$.
	\item 特别地,当$A=Q$\\
		$$p= A \left( A ^TA \right) ^ { - 1 } A ^Tb \ip p = Q Q ^Tb$$
	\item 一般地,当拟合函数非线性时,例如,以$t$为横坐标,$b$为纵坐标,
		$$p ( t ) = \alpha _ { 0 } + \alpha _ { 1 } t + \alpha _ { 2 } t ^ { 2 } + \cdots + \alpha _ { n - 1 } t ^ { n - 1 }$$
		在平面上有$m$个点:
		$$\mathcal { D } = \left\{ \left( t _ { 1 } , b _ { 1 } \right) , \quad \left( t _ { 2 } , b _ { 2 } \right) , \ldots , \left( t _ { m } , b _ { m } \right) \right\}$$
		\putfig{0.5}{1.png}
		则取
		$$\mathbf { A } = \left( \begin{array} { c c c c c } { 1 } & { t _ { 1 } } & { t _ { 1 } ^ { 2 } } & { \cdots } & { t _ { 1 } ^ { n - 1 } } \\ { 1 } & { t _ { 2 } } & { t _ { 2 } ^ { 2 } } & { \cdots } & { t _ { 2 } ^ { n - 1 } } \\ { \vdots } & { \vdots } & { \vdots } & { \ldots } & { \vdots } \\ { 1 } & { t _ { m } } & { t _ { m } ^ { 2 } } & { \cdots } & { t _ { m } ^ { n - 1 } } \end{array} \right) , \quad \mathbf { x } = \left( \begin{array} { c c } { \alpha _ { 0 } } \\ { \alpha _ { 1 } } \\ { \vdots } \\ { \alpha _ { n - 1 } } \end{array} \right) , \quad \text { and } \quad \mathbf { b } = \left( \begin{array} { c c } { b _ { 1 } } \\ { b _ { 2 } } \\ { \vdots } \\ { b _ { m } } \end{array} \right)$$
\end{itemize}
\section{Norms}
\begin{itemize}
	\item P-Norms
	$$\| x \| _ { p } = \left( \sum _ { i = 1 } ^ { n } \left| x _ { i } \right| ^ { p } \right) ^ { 1 / p }\qquad (p>1)$$
\end{itemize}
\end{document}