\documentclass[UTF8,9pt]{ctexart}
\usepackage{../template/Notes/notes}
\title{Mathematical Methods for Physics I}
\begin{document}
    \maketitle

 
    \section{复数}
    
     
    \begin{itemize}
        \item De Moivre公式: 
            $(\cos\theta+i\sin\theta)^n=\cos n\theta+i \sin n\theta$
        \item $(z_1+z_2)^n=\sum C_n^k z_1^k z_2^{n-k}$ 
        \item $\sqrt[n]{z}=\sqrt[n]{|z|} e^{i\frac{\rm{arg}\,z+2k\pi}{n}}$
        \item $z_1 \perp z_2 \Leftrightarrow \Re(z_1\bar{z}_2)=0$
        \item $$\mathrm{arg}z=\left\{\  
            \begin{array}{lr}  
                \arctan\dfrac{y}{x},&x>0, y>0, or \ y \leq 0,\\
                \pi+\arctan\dfrac{y}{x},&x<0, y \geq 0,\\
                -\pi+\arctan\dfrac{y}{x},&x<0, y <0,
            \end{array}\right.$$
        \item $f(z)$和$\overline{f(z)}$都解析时,$f(z)$必然为常数。
        \item $f(z)$解析时,$\overline{f(\bar{z})}$也解析。
        \item $z_1 \perp z_2$的充要条件为:$\Re(z_1z_2)=0$
        \item 幂函数$z^{\frac{p}{q}}$是q值函数
        \item 幂函数$z^{a}$当a是无理数或复数时是无穷多值函数。
    \end{itemize}
\section{解析函数}
    C-R方程极坐标形式:
    $$\begin{array}{lr}
        \pp{u}{r}=&\frac{1}{r}\pp{v}{\theta}\\ 
        \pp{v}{r}=&-\frac{1}{r}\pp{u}{\theta}
    \end{array}$$
\section{初等解析函数}
    \begin{itemize}
        \item 指数函数 $\exp(z) \equiv e^z$\\
            $T=2k\pi i$
        \item 对数函数 $\Ln z \equiv \ln|z|+i\arg z+2k\pi i$\\
            对数函数主支:$\ln z \equiv \ln|z|+ i \arg z$\\
            NOTICE:\\
            $(e^x)^y \neq e^{xy}$\\
            $z\ln a \neq \ln a^z$
        \item 幂函数 $f=z^a$
            $$f=z^a\left\{ \begin{array}{llr}
                a = n                      & n   \in \mathbb{N} & \text{单值函数}\\
                a = \frac{1}{n}            & n   \in \mathbb{N} & \text{n值函数}\\
                a = \frac{n}{m}            & m,n \in \mathbb{N} & \text{n值函数}\\
                a = \mathbb{C}/\mathbb{Q}  &                    & \infty \text{值函数}\\
            \end{array}\right.$$
        \item 三角函数 $sin(z), cos(z)$\\
            $$\begin{array}{rl}
                \sin(z) \equiv& \dfrac{e^{iz}-e^{-iz}}{2}\\
                \cos(z) \equiv& \dfrac{e^{iz}+e^{-iz}}{2}\\
                sh(z) \equiv& \dfrac{e^{z}-e^{-z}}{2}\\
                ch(z) \equiv& \dfrac{e^{z}+e^{-z}}{2}\\
            \end{array}$$
            
    \end{itemize}
\end{document}