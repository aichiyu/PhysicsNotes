\documentclass[UTF8,9pt]{ctexart}
\usepackage{../../template/homeworkTEMP/hw} 
\setcounter{secnumdepth}{0}
    \title{The 5th Homework of Optics}  
\begin{document}
\maketitle
\section{4-5}
a. $\frac{I_0}{I}=(\sqrt{2})^2=2$\\
b. $\frac{I_0}{I}=(\sqrt{2})^2=2$\\
c. $\frac{I_0}{I}=(1/2)^2=1/4$\\
d. $\frac{I_0}{I}=(2-2/2)^2=1$\\
e. $\frac{I_0}{I}=(\sqrt{2^2+1})^2=5$\\
f. $\frac{I_0}{I}=(2 \times \f{3}{4}-2 + 3/4)^2=1/16$ 
\se{4-7}
$$I=(50 \times 2-1)^2=99^2$$
\se{4-9}
$$\rho_1^2=f\l \ip \rho=\f{2\sqrt{5}}{5} mm \approx 0.57mm$$
$$I/I_0=1000=(2k/2)^2 \ip k \approx 32$$
$$\rho_k=\sqrt{32 \times 800 \times 400\e{-6}}=3.2mm$$
有效半径为$3.2mm$
\se{4-10}
$$\rho_1^2=f\l \ip 900 \times 30 = 632.8f' \ip f'=43cm$$\
\se{4-11}
\sub{(1)}
    考虑$\t$影响:
    $$U=C\int_{-a/2}^{a/2}e^{ik\D r}\d x=C\f{e^{-ikx\sin\t}}{-ik\sin\t}\bigg|^{x=a/2}_{x=-a/2}$$
    代入$k\l=2\pi$,
    $$U=2C\f{\sin\a}{\a},\quad \a=\f{\pi a (\sin\t-\sin\t_0)}{\l}$$
    $$\ip I =U^2=I_0(\f{\sin\a}{\a})^2,\quad \a=\f{\pi a(\sin\t-\sin\t_0)}{\l}$$
\sub{(2)}
    零级中心即$\a=0$,即各光线无光程差,根据费马原理,此点即几何光学像点。
\sub{(3)}
    根据(1)中式子,暗斑出现在$\sin\a=0,a \neq 0$处,即$\a=k\pi(k\neq0)$。\\
    一级暗斑即零级半角宽,$\f{\pi a (\sin\t-\sin\t_0)}{\l}=\pi$
    $$\ip \sin(\D\t+\t_0)-\sin\t_0=\l/a \ip \cos\t_0 \cdot \D\t \approx \l/a$$
    $$\D\t=\f{\l}{\cos\t_0 a}$$
\sub{(4)}
    在衍射处发生折射,$n=\f{\sin\t}{\sin\t'}$,使用$\t'$代替上式所有$\t$即可,此时:\\
    (1)    
    $$\ip I =U^2=I_0(\f{\sin\a}{\a})^2,\quad \a=\f{\pi a(\sin(\t/n)-\sin(\t_0/n))}{\l}$$
    (3)
    $$\D\t=\f{\l}{\cos(\t_0/n) a}$$
\se{4-12}
\sub{(1)}
    反射
    $$\D\t=\l/D=0.6/10000*180/Pi*60*60=12.4''$$
    折射
    $$\D\t=\l/nD=0.6/10000*180/Pi*60*60=8.2''$$
\sub{(2)}
    反射
    $$\D\t=\l/D=0.6/10000*180/Pi*60*60/\cos75=47.4''$$
    折射
    $$\D\t=\l/nD=0.6/10000*180/Pi*60*60/\cos75=10.7''$$
\sub{(2)}
    反射
    $$\D\t=\l/D=0.6/10000*180/Pi*60/\cos89=66'40''$$
    折射
    $$\D\t=\l/nD=0.6/10000*180/Pi*60/\cos89=11'1''$$
\se{4-17}
\sub{(1)}
    取$\l=550nm$
    $$\de y=0.61\l/N.A.=0.25\mu m $$
\sub{(2)}
    $$V=\f{0.075}{0.25/1000}=290$$
\sub{(3)}
    $$V=-\f{s_0\D}{f_Of_E}\ip \D=290*1.91*50/250=111mm$$
\se{4-19}
$$\D\t=1.22\l/D=6.7\e{-7} \ip \D s=s\D\t=255m$$
\se{4-23}
$$U=C\int_{-a/2}^{a/2}\d x \int_{-b/2}^{b/2} e^{ik\D r}\d y$$
$$=C\int_{-a/2}^{a/2} e^{-ik\D x} \d x \int_{-b/2}^{b/2} e^{-ik\D y}\d y$$
$$=abC\f{e^{-ikx\sin\t}}{-ik\sin\t}\bigg|^{x=a/2}_{x=-a/2}\f{e^{-iky\sin\t}}{-ik\sin\t}\bigg|^{y=b/2}_{y=-b/2}$$
令
$$\a'=\f{ka}{2}(\sin\t-\sin\t_0)$$
$$\beta'=\f{kb}{2}(\sin\t-\sin\t_0)$$
$$\ip I=U^2=I_0(\f{\sin\a'}{\a'})^2(\f{\sin\beta'}{\beta'})^2$$
\se{4-25}
$$U=[U(0)+U(-d)+U(-3d)]\f{\sin\a}{\a}$$
$$U(-d)=U(0)\f{\sin\a}{\a}, \a=\f{\pi a \sin\t}{\l}=\f{\pi a d/f}{\l}$$
$$U=U_0\f{\sin\a}{\a}\sqrt{(1+\cos\beta+\cos3\beta)^2+(\sin\beta+\sin3\beta)^2}$$
$$I=U^2=I_0(\f{\sin\a}{\a})^2[3+2(\cos2\beta+\cos4\beta+\cos6\beta)]$$
\se{4-27}
\sub{(1)}
$$\ip I=U^2=I_0(\f{\sin\a}{\a})^2(\f{\sin\beta}{\beta})^2$$
\sub{(2)}
$$\ip I=U^2=I_0(\f{\sin\a}{\a})^2(\f{\sin\beta}{\beta})^2$$
\sub{(3)}
$$\ip I=U^2=4I_0\cos^22\a(\f{\sin\a}{\a})^2(\f{\sin\beta}{\beta})^2$$
\se{4-32}
\sub{(1)}
    $$k_{max}=d/\l=2$$
    $$\de \l=\l/kN=0.05nm$$
\sub{(2)}
    $$1/D_\t=1/\f{k}{d\cos\t_k}=0.244nm/(')$$
\sub{(3)}
    $$\t_b \approx12^{\circ}39'$$
    与闪耀方向与光栅法线角度相同。
\se{4-34}
$$\begin{array}{c|c|c|c}
        &\text{光栅}&\text{棱镜}&\text{F-P腔}\\
    \hline
    \text{角分辨}&3\e{4}&3\e{3}&6\e{5}\\
    \hline
    \text{角色散}&2.2'/nm&0.31'/nm&39'/nm\\
    \hline
    \text{自由光谱}&\text{一级}850nm-1700nm&--&\l=550nm:0.003nm
\end{array}$$
\end{document}