\documentclass[UTF8,9pt]{ctexart}
\usepackage{../../template/homeworkTEMP/hw}
\setcounter{secnumdepth}{0}
\title{原子物理}   
\begin{document} 
\maketitle 
\se{Q1}
设光具有能量$E_p=h\nu$,动量$P_p=E_p/c$,电子有能量$E_e$,动量$P_e$,吸收光子前后能量动量守恒.取使得电子初始速度$v=0$的参考系,则吸收光子后:
$$E=h\nu$$
$$P=E_p/c=h\nu/c$$
电子能动量关系为:
$$E^2=p^2c^2+m_0^2c^4$$
代入上式
$$m_0=0$$
即电子静止质量为0,显然不可能,则假设“自由电子吸收光子”不成立。
\se{Q2}
设B层电子跃迁到A层,并在C层产生Auger电子,各能级结合能为$\phi$, Auger电子动能为:
$$E_{Aug}=\phi_A-\phi_B-\phi_C$$
%产生Auger电子要求光子能量不能低于$\phi_C$,为了使光子与电子有足够作用时间,也不能高出$\phi_C$过多。\\
近似根据波尔理论,$$E=\f{Rhc}{n^2}$$
代入可得:
$$E_{Aug}=Rhc(\ff{n_A^2}-\ff{n_B^2}-\ff{n_C^2})$$
%,对于一般$n<4$的情况,E与Rhc同数量级,,随着Z增大结合能降低
需要
$$\ff{n_A^2}>\ff{n_B^2}+\ff{n_C^2}$$
这只有$n_A$较小时成立,$Z$较大时内层电子不易激发,因此对于$Z$小的原子$n_A$更小,更易产生Auger电子。
\se{Q3}
\sub{为什麽在光子能量较小时,Z大时光电效应比康普顿效应更占主导?}
    光子能量越小,速度越慢,同时$Z$越大,原子半径越大,二者都导致光子与电子作用时间变长,使光电效应更容易发生。
\sub{为什么康普顿散射几率随光子能量增高而变大而光电效应散射几率随光子能量增高而变小?}
    光子能量变高,与电子作用时间变短,光电效应更难发生。\\
    相对地,光子能量变高,光子在被康普顿散射降低速度后更不容易被光电效应吸收,也就更容易穿越整个靶使康普顿效应被观测到。
\sub{为什么电子偶产生必须有原子核参与?}
    对于自由光子其能量动量满足$E=Pc$,则其产生的自由电子也满足该式。由于电子能动量关系为:
    $$E^2=p^2c^2+m_0^2c^4$$
    可导出两个电子静止质量$m_0=0$,因此不可能是自由电子,必须有原子核参与。
\se{Q4}
对于Cu,$\rho=8.9g/cm^3$其透射率为:
$$I_1/I_0=e^{[-48 \times 8.9 x]}$$
对于Zn,$\rho=8.9g/cm^3$其透射率为:
$$I_2/I_0=e^{[-325 \times 8.9 x]}$$
$$\ip \f{I_1}{I_2}=e^{277 \times 8.9 x}$$
取$x=9.4\e{-4} cm$
$$\f{I_1}{I_2}=e^{0.0277 \times 9.4}=10$$
则二者之比为$10:1$
\se{Q5}
$$K_{\beta1}={3}^2P_{3/2} \rightarrow {1}^2S_{1/2},\quad J=3/2$$
$$K_{\beta2}={3}^2P_{1/2} \rightarrow {1}^2S_{1/2},\quad J=1/2$$
原子由某一激发态向基态或较低能级跃迁发射谱线的强度与激发态原子数成正比,处于热力学平衡状态时,单位体积基态原子数为:
$$N i = N _ { 0 } \frac { g _ { i } } { g _ { 0 } } e ^ { - \frac { E i } { k T } }$$
式中$g_i, g_0$为激发态与基态的统计权重,$g=2J+1$,$E_i$为激发能,$k$为玻尔兹曼常数,$T$为激发温度,由此得到发射谱线的强度公式:
$$I _ { i } = A _ { i j } h \nu _ { i j } N _ { 0 } \frac { g _ { i } } { g _ { 0 } } e ^ { - \frac { E i } { k T } }$$
$I$为光谱强度,$A_{ij}$为跃迁几率,$\nu_{ij}$为发射谱线的频率。\\
两种态除$g$外其他参数均大致相等,因此强度之比即为$2J+1$之比。
$$I_{\beta1}/I_{\beta2}=\f{2\times3/2+1}{2\times1/2+1}=2/1$$
\se{Q6}
\sub{a)确定其基态原子组态}
由洪特规则:原子组态为1-4层全满+$5s^2 5p^6 5d^{10} 6s^1=[Xe]4f^{14} 5d^{10} 6s^1$
\sub{b)其$K_{\a1}$特征峰的能量(不考虑屏蔽效应,但是考虑$K$壳层的空穴)}
$$K_{\a1} ={2}^2P_{3/2} \rightarrow {1}^2S_{1/2}$$
设原子核有势能$\phi_n$,空穴有势能$\phi_e$.
$$\D \phi_n=\D U_n+\D U_{l,s}=\ff{2}m_e(\a c)^2(1-\ff{2^2})-\f{(\a Z)^4E_0}{4\times24}=1761eV$$
\sub{c)当散射角为60度时,康普顿电子的反冲能量和动量。}
根据散射公式:
$$\D\l=\f{h}{m_0c}(1-\cos\t)$$
可求出电子动能$$E=\D h\nu=h\nu\f{\gamma(1-\cos\t)}{1+\gamma(1-\cos\t)},\ \gamma=\f{h\nu}{m_0c^2}$$
取国际单位:
$$\ip E=\frac{6.10501\times 10^{12} h^2\nu^2}{6.10501\times 10^{12} h\nu+1}$$
$$P=\sqrt{2mE}=3.34\e{-9} h\nu\sqrt{\frac{1}{6.10501\times 10^{12} h\nu+1}}$$
\end{document}