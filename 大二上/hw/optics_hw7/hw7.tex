\documentclass[UTF8,9pt]{ctexart}
\usepackage{../../template/homeworkTEMP/hw} 
\setcounter{secnumdepth}{0}
    \title{The 7th Homework of Optics}  
\begin{document}
\maketitle
\section{5-2}
设物距为$s$,
$$U_{in}=\f{Ae^{ikz}}{z}e^{ik\f{x^2+y^2}{2s}}$$
$$U_{out}=U_{in}U_L=\f{Ae^{ikz}}{z}e^{ik\f{x^2+y^2}{2s}}\times e^{-ik\f{x^2+y^2}{2F}}=\f{Ae^{ikz}}{z}e^{ik\f{x^2+y^2}{2s'}}$$
$$\ip \o_{in}=k\f{x^2+y^2}{2s},\quad \o_{out}=k\f{x^2+y^2}{2s'}$$
$$\ip V=\f{\o_{out}}{\o_{in}}=\f{s'}{s}$$
\se{5-5}
在显微镜下观察的微小物体可近似看作一个点,且物近似位于物镜的前焦点上,对于相干照明,系统的截止频率由物镜孔径限制的最大孔径角$u_0$决定,故
$$f _ { c } = \frac { \sin u_ { 0 } } { \lambda } $$
截止频率的倒数即为分辨距离,
$$\delta _ { c } = \frac { \lambda } { \sin y _ { 0 } }$$
非相干时
$$\delta _ { c } =0.61 \frac { \lambda } { \sin y _ { 0 } } $$
\se{5-6}
\sub{(1)}
40cm
\sub{(2)}
5cm内
\sub{(3)}
$6.3\e{-3}rad=21'42''$
\sub{(4)}
6.3mm
\se{5-7}
\sub{(1)}
大于5cm
\sub{(2)}
半角宽$1'42''$\\
零级宽0.78mm
\sub{(3)}
半角宽$1'42''$\\
零级宽0.39mm
\sub{(4)}
半角宽$1'42''$\\
零级宽0.78mm
\se{5-9}
$$I=I_0(\f{\sin N\beta}{\sin\beta})^2,\quad \beta=\f{\pi dx'}{\l F}$$
\se{5-10}
\sub{(1)}
$$\ar[r|l]{
    x',y' & f_x,f_y\\
    0,0&0,0\\
    0,1&0,2.8\\
    \sqrt{2}/2,\sqrt{2}/2&2.0,2.0\\
    0.5,2&1.4,5.6\\
    3,-5&8.3,-13.97\\
    -10,-15&-27.8,-41.7
}$$
\sub{(2)}
$$42m^{-1}$$
\se{5-11}
\sub{(1)}
形状为与y正交的等距细线
\sub{(2)}
物光波:$\phi_o=ky\sin\t_o+\phi_0$,参考光波:$\phi_R=-ky\sin\t_o$
\sub{(3)}
相距$2\pi$相位的两个点:
$$L=\l=2d\sin\t_o=2d\sin\t/2 \ip d=\f{\l}{2\sin\t/2}$$
\sub{(4)}
将数据代入上式,
$$d_1=36.26\mu m, d_{60}=0.6328\mu m$$
\sub{(5)}
满足布拉格条件,构成体全息图。\\
$d=0.6328\mu m$而底片最小分辨$1000\mu m/3000=0.3\mu m$,匹配。
\sub{(6)}
$$\t_0=-\t/2,\ \t_{+1}=-\t/2+\t=\t/2,\ \sin\t_{-1}=-\t/2-\t=3/2\sin\t$$
\se{5-12}
$$\t_0=0,\ \sin\t_{+1}=2\sin\t/2,\ \sin\t_{-1}=2\sin\t/2$$
\se{5-13}
+1级,$ (0,y_0/2,z/2)$,\\
-1级,无穷远
\end{document}