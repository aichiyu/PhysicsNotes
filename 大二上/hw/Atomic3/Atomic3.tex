\documentclass[UTF8,9pt]{ctexart}
\usepackage{../../template/homeworkTEMP/hw}
\setcounter{secnumdepth}{0}
\title{原子物理}   
\begin{document} 
\maketitle 
\section{Q1. 请计算He原子激发态1s,2p态在LS耦合下的可能原子组态,并计算各个组态下的g因子。}   
根据$S$不同,有2种情况:
$$\ar[ccc]{
    L=1,&S=0,&J=1\\
    L=1,&S=1,&J=0,1,2
    }$$
根据$g$计算式可得:
$$\ar[ccc]{
    L=1,&S=0,& J=1,\ g=1\\
    L=1,&S=1,&\left\{\ar{J=0,&g\text{不存在}\\
                        J=1,&g=1.5\\
                        J=2,&g=1.5}\right.\\
    }$$
\section{Q2. 请计算Pb原子电子组态6p,7s在JJ耦合下的可能原子组态,并计算各个组态下的g因子。}
$6p$电子可能为:
$$s=1/2,\ l=1,\ j_1=1/2,3/2,\ g=0.67,1.33$$
$7s$电子可能为:
$$s=1/2,\ l=0,\ j_2=1/2,\ g=2$$
$$g=\f{g_1+g_2}{2}+\f{(g_1-g_2)(\hat{j_1}^2-\hat{j_2}^2)}{2\hat{j}^2}$$
则总$g=$
$$\ar{\left(\ff{2},\ff{2}\right)&,\ g=1.335\\
\left(\f{3}{2},\ff{2}\right)&,\ g=1.4975}$$
\section{Q3. 请画出Ca的激发态4s4p 在LS耦合下的可能原子组态及对应的能量(相对于基态)。}

共有${}^1P_1,{}^3P_0,{}^3P_1,{}^3P_2$四种态,由洪特规则,$S=0$的态能力最高,且遵守正常次序,$J$较小的能量低,且$U_{{}^3P_0}-U_{{}^3P_1}:U_{{}^3P_1}-U_{{}^3P_2}=1:2$:
$$\ip\left\{ \ar[ccccl]{
    &&&\rule{20pt}{1pt}&{}^1P_1\\
    \rule{20pt}{1pt}&&&&{}^3P_2\\
    $$$$\\
    &&\rule{20pt}{1pt}&&{}^3P_1\\
    &\rule{20pt}{1pt}&&&{}^3P_0
}\right.$$
根据自旋轨道耦合能量计算式,取$Z^*=2,n=4$
$$U  = \frac{{hc{\alpha ^2}R{Z^{*4}}}}{{{n^3}l(l + \frac{1}{2})(l + 1)}}\frac{{j(j + 1) - l(l + 1) - s(s + 1)}}{2} \propto {\alpha ^4},\ R = \frac{{m{c^2}{e^4}}}{{4{\rm{\pi }}{{(4{\rm{\pi }}\varepsilon _0^{})}^2}{{(c\hbar )}^3}}}=1.1\e{7}m^{-1}$$
$$\ar[ccl]{
    L=1,&S=0,&\quad J=1,\quad \D U=0\mu eV\\
    L=1,&S=1,&\left\{
                    \ar{J=0,&\D U=-120.8\mu eV\\
                        J=1,&\D U=-60.4\mu eV\\
                        J=2,&\D U=60.4\mu eV}\right.\\
    }$$
\section{Q4. 请给出Ca气体光谱的结构,仅考虑4s4p态到基态的跃迁。}
$4s4p\Rightarrow 4s^2$跃迁后$L=S=J=0$. 由于选择定则$\D S=0$,仅有$4^1P_1$态能一次跃迁到基态。即只有一条谱线。
\section{Q5. 请给出在磁场中Ca的基态和4s4p各个原子组态能级的移动情况及在磁场中Ca气体的光谱,仅考虑4s4p态到基态的跃迁。请考虑弱磁场和强磁场两种情况。}
\sub{弱场}
    $$\ar[lc]{
        {}^1P_1&\left\{\ar{
            m=1\\
            m=0\\
            m=-1
        }\right.\\
        {}^3P_0& m=0\\
        {}^3P_1&\left\{\ar{
            m=1\\
            m=0\\
            m=-1
        }\right.\\
        {}^3P_2&\left\{\ar{
            m=2\\
            m=1\\
            m=0\\
            m=-1\\
            m=-2
        }\right.
    }$$
    共12条\\
    其中由于$\D S=0$约束,能跃迁到基态的有三条:
    $${}^1P_1\left\{\ar{
        m=1\\
        m=0\\
        m=-1
    }\right.$$
\sub{强场}
    $$L=1, S=0,1, m_L+2m_S=-3,-2,-1,0,1,2,3$$
    $$\left\{\ar[lc]{
        m_L+2m_S=3\\
        m_L+2m_S=2\\
        m_L+2m_S=1\\
        m_L+2m_S=0\\
        m_L+2m_S=-1\\
        m_L+2m_S=-2\\
        m_L+2m_S=-3\\
    }\right.$$
    共7条\\
    其中由于$\D S=0$约束,能跃迁到基态的有三条:
    $${}^1P_1\left\{\ar{
        m_L+2m_S=1\\
        m_L+2m_S=0\\
        m_L+2m_S=-1\\
    }\right.$$
\end{document}