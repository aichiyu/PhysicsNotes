\documentclass[UTF8,9pt]{ctexart}

\usepackage{../template/Notes/notes}
\title{Statistical Physics}
\begin{document} 
\maketitle
\se{热力学基本规律}
\sub{热力学三定律}
    热力学第零定律:温度具有传递性,温度存在。\\  
    热力学第一定律$\d U=\dbar Q+\dbar W=\dbar Q-P\d V$\\
    热力学第二定律$\dbar Q \leq T\d S \ip \d U \leq T\d S -P\d V$。
\sub{状态方程} 
    体胀系数$\alpha$
    $$\alpha=\ff{V}(\pp{V}{T})_p$$
    压强系数$\beta$
    $$\beta=\ff{p}(\pp{p}{T})_V$$
    等温压缩系数$\kappa_T$
    $$\kappa_T=-\ff{V} (\pp{V}{p})_T $$
    三者关系:
    $$\alpha=\kappa_T\beta p,\quad (\pp{p}{T})_V=\f{\a}{\kappa_T}$$
    理想气体状态方程:$pV=nRT,\ pV=NkT,\ p=n_0kT$\\
    考虑分子斥力$(nb)$和引力$(\f{an^2}{V^2})$或通过平均场近似可得到Van der Waals方程:$(p+\f{an^2}{V^2})(V-nb)=nRT$\\
    将物态方程位力展开得到Onnes方程:$p=(\f{nRT}{V})\left[1=\f{n}{V}B(T)+(\f{n}{V})^2C(T)+\cdots\right]$, 低温引力显著,$B<0$,高温斥力显著,$B>0$。\\
    简单固体液体:$V(T,p)=V_0(T_0,0)\left[1+\alpha(T-T_0)-\kappa_T p\right],\quad \alpha= \ff{V}(\pp{V}{T})_p,\ \kappa_T=-\ff{V} (\pp{V}{p})_T$\\
    顺磁性固体:$M=\f{C}{T}H$(居里定律)或$M=\f{C}{T-\t}H$.
\sub{温度定义}
    由第零定律,存在一个函数$f(p,V)$,使得当两个可以交换热量的体系平衡时,$f(p_1,V_1)=f(p_2,V_2)$,称$\t=f(p,V)$为经验温度,经验温度的零点可以任意定义。\\
    利用卡诺热机$1-\eta=\f{Q_1}{Q_2}$关系,由于卡诺热机效率仅与温度有关,因此$1-\eta=f(\t_1,\t_2)=\f{Q_1/Q_3}{Q_2/Q_3}=\f{f(\t_1,\t_3)}{f(\t_2,\t_3)}$。观察后发现可以取$f(\t_1,\t_2)=\f{T_2}{T_1}$. $T$即为绝对温标。
\sub{热容}
    焦耳定律:理想气体内能只是温度的函数,即$(\pp{U}{V})_T=0$,由于$H=U+nRT$,$H$也只是温度的函数。$C=\f{\dbar Q}{\d T}$,
    $$C_p=(\pp{Q}{T})_p=(\pp{H}{T})_p=T(\pp{S}{T})_p=\dd{H}{T}\text{(理想气体)}$$
    $$C_V=(\pp{Q}{T})_V=(\pp{U}{T})_V=T(\pp{S}{T})_V=\dd{U}{T}\text{(理想气体)}$$
    迈耶公式 $C_p-C_V=T(\pp{p}{T})_V(\pp{V}{T})_p=\dd{(H-U)}{T}=\f{VT\a^2}{\kappa_T}=nR$
\sub{几个常见过程}
    准静态过程:气体外界做功$\dbar W=-p\d V$。对磁性体系$\dbar W=H\d M$, 对电学体系$\dbar W=E\d P$, 化学反应粒子数变化$\dbar W=\mu\d N$,一般地,$\dbar W=\sum Y_i\d y_i$,即做功为外参量$y_i$与广义力$Y_i$之乘积。\\
    绝热过程:无热量交换,有能量交换,等熵,气体可以做功,在$p-V$图上比反比例函数陡。绝热过程的状态方程为 $pV^\gamma=const,\ TV^{\gamma-1}=const,\ \f{p^{\gamma-1}}{T^\gamma}=const,\ (\gamma=C_p/C_V,\ C_V+NK_B=C_p)$\\
    等温过程:有能量转化,理想气体内能保持不变,在$p-V$图上是反比例函数。\\
    等压过程:$\d H=\d U+p\d V=\dbar Q$\\
    等容过程:$\d H=\d U+V\d p,\ \d U=\dbar Q$
\sub{几个常见体系}
    孤立体系:无能量,物质交换,无体积变化。\\
    封闭体系:无物质交换。\\
    开放体系:有物质交换。
\sub{四个热力学基本量}
    体积 $V$\\
    压强 $p$\\
    温度 $T$ 理想气体内能只与温度有关。\\
    热 $\dbar Q = T\d S$\\
    熵 $S$对于可逆过程,熵$S$是一个态函数,积分与路径无关;对于绝热过程中,熵永不减少。$\d S=\f{\dbar Q}{T},\ S=\f{3}{2}Nk\ln (pV^{5/3})+const,\ S=k\ln\o$ \\
    $T,P,S,V$ 中只有两个独立变量。
\se{均匀物质的热力学性质}
\sub{四个势函数}
    \def({\lparen} 
    \def){\rparen}
    \setlength{\arraycolsep}{5pt}                                         %space after &
    $$\ar[rlrllc]{
        U= & Q+W & \d U & =C_V\d T + \left[T(\pp{p}{T})_V-p\right]\d V+\left[\mu-T(\pp{\mu}{T})_{V,n}\right]\d n & =T\d S-p\d V+\mu\d n \\
        H= & U+pV & \d H  & =C_p\d T+\left[V-T(\pp{V}{T})_p\right]\d p & =T\d S+V\d p+\mu\d n \\  
        F= & U-TS & \d F & \d F_{T,V} \leq 0 & =-S\d T-p\d V+\mu\d n \\
        G= & U-TS+pV=\mu n & \d G & \d G _{p,T} \leq 0 & =-S\d T+V\d p+\mu\d n \\
        S &  & \d S & =\f{C_V}{T}\d T+(\pp{p}{T})_V\d V-(\pp{\mu}{T})_{V,n}\d n & \\
        S  &   & \d S & =\f{C_p}{T}\d T-(\pp{V}{T})_p\d p\\
        J= & F-\mu n= -pV & \d J && =-S\d T-p\d V-n\d \mu\\
        \mu  &   & \d\mu & & =-S_m\d T+V_m\d p = G_m(\text{单相})
    }$$
    \setlength{\arraycolsep}{1pt}
    麦克斯韦: 
    $$\ar{
        -(\pp{p}{S})_V &= (\pp{T}{V})_S\\
        (\pp{T}{p})_S &= (\pp{V}{S})_p\\
        (\pp{S}{V})_T &= (\pp{p}{T})_V\\
        -(\pp{S}{p})_T &= (\pp{V}{T})_p\\
        \hline
        (\pp{S}{n})_{T,V} &= -(\pp{\mu}{T})_{V,n}
    }$$
    \def({\ifmmode \left\lparen \else\lparen\fi} 
    \def){\ifmmode \right\rparen \else\rparen\fi}
    记忆:Good Physicists Have Studied Under Very Fine Teacher. 按首字母顺序画出下图,势函数紧挨着的两个量是d里面的,若箭头指出则为正,箭头指入则为负。
    \putfig{0.15}{1.png}
\sub{气体的节流过程和绝热膨胀过程}
    获得低温的方法主要有节流过程和绝热膨胀过程;节流过程前后气体的温度发生了变化,这个效应称之为:焦耳-汤姆孙效应;对于理想气体,节流过程前后温度不变。
\sub{磁介质热力学}
$\dbar W=V\d(\ff{2}{\mu_0H^2}+\mu_0VH\d M)$, 若体系只有磁介质, 没有磁场, $\dbar W=\mu_0VH\d M$. 与气体之间有代换关系: $p \rightarrow -\mu_0H,\ V\rightarrow m$. 热容$C_H=T(\pp{S}{T})_H$. 居里定律$m=\f{CV}{T}H$. 
\se{单元系的相变}
1. 孤立系统达到平衡态的时候,系统的熵处于极大值状态,这是孤立系统平衡态的判据;如果极大值不止一个,则当系统处于较小的极大值的时候,系统处于亚稳平衡态。\\
2. 孤立系统处在稳定平衡态的充要条件是:$\D S<0$ ;等温等容系统处在稳定平衡态的充要条件是:$\D F>0$ ;等温等压系统处在稳定平衡态的充要条件是:$\D G>0$ 。\\
3. 当系统对于平衡状态而发生某种偏离的时候,系统中将会自发地产生相应的过程,直到恢复系统的平衡。\\
4. 开系的热力学基本方程:$\d U=T\d S-p\d V+\mu \d n$\\
5. 单元系的复相平衡条件:$T^\alpha=T^\beta,\ p^\alpha=p^\beta,\ \mu^\alpha=\mu^\beta$\\ 
6. 汽化线、熔解线与升华线的交点称为三相点,在三相点固、液、气三相可以平衡共存。\\
7. 单元系三相共存时,$\left\{\ar{&T^\alpha=T^\beta=T^\gamma=T_0\\&p^\alpha=p^\beta=p^\gamma=p_0\\&\mu^\alpha(T,p)=\mu^\beta(T,p)=\mu^\gamma(T,p)}\right.$,  即三相$(\alpha,\beta,\gamma)$的温度、压强和化学势必须相等。\\
8. 相变时$L=T\D S$, $\dd{p}{T}=\f{L}{T\D V}$, 对理想气体可简化为$\ff{p}{\dd{p}{T}=\f{L}{RT^2}}$.
9. 相变时$T,p$不变, 相变潜热$L=H_m$.
\se{多元系的复相平衡和化学平衡}
1、多元系是由含有两种或两种以上化学组分组成的系统,在多元系既可以发生相变,也可以发生化学变化。\\
2、在系统的$T$和$p$不变时,若各组元的摩尔数都增加$\l$倍,系统的$V,U,S$也应增加$\l$倍。\\
3、多元系的热力学基本方程:$\d U=T\d S-p\d V+\sum\mu_i\d n_i$\\
4、吉布斯关系: $S\d T-V\d p+\sum n_i\d \mu_i=0$\\
5、多元系的复相平衡条件:整个系统达到平衡的时候,两相中各组元的化学势必须分别相等,即$\mu_i^\alpha=\mu_i^\beta$。\\
6、化学反应(所有的反应物和生成物都在同一相):$\sum\nu_iA_i=0$;其化学平衡条件为:$\sum\nu_i\mu_i=0$.\\
7、道尔顿分压定律:混合理想气体的压强等于各组元的分压之和,即$p=\sum p_i$.\\
8、理想气体在混合前后的焓值相等,所以理想气体在等温等压下混合过程中与外界没有热量交换。\\
9、偏摩尔体积、偏摩尔内能和偏摩尔熵:
$$\ar{
    V=\sum n_i(\pp{V}{n_i})_{T,p,n_j}=\sum n_i\nu_i\\
    U=\sum n_i(\pp{U}{n_i})_{T,p,n_j}=\sum n_iu_i\\
    S=\sum n_i(\pp{S}{n_i})_{T,p,n_j}=\sum n_is_i
}$$
物理意义:在保持温度($T$)、压强($p$)和其他组元($n_j$)摩尔数不变的条件下,每增加1 mol的第$i$组元物质,系统体积(或内能、熵)的增量。\\
10、混合理想气体的物态方程:$pV= RT\sum n_i$,由此可得摩尔分数$\f{p_i}{p}=\f{n_i}{\sum n_i}=x_i$。\\
11、混合理想气体的吉布斯函数$G=\sum n_i\mu_i=\sum n_iRT\left[\phi_i+\ln(x_ip)\right]$,混合理想气体的内能$U=\sum n_i(\int c_{\nu i}\d T+u_{i0})$(混合理想气体的内能等于分内能之和),混合理想气体的熵$S=\sum n_i\left[\int\f{c_{pi}}{T}\d T-R\ln(x_ip)+s_{i0}\right]$. 
\se{近独立粒子的最概然分布}
\sub{宏观态与微观态}
    宏观态:体积,温度等。\\
    微观态:经典力学下是每个粒子的速度,位置。量子力学下是每个粒子的量子数。\\
    等概率原理(统计物理基本假定):系统处于一个平衡态时,有确定的宏观态,但其微观态有多种可能。等概率原理表明,其各个微观态出现的概率相等。
\sub{压强定义($x$方向)}
    把粒子分为速度为$v_1,,v_2...$的许多类,每类有$N_i$个粒子. 时间$\d t$内共有$\sum_{v_x>0} Av_{ix}\d t\f{N_i}{V}$个粒子撞击表面,产生冲量为$\d I=\sum_{v_x>0} 2mv_{ix} \cdot Av_{ix}\d t\f{N_i}{V}=F_x\d t$
    因此
    $$\ar{
        P_x=&\f{F_x}{A}=\sum_{v_x>0} 2mv_{ix}^2\f{N_i}{V}\\
        =&\f{2m}{V}\sum_{v_x>0} N_iv_{ix}^2\\
        =&\f{mN}{V}\sum \f{N_i}{N}v_{ix}^2\\
        =& mN\sum \f{v_{ix}^2}{N}\f{N_i}{V}\\
        =&mN\int \f{v_x^2}{N}\d n=mN\int v_x^2f(v)d\vec{v}\\
        =&mN\bar{v_x^2}
    }$$
    又由于$x,y,z$各向同性,$\bar{v_x^2}=\ff{3}\bar{v^2},\ P=\ff{3}nm\bar{v^2}$.
\se{Boltzmann统计}
\sub{Maxwell分布}
    $\int \f{v^2}{N}\d n=\int v^2f(v)d\vec{v}$代换将位形空间变换到相空间。\\
    即为Maxwell分布:
    $$f{\d n(\vec{v})}{N}=f(\vec{v})\d \vec v =(\f{m}{2\pi k_BT})^{3/2}\exp (-\f{mv^2/2+V(x)}{k_BT})\d \vec v$$
\sub{理想气体的熵}
    Boltzmann的方法:\\
    定义$H_0=\bar{\ln f}=\int f\ln f\d\vec v= n\left\{\ln\left[n(\ff{2\pi mkT})^{3/2}\right]-\f{3}{2}\right\}$ , 又由于$H=-\f{S}{Vk}$,可以得到熵的定义:
    $$S=-kVH_0=\f{3}{2}Nk\ln (pV^{5/3})+const$$
    Planck的方法\\
    一个系统最开始的状态总是不那么可几的(状态数较少),系统总是会逐渐趋向于一个最可几的(状态数最多)的状态。由于状态数$\o$随时间变化与熵$S$有高度的相似性,可以假设二者具有一个直接的关系$S=f(\o)$\\
    考虑相互独立的体系,熵具有可加性(广延量)。总的状态数为两个独立体系相乘。
    $$S_{12}=S_1+S_2,\quad \o_{12}=\o_1\o_2$$
    因此
    $$f(\o_1\o_2)=f(\o_1)+f(\o_2)$$
    两边对$\o_1$求导
    $$\pp{f(\o_1\o_2)}{\o_1\o_2}\pp{\o_1\o_2}{\o_1}=\pp{f(\o_1\o_2)}{\o_1\o_2}\o_2=\pp{f(\o_1)}{\o_1}$$
    再对$\o_2$求导
    $$\pp{f(\o_1\o_2)}{\o_1\o_2}+\dd{}{\o_2}(\pp{f(\o_1\o_2)}{\o_1\o_2})\o_2=0$$
    $$\ip \pp{f(\o_1\o_2)}{\o_1\o_2}+\pp[2]{f(\o_1\o_2)}{(\o_1\o_2)}\o_1\o_2=0$$
    代入$\o=\o_1+\o_2$
    $$\pp{f(\o)}{\o}+\pp[2]{f(\o)}{\o}\o=0$$
    积分后再代入$f(\o_1\o_2)=f(\o_1)+f(\o_2)$约去常数后可得
    $$S=f(\o)=k_B\ln \o$$
\sub{热力学基本方程}
\se{玻色统计和费米统计}
\se{系综理论}
\se{涨落理论}
\end{document}