\documentclass[UTF8,9pt]{ctexart}
\usepackage{../template/Notes/notes}
\usepackage[colorlinks,linkcolor=red]{hyperref}
\title{CUPT理论分析-Slinky 弹簧} 
\begin{document} 
\maketitle
\se{弹簧属性}
\sub{概况}
    波速$$a=\sqrt{(kx^2)/m}$$\\
    固有频率$$f=\frac{1}{2\pi}\sqrt{\frac{k}{m}}$$\\
    忽略完全收缩长度,则竖直时受重力后长度为$$L=\frac{W}{2k}$$\\
    质量分布
    $$p(n)=L(n-1)^2$$
    其中$n=\text{该点上方质量/总质量}  \in [0,1]$,$p(n)$给出了$n$从slinky底部向上的位置。
\sub{Slinky弹簧}
    由于弹簧太软只能拉伸不能压缩, 弹簧自由塌缩的速度快于弹性波波速时( 自由塌缩指的是始终满足胡克定律的弹簧自由振动塌缩的速度 ), 会出现波源运动速度比波速快的现象, 因此弹簧中会出现冲击波(激波). 而冲击波传播到末端的时间要比自由弹性波短很多.
\sub{软度}
    其轴向刚度k=0.003N/mm(compared with 1-200N/mm for others)\\
    \url{https://pic2.zhimg.com/v2-f46e972de8f2df71ab6753e39a60b819_b.gif}
\se{Euler–Bernoulli beam theory}
Euler-Bernoulli梁理论(也称为工程梁理论或经典梁理论)[1]是线性弹性理论的简化,它提供了一种计算梁的承载和挠度特性的方法。它涵盖了仅受到横向载荷的梁的小挠度的情况。\\
对于动态梁,其上某一质点:\\
$$T=\ff{2}\mu (\pt {\o} )^2$$
$$V_{in}=\ff{2}EI(\pt[2]{\o})^2$$
$$V_{out}=-q(x)\o(x,t)$$
有拉格朗日方程:
$$\frac { \partial ^ { 2 } } { \partial x ^ { 2 } }( E I \frac { \partial ^ { 2 } w } { \partial x ^ { 2 } }) = - \mu \frac { \partial ^ { 2 } w } { \partial t ^ { 2 } } + q ( x )$$
当梁均匀,$E,I$独立,
$$E I \frac { \partial ^ { 4 } w } { \partial x ^ { 4 } } = - \mu \frac { \partial ^ { 2 } w } { \partial t ^ { 2 } } + q$$
其中,\\
曲线$\o(x)$描述了梁在z方向上在某个位置x的偏转(回忆一下,梁被建模为一维对象)。q是一个分布式负载,换句话说,一个单位长度的力(类似于压力是一个单位面积的力);它可能是$x,\o$或其他变量的函数。\\
$E$为弹性模量,I是梁截面面积的二阶矩。I必须根据穿过截面质心并垂直于所应用的加载的轴来计算。式中,对于轴向为x,荷载沿z方向的梁,其截面在yz平面上,相应的面积二阶矩为$$I=\iint z^{2}\d y\d z$$
通常,乘积$EI$(称为抗弯刚度)是一个常数,所以
$$EI{\frac  {{\mathrm  {d}}^{4}w}{{\mathrm  {d}}x^{4}}}=q(x).\,$$
该方程描述了均匀静梁的挠度,
\se{建模}
\sub{数学模型}
压电换能器,测量脉冲响应实验显示,其色散为低频比高频传播慢。\\
\url{http://dafx10.iem.at/papers/ParkerPenttinenBilbaoAbel_DAFx10_P80.pdf}\\
使用无量纲形式伯努利方程:
$$\frac { \partial ^ { 2 } u } { \partial t ^ { 2 } } = - \kappa ^ { 2 } \frac { \partial ^ { 4 } u } { \partial x^{4}} + \left[-2\sigma _0\frac { \partial u } { \partial t } + 2 \sigma _ { 1 } \frac { \partial ^ { 3 } u } { \partial t \partial x ^ { 2 } } \right]$$
其中$u$为横向位移,$x$为沿弹簧轴方向的位置,$t$为时间,$\kappa$为与密度,杨氏模量,长度有关的无量纲常数(需要用实验确定大小)。括号内为对理想杆的修正,$\sigma$控制损失特征[S. Bilbao,Numerical Sound Synthesis, John Wiley and Sons,2009]
\sub{实验}
现假设脉冲从$x=0$传导至$x=1$,考虑理想杆模型$(\sigma_0=\sigma_1=0)$,有特定频率
$$T _ { D } = \frac { 1 } { 2 \sqrt { 2 \pi \kappa f } }$$
表达式给出了色散下第一个到达的波,通过该式可求得$\kappa$(求出后与实验对照)。\\
其中需要通过FIR带通滤波测量频率,且两次测到该频率的时间间隔给出该波传播花费的时间。
\se{MORE}
\sub{MORE1:有限差分}
有限差分可以对微分方程离散化,进行离散逼近。
\sub{MORE2:波导}
\sub{MORE3:螺旋运动矩阵}
\sub{MORE4:驻波}
推导声波纵波与弹簧谐波关系-利用驻波。
\section{购买}
购买,21.56元:\\
\url{https://item.taobao.com/item.htm?spm=a230r.1.14.6.71d846cdb52CTO&id=560671174453&ns=1&abbucket=3#detail}
\se{Ref}
\url{https://www.guokr.com/question/435419/}\\
\url{https://www.zhihu.com/question/45066646}\\
\url{https://en.wikipedia.org/wiki/Euler%E2%80%93Bernoulli_beam_theory}

\end{document}
