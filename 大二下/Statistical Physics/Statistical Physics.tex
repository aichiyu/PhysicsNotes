\documentclass[UTF8,9pt]{ctexart}
\usepackage{color}
\usepackage{../template/Notes/notes}
\newcommand{\q}[1]{{\color{red} #1}}
\title{Statistical Physics}
\begin{document}
\maketitle
\se{热力学基本规律}
\sub{热力学四定律}
\q{热力学第零定律}: 如果两个系统的每一个都与第三个系统单独处于热平衡, 则它们彼此也
处于热平衡.

\q{热力学第一定律}$\d U=\dbar Q+\dbar W,\ \dbar W=-P\d V=F\d x=\sg \d A = -m\d B$.

\q{热力学第二定律}$\dbar Q = T\d S \ip \d U = T\d S -P\d V$. \\
克劳修斯表述:不可能把热量从低温物体传到高温物体而不引起其他变化;克劳修斯等式与不等式$\frac{Q_{1}}{T_{1}}+\frac{Q_{2}}{T_{2}} \leq 0,\oint \f{\d Q}{T}\leq 0$\\
开尔文(汤姆孙)表述:不可能从单一热源吸收热量使之完全变成有用的功而不引起其他变化;\\
另一种开氏表述:第二类永动机不可能造成的。

\q{热力学第三定律}: 绝对零度不能达到. 接近于绝对零度时,处于内平衡的一个系统中的所有反应
发生时熵不变;
\sub{状态方程}
体胀系数$\alpha$
$$\alpha=\ff{V}(\pp{V}{T})_p$$
压强系数$\beta$
$$\beta=\ff{p}(\pp{p}{T})_V$$
等温压缩系数$\kappa_T$
$$\kappa_T=-\ff{V} (\pp{V}{p})_T $$
三者关系:
$$\alpha=\kappa_T\beta p,\quad (\pp{p}{T})_V=\f{\a}{\kappa_T}$$
理想气体状态方程: $pV=nRT,\ pV=NkT,\ p=n_0kT$\\
考虑分子斥力$(nb)$和引力$(\f{an^2}{V^2})$或通过平均场近似可得到\q{Van der Waals方程}: $(p+\f{an^2}{V^2})(V-nb)=nRT$. Van der Waals气体的定容热容与理想气体相同. \\
将物态方程位力展开得到\q{Onnes方程}: $p=(\f{nRT}{V})\left[1+\f{n}{V}B(T)+(\f{n}{V})^2C(T)+\cdots\right]$, 低温引力显著, $B<0$, 高温斥力显著, $B>0$. \\
简单固体液体: $V(T,p)=V_0(T_0,0)\left[1+\alpha(T-T_0)-\kappa_T p\right]$
\sub{温度定义}
由第零定律, 存在一个函数$f(p,V)$, 使得当两个可以交换热量的体系平衡时, $f(p_1,V_1)=f(p_2,V_2)$, 称$\t=f(p,V)$为经验温度, 经验温度的零点可以任意定义. \\
利用卡诺热机$1-\eta=\f{Q_1}{Q_2}$关系, 由于卡诺热机效率仅与温度有关, 因此$1-\eta=f(\t_1,\t_2)=\f{Q_1/Q_3}{Q_2/Q_3}=\f{f(\t_1,\t_3)}{f(\t_2,\t_3)}$. 观察后发现可以取$f(\t_1,\t_2)=\f{T_2}{T_1}$. $T$即为绝对温标.
\sub{热容}
热容量: 使系统温度升高1K所吸收的热量, 摩尔热容$C_m=C/n$.\\
\q{焦耳定律}: 理想气体内能只是温度的函数, 即$(\pp{U}{V})_T=0$, 由于$H=U+nRT$, $H$也只是温度的函数. $C=\f{\dbar Q}{\d T}$,
$$\ar{
		C_p\x (\pp{Q}{T})_p=(\pp{H}{T})_p=T(\pp{S}{T})_p=\dd{H}{T}\text{(理想气体)}=\f{5}{2}nR\\
		C_V\x (\pp{Q}{T})_V=(\pp{U}{T})_V=T(\pp{S}{T})_V=\dd{U}{T}\text{(理想气体)}=\f{3}{2}nR\\
		& (\pp{C_V}{V})_T=T(\pp[2]{p}{T})_V,(\pp{C_p}{p})_T=-T(\pp[2]{V}{T})_p
	}$$
迈耶公式 $C_p-C_V=[(\pp{U}{V})_T+p](\pp{V}{T})_p = T(\pp{p}{T})_V(\pp{V}{T})_p=(\text{理想气体})\dd{(H-U)}{T}=\f{VT\a^2}{\kappa_T}=nR$
\sub{几个常见过程}
\q{准静态过程}: 系统的热力学过程进行的无限缓慢、以至于每个中间态都可视作平衡态.气体外界做功$\dbar W=-p\d V$. 对磁性体系$\dbar W=H\d M$, 对电学体系$\dbar W=E\d P$, 化学反应粒子数变化$\dbar W=\mu\d N$, 一般地, $\dbar W=\sum Y_i\d y_i$, 即做功为外参量$y_i$与广义力$Y_i$之乘积. \\
\q{绝热过程}: 无热量交换, 有能量交换, 等熵, 气体可以做功, 在$p-V$图上比反比例函数陡. 绝热过程的状态方程为 $pV^\gamma=C_1,\ TV^{\gamma-1}=C_2,\ \f{p^{\gamma-1}}{T^\gamma}=C_3$, (\q{$\gamma=C_p/C_V$})\\
\q{等温过程}: 有能量转化, 理想气体内能保持不变, 在$p-V$图上是反比例函数. 1mol理想气体等温膨胀吸热$\Delta Q=R T \ln \frac{V_{2}}{V_{1}}$\\
等压过程: $\d H=\d U+p\d V=\dbar Q$\\
等容过程: $\d H=\d U+V\d p,\ \d U=\dbar Q$
\sub{热力学基本量}
体积 $V$\\
压强 $p$\\
温度 $T$ 理想气体内能只与温度有关. \\
热 $\dbar Q = T\d S$\\
\q{熵} $S$对于可逆过程, 熵$S$是一个态函数, 积分与路径无关;对于绝热过程中, 熵永不减少. \\
$\d S=\f{\dbar Q}{T},\ S=\f{3}{2}Nk\ln (pV^{5/3})+const,\ S=k\ln\o$. $S(T,p) = \int_0^T\f{C_p(T)}{T}\d T$\\
理想气体$S=nC_{V,m}\ln T+nR\ln V+S_0=nC_{p,m}\ln T-nR\ln p+S_0$ \\
$T,p,S,V$ 中只有两个独立变量.
\se{均匀物质的热力学性质}
\sub{空窖辐射}
$$\begin{array}{rl}
		U\x U(T, V)=V u(T) =aVT^4                        \\
		p\x \frac{1}{3} u(T)                             \\
		u\x aT^4                                         \\
		S\x \f{4}{3}a T^3V                               \\
		\text{绝热时熵不变}: & T^3V=const,pV^{4/3}=const \\
		F\x -\ff{3}VT^4                                  \\
		H\x \f{4}{3}aVT^4                                \\
		G\x 0
	\end{array}$$
能量密度与通量密度的关系: \q{斯特藩-玻尔兹曼定律}:
$$J=\frac{1}{4} c u=\frac{1}{4} a c T^{4}=\sigma T^{4},\sg=5.67 \times 10^{-8} W \cdot m^{-2} \cdot K^{-4}$$
吸收因数$\a_\o$(能量吸收百分比), 面辐射强度$e_\o$ (单位时间单位面积发射能量)有基尔霍夫定律:
$$
	\frac{e_{\omega}}{\alpha_{\omega}}=\frac{c}{4} u(\omega, T)
$$
黑体吸收因数$\a_\o=1$. 能完全吸收入射到表面上的所有辐射, 平衡辐射也称黑体辐射.
\sub{四个势函数}
$$\ar{{rlrllc}
		U= & Q+W & \d U & = \ar{C_V\d T + \left[T(\pp{p}{T})_V-p\right]\d V+\\\left[\mu-T(\pp{\mu}{T})_{V,n}\right]\d n} & =T\d S-p\d V+\mu\d n \\
		H= & U+pV & \d H  & =C_p\d T+\left[V-T(\pp{V}{T})_p\right]\d p & =T\d S+V\d p+\mu\d n \\
		F= & U-TS & \d F & \d F_{T,V} \leq 0 & =-S\d T-p\d V+\mu\d n \\
		G= & U-TS+pV=\mu n & \d G & \d G _{p,T} \leq 0 & =-S\d T+V\d p+\mu\d n \\
		S &  & \d S & =\f{C_V}{T}\d T+(\pp{p}{T})_V\d V-(\pp{\mu}{T})_{V,n}\d n & \\
		S  &   & \d S & =\f{C_p}{T}\d T-(\pp{V}{T})_p\d p\\
		J= & F-\mu n= -pV & \d J && =-S\d T-p\d V-n\d \mu\\
		\mu  &   & \d\mu & & =-S_m\d T+V_m\d p = G_m(\text{单相})
	}$$
麦克斯韦:
$$\ar{
		-(\pp{p}{S})_V &= (\pp{T}{V})_S\\
		(\pp{T}{p})_S &= (\pp{V}{S})_p\\
		(\pp{S}{V})_T &= (\pp{p}{T})_V\\
		-(\pp{S}{p})_T &= (\pp{V}{T})_p\\
		\hline
		(\pp{S}{n})_{T,V} &= -(\pp{\mu}{T})_{V,n}
	}$$
记忆: Good Physicists Have Studied Under Very Fine Teacher. 按首字母顺序画出下图, 势函数紧挨着的两个量是d里面的, 若箭头指出则为正, 箭头指入则为负.
\putfig{1.png}
\q{吉布斯-亥姆霍兹方程}
$$\ar{
		U=F+TS \x  F-T\left(\frac{\partial F}{\partial T}\right)_{V}\\
		H=G+TS \x  G-T\left(\frac{\partial G}{\partial T}\right)_{p}
	}$$
\sub{气体的节流过程和绝热膨胀过程}
获得低温的方法主要有节流过程和绝热膨胀过程;\\
对绝热膨胀, 温度降低:
$$
	\mu_{S}=\left(\frac{\partial T}{\partial p}\right)_{s}=-\left(\frac{\partial T}{\partial S}\right)_{p}\left(\frac{\partial S}{\partial p}\right)_{T}=\frac{T}{C_{p}}\left(\frac{\partial V}{\partial T}\right)_{p}=\frac{V T \alpha}{C_{p}}
$$
节流过程前后气体的温度发生了变化, 这个效应称之为焦耳-汤姆孙效应;对于理想气体, 节流过程前后温度不变($\a=0$). \q{绝热节流过程是等焓过程}.\\
节流制冷优缺点: 一定压强, 温度越低温度降落越大. 但初始温度必须低于反转温度, 可先绝热预冷, 再节流冷却.
定义\q{焦-汤系数}:
$$
	\mu=(\frac{\partial T}{\partial p})_{H}= \frac{V}{C_{p}}(T \alpha-1)
$$
\sub{磁介质热力学}
$\dbar W=V\d(\ff{2}{\mu_0H^2}+\mu_0VH\d M)$, 若体系只有磁介质, 没有磁场, $\dbar W=\mu_0VH\d M$. 与气体之间有代换关系: \q{$p \rightarrow -\mu_0H,\ V\rightarrow m$}. 热容$C_H=T(\pp{S}{T})_H$. 居里定律$m=\f{CV}{T}H$, $M=\f{C}{T}H$ (居里定律)或$M=\f{C}{T-\t}H$. 绝热条件下减少磁场, 磁介质温度将降低.
\se{单元系的相变}
\rk{
\item  孤立系统\q{平衡态的判据}: $\D S<0$ ;等温等容系统: $\D F>0$ ;等温等压系统: $\D G>0$ . 当系统处于较小的极大值的时候, 系统处于亚稳平衡态. 对于\q{多元复相平衡}, 要求各组元的每个相的$\mu$相等. 但不同组元的$\mu$可以不同.
%\item  当系统对于平衡状态而发生某种偏离的时候, 系统中将会自发地产生相应的过程, 直到恢复系统的平衡. 
%\item  \q{开系的热力学基本方程}: $\d U=T\d S-p\d V+\mu \d n$
%\item  \q{单元系的复相平衡条件}: 1. 等温等压系统---吉布斯判据(当吉布斯函数减至最小时,系统达到平衡);2. 等温等容---自由能判据. 另外需要整个系统达到平衡时,两相中的化学势都必须相等。$T^\alpha=T^\beta,\ p^\alpha=p^\beta,\ \mu^\alpha=\mu^\beta$
\item  汽化线、熔解线与升华线的交点称为三相点, 在三相点固、液、气三相可以平衡共存.
%\item  单元系三相共存时, $\left\{\ar{&T^\alpha=T^\beta=T^\gamma=T_0\\&p^\alpha=p^\beta=p^\gamma=p_0\\&\mu^\alpha(T,p)=\mu^\beta(T,p)=\mu^\gamma(T,p)}\right.$,  即三相$(\alpha,\beta,\gamma)$的温度、压强和化学势必须相等. 
\item  相变时\q{相变潜热}$L=T\D S$, $\dd{p}{T}=\f{L}{T\D V}$, 对理想气体可简化为$\ff{p}{\dd{p}{T}=\f{L}{RT^2}}$.
\item  相变时若$T,p$不变, 则$L=H_m$.
\item \q{克拉珀龙方程}: $\frac{\mathrm{d} p}{\mathrm{d} T}=\frac{L}{T\D V_m}$, 其中$\D V_m$为比容$V/m$变化.
\item \q{饱和蒸气压}公式: $p=p_{0} e^{-\frac{L}{R T}}$
\item 相变分类: 一级相变: 有$S,V$突变. 二级相变: $\pp[2]{\mu}{T},\pp[2]{\mu}{p}$突变.
}
\se{多元系的复相平衡和化学平衡}
\sub{多元系}
\rk{
%\item 多元系是由含有两种或两种以上化学组分组成的系统, 在多元系既可以发生相变, 也可以发生化学变化. 
\item 在系统的$T$和$p$不变时, 若各组元的摩尔数都增加$\l$倍, 系统的$V,U,S$也应增加$\l$倍.
\item 若$f(\l x_i)=\l^mf(x_i)$, 称$f$为$m$次齐函数, 满足$\sum x_i\pp{f}{x_i}=mf$.
\item 多元系的热力学基本方程: $\d U=T\d S-p\d V+\sum\mu_i\d n_i$
\item \q{吉布斯关系}:  $S\d T-V\d p+\sum n_i\d \mu_i=0$
\item \q{吉布斯相律}: $\vp$个相, 每个相有$k$个组元, 则总数为$(k+1)\vp$个强度量可以独立改变的有$f=k+2-\vp$个.
%\item 多元系的复相平衡条件: 整个系统达到平衡的时候, 两相中各组元的化学势必须分别相等, 即$\mu_i^\alpha=\mu_i^\beta$. 
\item 道尔顿分压定律: 混合理想气体的压强等于各组元的分压之和, 即$p=\sum p_i$. 能透过半透膜的物质在膜两边$p_i$相等.
\item \q{理想气体在混合前后的焓值相等}, 所以理想气体在等温等压下混合过程中与外界没有热量交换.
\item 偏摩尔体积、偏摩尔内能和偏摩尔熵:
$$\ar{
		V=\sum n_i(\pp{V}{n_i})_{T,p,n_j}=\sum n_i\nu_i\\
		U=\sum n_i(\pp{U}{n_i})_{T,p,n_j}=\sum n_iu_i\\
		S=\sum n_i(\pp{S}{n_i})_{T,p,n_j}=\sum n_is_i
	}$$
物理意义: 在保持温度 ($T$)、压强 ($p$)和其他组元 ($n_j$)摩尔数不变的条件下, 每增加1 mol的第$i$组元物质, 系统体积 (或内能、熵)的增量. \\
即: \q{偏摩尔X为某组元单位摩尔的X. }
%\item 混合理想气体的物态方程: $pV= RT\sum n_i$, 由此可得摩尔分数$\f{p_i}{p}=\f{n_i}{\sum n_i}=x_i$. 
\item 对于混合\q{理想气体}$$\ar{
	\mu\x \sum RT[\vp_i+\ln(x_ip)]\\
	G\x \sum n_iRT\left[\vp_i+\ln(x_ip)\right]\\
	H\x \sum n_i(\int c_{p i}\d T+h_{i0})\\
	U\x \sum n_i(\int c_{\nu i}\d T+u_{i0})\\
	S\x \sum n_i\left[\int\f{c_{pi}}{T}\d T-R\ln(x_ip)+s_{i0}\right]
	}$$
\item 等$n$两种气体等温等压混合, 熵增为$C=2nR\ln 2$.(\q{吉布斯佯谬})
\item \q{理想溶液化学势 (亨利定律, 半透膜)}$\mu_i(T,p)=g_i(T,p)+RT\ln x_i^L$. $g_i$为纯$i$的化学势.
\item \q{摩尔吉布斯函数}公式: $\left(\frac{\partial g}{\partial p}\right)_{r}=V_{m}$
}
\sub{化学平衡}
\rk{
\item 化学反应 (所有的反应物和生成物都在同一相)设个组元在反应方程中系数为$\nu_i$, 则: $\sum\nu_iA_i=0$. 其化学平衡条件为: $\sum\nu_i\mu_i=0$.
\item 定压反应热为$Q_P=D H$.
\item 定义\q{反应度}为$\ep = \f{\D n-\D n_b}{\D n_a-\D n_b}$. 其中$\D n_b\leq \D n\leq \D n$.
\item 表面张力系数为$\sigma=(\frac{\partial G}{\partial A})_{T, p}$
\item 平衡常数(\q{质量作用律}) $K_p=\prod p_i^{\nu_i}$.
\item $K(T,P) = p^{-\nu}K_p\ (\nu = \sum \nu_i)$.
}
\se{近独立粒子的最概然分布}
\sub{粒子运动状态的描述}
各个状态的分割: 三维体积元中量子态数为: $\d n_x\d n_y\d n_z = \f{V}{h^3}\sum \d p_i = \ff{h^3}\sum \d x_i\d p_i$\\
球坐标下动量空间体积元内状态数为:$\f{Vp^2\sin\t\d p\d \t \d \vp}{h^3}$. \\
动量$p$到$p+\d p$内的状态数为$\f{4\pi V}{h^3} p^2\d p$(二维$\f{2\pi L^2}{h^2}p\d p$).\\
能量$\ep$到$\ep+\d\ep$内的状态数为:$D(\ep)\d\ep=\f{2\pi V}{h^3}(2m)^{3/2}\ep^{1/2}\d\ep$(二维$\f{2\pi L^2}{h^2}m\d\ep$).\\
边长为$L$的立方体中, \q{能量本征值}为:
 m$$\ep_{n_xn_yn_z} = \ff{2m}\of{\f{2\pi \hbar}{L}}^2(n_x^2+n_y^2+n_z^2)$$
\q{近独立粒子系统}:粒子之间的相互作用很弱,相互作用的平均能量远小于单个粒子的平均能量,因而可以忽略粒子之间的相互作用。 \\
\q{全同性粒子系统}:由具有完全相同属性(相同的质量、自旋、电荷等)的同类粒子所组成的系统。\\
经典描述:自由粒子, 线性谐振子, 转子
$$\begin{array}{l}
		{\varepsilon=\frac{1}{2 m}\left(p_{x}^{2}+p_{y}^{2}+p_{z}^{2}\right)} \\
		{\varepsilon=\frac{p^{2}}{2 m}+\frac{1}{2} m \omega^{2} x^{2}}        \\
		{\varepsilon=\frac{1}{2 I}\left(P_{\theta}^{2}+\frac{1}{\sin ^{2} \theta} p_{\varphi}^{2}\right) \rightarrow \varepsilon=\frac{p_{\varphi}^{2}}{2 I}}
	\end{array}$$
量子描述:线性谐振子, 转子, 自旋角动量, 自由粒子
$$
	\begin{array}{l}
		{\varepsilon_{n}=\hbar \omega\left(n+\frac{1}{2}\right)}, n=1,2,3...                              \\
		{\varepsilon=\frac{L^{2}}{2 I} ; L^{2}=l(l+1) \hbar^{2}(l=0,1\cdots) ; L_{z}=m \hbar(m=-l\sim l)} \\
		{\varepsilon=-\vec{\mu} \cdot \vec{B}=\frac{e \hbar m_{s}}{m} B_{z}, S^2=s(s+1)\hbar^2}           \\
		{\varepsilon=\sum \frac{1}{2 m}p_i^2 ; p_{i}=\frac{h}{L} n_{i} ; d^3\bm{n}=\frac{V}{h^{3}} d^3 \bm{p}}
	\end{array}
$$
\sub{宏观态与微观态}
宏观态: 体积, 温度等.

微观态: 经典力学下是每个粒子的速度, 位置. 量子力学下是每个粒子的量子数.

\q{等概率原理 (统计物理基本假定)}: 系统处于一个平衡态时, 有确定的宏观态, 但其微观态有多种可能. 等概率原理表明, 其各个微观态出现的概率相等.

分布: 能级$\ep$, 简并度$\o$, 粒子数$a$.

各体系微观态的描述: \\
当$e^\a>>1 \iff a_l<<\o_l$, 玻色和费米分布过渡到玻尔兹曼, $\O_{B.E.}=\O_{F.D.}=\f{\O_{B.E.}}{N!}$.
\begin{table}[htbp]
	\centering
	\begin{tabular}{|c|c|c|c|}
		\hline
		             & 微观状态数                                                                                                & 分布                                                               & 平均粒子数密度                                         \\ \hline
		玻尔兹曼系统 & $\Omega_{M . B .}=\frac{N !}{\Pi_{l} a_{l} !} \Pi_{l} \omega_{l}^{a_{l}}$                                 & $\alpha_{l}=\omega_{l} e^{-\alpha-\beta \varepsilon_{l}}$          & $f_{s}=e^{-\alpha-\beta \varepsilon_{s}}$              \\ \hline
		玻色系统     & $\Omega_{B . E .}=\prod_{l} \frac{\left(\omega_{l}+a_{l}-1\right) !}{a_{l} !\left(\omega_{l}-1\right) !}$ & $\alpha_{l}=\frac{\omega_{l}}{e^{\alpha+\beta \varepsilon_{l}}-1}$ & $f_{s}=\frac{1}{e^{\alpha+\beta \varepsilon_{s}} - 1}$ \\ \hline
		费米系统     & $\Omega_{F . D .}=\prod_{l} \frac{\omega_{l} !}{a_l!\left(\omega_{l}-a_{l}\right) !}$                     & $\alpha_{l}=\frac{\omega_{l}}{e^{\alpha+\beta \varepsilon_{l}}+1}$ & $f_{s}=\frac{1}{e^{\alpha+\beta \varepsilon_{s}} + 1}$ \\ \hline
	\end{tabular}
\end{table}\\
\q{斯特林公式}: $\ln m~ =m(\ln m-1)$
\sub{压强定义($x$方向)}
把粒子分为速度为$v_1,,v_2...$的许多类, 每类有$N_i$个粒子. 时间$\d t$内共有$\sum_{v_x>0} Av_{ix}\d t\f{N_i}{V}$个粒子撞击表面, 产生冲量为$\d I=\sum_{v_x>0} 2mv_{ix} \cdot Av_{ix}\d t\f{N_i}{V}=F_x\d t$
因此
$$\ar{
	P_x\x \f{F_x}{A}=\sum_{v_x>0} 2mv_{ix}^2\f{N_i}{V}\\
	\x \f{2m}{V}\sum_{v_x>0} N_iv_{ix}^2\\
	\x \f{mN}{V}\sum \f{N_i}{N}v_{ix}^2\\
	\x  mN\sum \f{v_{ix}^2}{N}\f{N_i}{V}\\
	\x mN\int \f{v_x^2}{N}\d n=mN\int v_x^2f(v)d\vec{v}\\
	\x mN\bar{v_x^2}
	}$$
又由于$x,y,z$各向同性, $\bar{v_x^2}=\ff{3}\bar{v^2},\ P=\ff{3}nm\bar{v^2}$.
\sub{玻尔兹曼分布}
玻尔兹曼分布是粒子数和总能量守恒下(微正则系综)的最大熵分布.\\
处在能量为$\ep_s$的量子态$s$的平均粒子数为:
$$f_s=e^{-\a-\b\ep_s}$$
$$N=\sum e^{-\a-\b\ep_s}, \quad E=\sum \ep_s e^{-\a-\b\ep_s}, \quad \b=\ff{kT}$$

相格大小为$\prod \D q_r\D p_r=h^r$.\\
体积$V$内状态数:
$$D(p)\d p=\f{4\pi V}{h^3}p^2 \d p, \quad D(\ep)\d\ep = \f{2\pi V}{h^3}(2m)^{3/2}\ep^{1/2}\d\ep$$
\sub{理想气体的熵}
Boltzmann的方法: \\
定义$H_0=\bar{\ln f}=\int f\ln f\d\vec v= n\left\{\ln\left[n(\ff{2\pi mkT})^{3/2}\right]-\f{3}{2}\right\}$ , 又由于$H=-\f{S}{Vk}$, 可以得到熵的定义:
$$S=-kVH_0=\f{3}{2}Nk\ln (pV^{5/3})+const$$
Planck的方法\\
一个系统最开始的状态总是不那么可几的 (状态数较少), 系统总是会逐渐趋向于一个最可几的 (状态数最多)的状态. 由于状态数$\o$随时间变化与熵$S$有高度的相似性, 可以假设二者具有一个直接的关系$S=f(\o)$\\
考虑相互独立的体系, 熵具有可加性 (广延量). 总的状态数为两个独立体系相乘.
$$S_{12}=S_1+S_2,\quad \o_{12}=\o_1\o_2$$
因此
$$f(\o_1\o_2)=f(\o_1)+f(\o_2)$$
两边对$\o_1$求导
$$\pp{f(\o_1\o_2)}{\o_1\o_2}\pp{\o_1\o_2}{\o_1}=\pp{f(\o_1\o_2)}{\o_1\o_2}\o_2=\pp{f(\o_1)}{\o_1}$$
再对$\o_2$求导
$$\pp{f(\o_1\o_2)}{\o_1\o_2}+\dd{}{\o_2}(\pp{f(\o_1\o_2)}{\o_1\o_2})\o_2=0$$
$$\ip \pp{f(\o_1\o_2)}{\o_1\o_2}+\pp[2]{f(\o_1\o_2)}{(\o_1\o_2)}\o_1\o_2=0$$
代入$\o=\o_1+\o_2$
$$\pp{f(\o)}{\o}+\pp[2]{f(\o)}{\o}\o=0$$
积分后再代入$f(\o_1\o_2)=f(\o_1)+f(\o_2)$约去常数后可得
$$S=f(\o)=k_B\ln \o$$
\sub{玻色分布}
处在能量为$\ep_s$的量子态$s$的平均粒子数为:
$$f_s=\ff{e^{\a+\b\ep_s}-1}$$
\sub{费米分布}
处在能量为$\ep_s$的量子态$s$的平均粒子数为:
$$f_s=\ff{e^{\a+\b\ep_s}+1}$$
\se{玻尔兹曼统计}
\sub{热力学量的统计表达式}
定义配分函数:
$$Z_1=\sum \o_le^{-\b\ep_l}\text{  or  }Z_1=\sum e^{-\b\ep_l}\f{\D \o_l}{h_0^r}$$
满足$$N=e^{-\a}Z_1$$
各种热力学量为([]中为费米/玻色子在经典极限下较玻尔兹曼多出来的一项)
$$\begin{array}{rl|rl}
		U \x -N \pp{}{\b}\ln Z_1                      & \d U \x \sum a_l\d \ep_l+\ep_l\d a_l                                                               \\
		                                              &                                      & \d Q \x -N\d(\pp{\ln z_1}{\b})+\f{N}{\b}\pp{\ln Z_1}{y}\d y \\
		S \x Nk(\ln Z_1-\b\pp{}{\b}\ln Z_1)[-k\ln N!] &                                      &                                                             \\
		\x k\ln\O[-k\ln N!]                                                                                                                                \\
		F \x -NkT\ln Z_1[+kT\ln N!]                   &                                      &
	\end{array}$$
	\sub{理想气体状态方程}
	$$p=\f{N}{\b}\pp{}{V}\ln Z_1$$
	经典极限条件为$e^\a = \f{V}{N}\of{\f{2\pi mkT}{h^2}}^{3/2}>>1$. 即气体稀薄, 温度高, 分子质量大, 德布罗意热波长<<分子间距, 体积$\l^3$内分子数<<1.
\sub{Maxwell速度分布}
$\int \f{v^2}{N}\d n=\int v^2f(v)d\vec{v}$代换将位形空间变换到相空间. \\
即为Maxwell分布:
$$\ar{f[ n(\vec{v})]{\d N}\x f(\vec{v})\d \vec v \\
		\x(\f{m}{2\pi k_BT})^{3/2}\exp (-\f{mv^2/2+V(x)}{k_BT})\d \vec v}$$
麦克斯韦玻尔兹曼分布的重要均值:$\langle v\rangle=\sqrt{\frac{8 k_{B} T}{\pi m}},\left\langle v^{2}\right\rangle=\frac{3 k_{B} T}{m}$\\
理想气体压强:$p=\int_{0}^{\infty} \d v \int_{0}^{\pi / 2} 2 m v \cos \theta \d \theta\left(v \cos \theta n f(v) \frac{1}{2} \sin \theta\right)$
$=\frac{1}{3} n m\left\langle v^{2}\right\rangle$\\
理想气体状态方程:$p=\frac{1}{3} n m\left\langle v^{2}\right\rangle=\frac{N k_{B} T}{V}$
单位时间逃逸的分子数密度为: $\Gamma = \pi  n \of{\f{m}{2\pi kT}}^{3/2} \exp(-\f{mv^2}{2kT}) \d v$. \\
\sub{理想气体的内能和热容}
平动为
$$Z_1^t=V\of{\f{2\pi m}{h^2\b}}^{3/2},\quad U^t=-N\pp{}{\b}\ln Z_1^t=\f{3}{2}NkT,\quad C_V^t=\f{3}{2}Nk$$
振动为
$$Z_1^v=\f{e^{-\b\hbar\o}{2}}{1-e^{-\b\hbar\o}},\quad U^v=\f{N\hbar\o}{2}+\f{N\hbar\o}{e^{\b\hbar\o}-1},\quad C_V^t=Nk\of{\f{\hbar\o}{kT}}^2\f{e^{\hbar\o/kT}}{(e^{\hbar\o/kT}-1)^2}$$
定义$k\t_v=\hbar\o$, 常温下近似
$$U^v=\f{Nk\t_v}{2}+Nk\t_ve^{-\t_v/T},\quad C_V^v=Nk\of{\f{\t_v}{T}}^2e^{-\f{\t_v}{T}}$$
异核双原子转动为: 定义$k\t_r=\hbar^2/(2I)$,
$$Z_1^r=\sumzi(2l+1)e^{-\f{\t_r}{T}l(l+1)},\quad U^r=NkT,\quad C_V^r=Nk$$
\sub{理想气体的熵}
单原子理想气体的熵为
$$S=\f{3}{2}Nk\ln T+Nk\ln\f{V}{N}+\f{3}{2}Nk\of{\f{5}{3}+\ln\of{\f{2\pi mk}{h^2}}}$$
蒸气压(萨库尔-铁特罗特公式)
$$\ln p=-\f{L}{RT}+\f{5}{2}\ln T+\f{5}{2}+\ln\of{k^{5/2}\of{\f{2\pi m}{h^2}}^{3/2}}$$
$$\mu=kT\ln\of{\f{N}{V}\of{\f{h^2}{2\pi mkT}}^{3/2}}<0$$
\sub{固体热容的爱因斯坦理论}
假设3N个振子频率相同. 
$$U=3N\f{\hbar\o}{2}+\f{3N\hbar\o}{e^{\b\hbar\o}-1}$$
引入特征温度$k\t_E=\hbar\o$
$$C_V=3Nk\of{\f{\t_E}{T}}^2\f{e^{\f{\t_E}{T}}}{\of{e^{\f{\t_E}{T}}-1}^2}$$
高温下$C_v=3Nk$, 低温下$C_V=3Nk\of{\f{\t_E}{T}}^2e^{-\f{\t_E}{T}}$
\se{玻色统计和费米统计}
\sub{热力学量的统计表达式}
非简并条件:
$$e^\a = \f{V}{N}\of{ \f{2\pi nkT}{h^2} }^{3/2}>>1,\ \text{or}\ n\l^3 = \f{N}{V}\of{\f{h^2}{2\pi nkT}}^{3/2}<<1$$
$$\ln\Xi=\sum\o_l\ln(1+e^{-\a-\b\ep_l})$$
$$\bar{N} = \sum \f{\o_l}{e^{\a+\b\ep_l}-1}=-\pp{}{\a}\ln\Xi$$
$$U=\sum\f{\ep_l\o_l}{e^{\a+\b\ep_l}-1} = -\pp{}{\b}\ln\Xi$$
$$p=\ff{\b}\pp{}{V}\ln\Xi$$
$$S = k(\ln\Xi+\a\bar{N}+\b U)=k\ln\O$$
\sub{弱简并费米(玻色)气体}
弱简并费米(玻色)气体的内能为
$$U=\f{3}{2}NkT\left[ 1\pm \ff{2^{5/2}}\ff{g}\f{N}{V}\of{\f{h^2}{2\pi mkT}}^{3/2} \right]$$
压强为:
$$nkT\left[ 1\pm \ff{2^{5/2}}\ff{g}n\of{\f{h^2}{2\pi mkT}}^{3/2} \right]$$
\sub{玻色爱因斯坦凝聚}
根据玻色分布$a_l=\f{\o_l}{e^{\f{\ep_l-\mu}{kT}}-1}$, $\o_l=D(\ep)\d\ep$, $\ep_l=0>\mu$. 温度越低$|\mu|$越小. 存在临界温度$T_c=\f{2\pi}{2.612^{2/3}}\f{\hbar^2}{mk}n^{2/3}$. 当$T<T_c$时, 
$$n_0(T)+\f{2\pi}{h^3}(2m)^{3/2}\intzi \f{\ep^{1/2}\d\ep}{e^{\f{\ep}{kT}}-1}=n\ip n_0=n\of{1-\of{\f{T}{T_c}}^{3/2}}$$
$$U=0.77NkT\of{\f{T}{T_c}}^{3/2}$$
$$C_V=1.925Nk\of{\f{T}{T_c}}^{3/2}$$
出现凝聚体的条件为
$$n\l^3\geq2.612$$
二维自由理想玻色气体无玻色凝聚:
$$\f{2\pi L^2}{h^2}m\intzi\f{\d\ep}{e^{\f{\ep}{kT_c}}-1}=n$$
磁光陷阱中3[2]维为
$$\f{N_0}{N}=1-\of{\f{T}{T_c}}^{3[2]}$$
$$\f{kT_c}{h\bar\o}=\of{\f{N}{1.202}}^{1/3}[\of{\f{N}{1.645}}^{1/2}]$$
\sub{光子气体}
光子量子态数
$$\f{8\pi VV}{h^3}p^2\d p=\f{V}{\pi^2c^3}\o^2\d\o,\quad a_l=\f{\o_l}{e^{\b\ep_l}-1}$$
辐射场内能(普朗克公式)
$$U\d\o=\f{V}{\pi^2c^3}\f{\hbar\o^3}{e^{\hbar\o/kT}-1}\d\o$$
低频$\hbar\o<<kT$极限瑞利金斯
$$U=\f{V}{\pi^2c^3}\o^2kT$$
高频极限维恩公式
$$U=\f{V}{\pi^2c^3}\hbar\o^3e^{-\f{\hbar\o}{kT}}$$
普朗克公式积分得
$$U=\f{\pi^2k^4}{15c^3\hbar^3}VT^4$$
内能密度极大值与温度成正比(维恩位移律):
$$\f{\hbar\o_m}{kT}=2.822$$
光子气体巨配分函数
$$\ln\Xi=\f{\pi^2V}{45c^3(\b\hbar)^3}$$
辐射通量密度
$$J_u=\f{\pi^2k^4}{60c^2\hbar^3}T^4$$
\sub{金属中的自由电子}
自由电子热容非常小. 化学势由下式确定
$$\f{4\pi V}{h^3}(2m)^{3/2}\intzi \f{\ep^{1/2}\d\ep}{e^{\f{\ep-\mu}{kT}}+1}=N$$
并满足0K化学势$\mu(0)$(费米能级)为
$$\mu(0)=\f{\hbar^2}{2m}\of{3\pi^2\f{N}{V}}^{2/3}$$
令$P_F^2=2m\mu(0)$, 费米动量为
$$P_F=(2\pi^2n)^{1/3}\hbar$$
费米能级在eV级, 定义$kT_F=\mu(0)$, 费米温度在$1\e{4}$级别, 远高于常温.0K时电子气体内能为$U(0)=\f{3N}{5}\mu(0)$, 压强为$p(0)=\f{2}{5}n\mu(0)$. \\
$T>0$时只有能量在$\mu$附近, 量级为$kT$范围的电子对热容有贡献. 这段范围的有效电子数为$N_{eff}=\f{kT}{\mu}N$. 每一有效电子贡献热容$\f{3}{2}kT$. 因此电子热容为$C_V=\f{3}{2}Nk\f{T}{T_F}$, 精确值为
$$C_V=Nk\f{\pi^2}{2}\f{kT}{\mu(0)}$$
\se{系综理论}
\sub{微正则系综(确定$N,V,S,U$)}
归一化条件:$\int\rho(q,p,t)\d\O=1$. 称$\rho$为系综分布函数. \q{确定分布函数是系综理论的根本问题}. 若$B(t)$为依微观量, $\bar{B}(t)=\sum \rho_sB_s$为对应的宏观量. \\
等概率原理:$\rho_s=1/\O$.\\
\sub{正则系综(确定$N,V,T,F$)}
以$\O_l$表示简并度, 量子和经典的表述分别为
$$\rho_i=\ff{Z}e^{-\b E_s} = \ff{Z}\O_le^{-\b E_l} = \ff{N!h^{Nr}}e^{-\b E(q,p)}\d\O$$
$$Z=\sum e^{-\b E_s}=\sum\O_le^{-\b E_l}=\ff{N!h^{Nr}}\int e^{-\b E(q,p)}\d\O$$
$$U = -\pp{}{\b}\ln Z$$
$$p = \ff{\b}\pp{}{V}\ln Z, \b=\ff{kT}$$
$$S = k\of{\ln Z-\b\pp{}{\b}\ln Z}$$
$$F=U-TS=-kT\ln Z$$
能量的涨落为(证明了$C_V$恒正)
$$\bar{(E-\bar{E})^2}=-\pp{\bar{E}}{\b}=kT^2\pp{\bar{E}}{T}=kT^2C_V $$

\sub{实际气体}
Onnes方程, 低温引力显著, $B<0$, 高温斥力显著, $B>0$. 
$$pV=NkT(1+\f{nB}{V}),\ B=-\f{N_A}{2}\int f_{12}\d\bm{r}$$
考虑分子斥力$(nb)$和引力$(\f{an^2}{V^2})$或通过平均场近似可得到\q{Van der Waals方程}: $(p+\f{an^2}{V^2})(V-nb)=nRT$. Van der Waals气体的定容热容与理想气体相同. \\
\sub{固体的热容}
3N个简正振动能量
$$E=\p_0+\sum_1^{3N}\hbar\o_i\of{n_i+\ff{2}},n_i=0,1...$$
$$U=U_0+\sum_{i=1}^{3N}\f{\hbar\o_i}{e^{\b\hbar\o_i}-1}$$
其中$U_0$是负数的结合能, 绝对值大于零点能量. 
Debye将固体看做连续介质, 3N个简正振动是介质波动. 其中有纵波l与横波t. 简正振动数为
$$D(\ep)\d\ep=B\o^2\d\o,\ B=\f{V}{2\pi^2}\of{\ff{c_l^3}+\f{2}{c^3_t}}$$
最大圆频率(德拜频率)$\o_D^3=\f{9N}{B}$, 该式称为德拜频谱. 引入德拜函数, 可以将内能表示为
$$U=U_0+3NkT\mathfrak{D}(x),\ \mathfrak{D}(x)=\f{3}{x^3}\int_0^x\f{y^3\d y}{e^y-1},\ x=\f{\hbar\o}{kT},y=\f{\hbar\o_D}{kT}$$
高温下$x<<1$. $\mathfrak{D}(x)=1$. $C_V=3Nk$. 符合经典统计. \\
低温下$x>>1$. $\mathfrak{D}(x)=\f{\pi^4}{5x^3}$. $C_V=3Nk\f{4\pi^4}{5}\of{\f{T}{\t_D}}^3$. 为德拜$T^3$律. 对于非金属和3K以上金属符合. 对于$\l>>a$的低频波符合好. 对于铜的实验现象, 爱因斯坦理论高温下符合的好(偏小), 德拜理论一直符合得好. T3律低温符合得好. \\
声子的准动量和能量为:$p\hbar k,\ep=\hbar\o$. 两个方向的能动量关系为$\ep=c_lp,\ep=c_tp$. 一个状态上的平均声子数为$1/(e^{\hbar\o/kT}-1)$.
\sub{巨正则系综(确定$N,V,\mu$)}
量子和经典表达式为($r$代表源):
$$\rho_{Ns}=\ff{\Xi}e^{-\a N-\b E_s},\rho_N\d q\d p=\ff{N!h^{Nr}}\f{e^{-\a N-\b E}}{\Xi}\d\O$$
$$\Xi=\sum_{N=0}^\inf\sum_s e^{-\a N-\b E_s}=\sum_N\f{e^{-\a N}}{N!h^{Nr}}\int e^{-\b E}\d\O$$
$U,p$表达式与正则系综相同, 熵变为
$$S = k\of{\ln \Xi-\a\pp{\ln \Xi}{\a}-\b\pp{\ln \Xi}{\b}},\a=-\f{\mu}{kT},\b=\ff{kT}$$
平均分子数为
$$\bar{N} = \sum \f{\o_l}{e^{\a+\b\ep_l}-1}=-\pp{}{\a}\ln\Xi$$
涨落为
$$\bar{(N-\bar{N})^2}=-\pp{\bar{N}}{\a}=kT\pp{\bar{N}}{\mu}=\bar{N}^2\f{kT}{V}\kappa_T$$
当$\kappa$为有限值, 涨落很小. 但在一级相变的两相共存区和液气临界点, $\kappa\to\inf$, 涨落也很大. 两相共存是一种动态平衡, 液气临界点是形成分子集团. 对于粒子数涨落大的情形, 选取巨正则系综比正则系综方便, 但二者等价, 这是因为将整个系统或者将系统的一部分看做热力学系统得到的信息应该相同, 二者只是相当于选取了不同的自变量. 
\se{涨落理论}
\sub{涨落的准热力学理论}
对正则系综
$$\bar{\D T\cdot\D V}=0,\bar{(\D T)^2}=\f{kT^2}{C_V},\bar{(\D V)^2}=-kT\of{\pp{V}{p}}_T$$
$$\bar{\D T\cdot\D m}=0,\bar{(\D T)^2}=\f{kT^2}{C_m},\bar{(\D m)^2}=\f{kT}{\mu_0}\of{\pp{m}{H}}_T$$
对巨正则系综
$$\bar{\D N\cdot\D T}=0,\bar{(\D T)^2}=\f{kT^2}{C_V},\bar{(\D N)^2}=kT\of{\pp{N}{\mu}}_T$$
\sub{布朗运动}
朗之万方程($a$为小球半径,$\eta$为黏度):
$$m\dt[2]x=-\a\dt{x}+F(t)+\mathcal{F}(t)$$
若无外力可得($\a=6\pi a\eta$为黏滞系数)
$$\dt[2]{}\bar{x^2}+\f{\a}{m}\dt{}\bar{x^2}-\f{2kT}{m}=0$$
$$\bar{x^2}=\f{2kT}{\a}t$$
扩散方程:(D为扩散系数)
$$\pp{n}{t}=D\nabla^2n$$
爱因斯坦关系
$$D=\f{kT}{\a}$$
热噪声
$$L\dd{i}{t}=\mathcal{V}-Ri+V(t)$$
尼奎斯定理表明涨落电压不同频率分量统计独立. 且大小与$\o$无关, 称为白噪声:
$$\bar{V(\o)V^*(\o')}=\f{kTR}{\pi}\de(\o-\o')$$
光学黏胶与多普勒制冷\\
多普勒激光中朗之万方程为
$$m\dd{v_i}{t}=-\a v_i+F_i$$
其平衡温度为$kT\f{D_p}{\a}$, 估算可得最低温度$kT_D=\f{\hbar\Gamma}{2}\sim100\mu K$. 
磁光陷阱\\
朗之万方程为
$$m\dt{v_i}=-Kx_i-\a v_i+F_i(t)$$ 
\end{document}