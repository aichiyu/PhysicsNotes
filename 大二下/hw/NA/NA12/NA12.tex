\documentclass[UTF8]{ctexart}
\usepackage{../../template/homeworkTEMP/hw}
\setcounter{secnumdepth}{0}
\title{数值分析第十二次作业} 
\begin{document} 
\maketitle
\se{1}
$$z^{(1)}=Ax^{(0)} = \of{\ar{2\\4\\5}},\quad z^{(2)} = Ax^{(1)}/5 = \of{\ar{4/5\\17/5\\24/5}}$$
第2次近似值为$\l^{(2)} = 24/5$, 对应的特征向量为$z^{(2)}/\l^{(2)} = \of{\ar{1/6\\17/24\\1}}$
\se{2}
$$x_{k}= \f{y_{k-1}}{3}=\f{A}{3}x_{k-1}=\of{\f{A}{3}}^kx_0$$
当$k\to \inf$, $x_k$变为特征向量的倍数, 因此
$$\lim_{j\to \inf}x_{j+k} = \lim_{j\to \inf} \of{\f{A}{3}}^kx_j=\of{\f{\l_m}{3}}^kx_j$$
由于$|A|=3,\tr(A)=4$, 因此其特征值为$1,3$, 最大特征值为$\l_m=3$. 于是
$$\lim_{j\to \inf}x_{j+k}=\of{\f{\l_m}{3}}^kx_j=x_j$$
因此$x_j$存在极限, 且极限为$\l_m$对应的特征向量, 为$\of{\ar{1\\1}}$. 也可以根据Rayleigh商的性质得到$\lim_{k\to \inf}\tau_k=\l_m=3$. 
\se{3}
(1)\\
$$L=\begin{pmatrix}1&0&0\\1/2&1&0\\0&-1&1 \end{pmatrix},\quad U = \begin{pmatrix}2&0&2\\0&1&1\\0&0&-2\end{pmatrix}$$
(2)\\
$$L^{-1}=\begin{pmatrix}1&0&0\\-1/2&1&0\\-1/2&1&1 \end{pmatrix},\quad U^{-1} = \begin{pmatrix}1/2&0&1/2\\0&1&1/2\\0&0&-1/2\end{pmatrix}$$
    因此$$v_1=U^{-1}L^{-1}v_0 = \begin{pmatrix} 3/4\\3/4\\-1/4\end{pmatrix}$$
    因此$\l_3=1/(3/4)=4/3$
\end{document}