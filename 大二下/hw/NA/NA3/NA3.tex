
\documentclass[UTF8,9pt]{ctexart}
\usepackage{../../template/homeworkTEMP/hw}
\setcounter{secnumdepth}{0}
\title{数值分析第三次作业} 
\begin{document} 
\maketitle
\se{1}
(1)
$$
D^{-1}=(\ar[ccc]{
    1/\a & 0 & 0 \\
    0 & 1/\a & 0 \\
    0 & 0 & 2
})
$$
$$\ip J=I-D^{-1}A=
-(\ar[ccc]{
    0 & 2/\a & 1/\a\\
    2/\a & 0 & -1/\a\\
    2 & 2 & 0
})$$

(2)
 $\l =0, \f{2}{\a}, -\f{2}{\a} \ip rho = \f{2}{\a}$. 当$rho<1 \ip 2>\a$时收敛. 
\se{2}
$$(D-L)^{-1}=(\ar[ccc]{
    2 &  & \\ 
    1 & 1 & \\ 
    1 & 1 & -2
})^{-1}=-\ff{2}
(\ar[ccc]{
    -1&  & \\ 
    1& -2 & \\ 
    0& -1 & 1
})$$
$$G=(D-L)^{-1}U=\ff{2}(\ar[ccc]{
   0 & 1 & -1\\ 
   0 &  -1 & -1\\ 
   0 & 0 & -1
}),\ f=(D-L)^{-1}b=\ff{2}(\ar[c]{
   1\\ 
   1\\ 
   0
})$$
由于$|G|=0<1$, 收敛.
\se{3}
(1)
$$(D-L)^{-1}=(\ar[ccc]{
    2 &  & \\ 
    2 & 2 & \\ 
    0 & -1 & 2
})^{-1}=
(\ar[ccc]{
    1/2 &  & \\
    -1/2 & -1/2 & \\
    -1/4 & 1/4 & 1/2
})$$

$$G = (\ar[ccc]{
    0 & -a/2 & -1/2\\
    -1 & a/2-1 & (\left.1-a)/2\\
    -1/2 & a/4 & -(a-3)/4
})$$
$$\ar{
    \l_1 =& -1\\
    \l_2 =& a/8 - ((a + 1)(a + 25))^{1/2}/8 - 3/8\\
    \l_3 =& a/8 + ((a + 1)(a + 25))^{1/2}/8 - 3/8
}$$
令$\l_3=1$, 则$a=2$. 

(2)
令$\l_3=0$, 则$a=-\ff{2}$.

\se{4}
迭代公式等价于
$$x=D^{-1}(b+(L+U)(\ar{x_2\\x_1}))$$
则收敛的充要条件为: $\rho(D^{-1}(L+U))<1$. 
$$D^{-1}(L+U)=(\ar{
    0 & \f{a_{12}}{a_{11}}\\
    \f{a_{21}}{a_{22}} & 0
})$$
$$\l=\pm \f{a_{12}a_{21}}{a_{11}a_{22}}$$
则收敛条件为:
$$\rho = |\l|=\left|\f{a_{12}a_{21}}{a_{11}a_{22}}\right|<1 $$
\se{5}
设$I-\o A$的特征值为$\l'$, 收敛的充要条件为: $\rho(I-\o A)=\norm{I-\o A}_2<1 \iff |\l'_{max}|<1$. 满足$|(1-\l')I-\o A|=0$, 可得$\l'=\o\l-1$. \\
当$0<\o<\f{2}{\beta}$, $-1<\l'=\o\l-1<\o\beta-1<1 $. 即$|\l'|=\rho(I-\o A)<1$, 收敛.
\se{6}
(1) 上式等价于
$$x^{(k+1)}=\f{2I-\o A}{2I+\o A}x^{(k)}-\f{b\o}{I+\f{\o A}{2}}$$
则收敛条件为$\rho(\f{2I-\o A}{2I+\o A})<1$. 设$A$的特征值为$\l$, $\f{2I-\o A}{2I+\o A}$的特征值为$\l'$. 
$$\l'=\f{2-\l\o}{2+\l\o}=1-\f{2\l\o}{2+\l\o}$$
$$\l'=\f{2-\l\o}{2+\l\o}=-1+\f{4}{2+\l\o}$$
由于$\l,\o>0$, $-1<\l'<1$. 因此$\rho(\f{2I-\o A}{2I+\o A}) = |\l'_{max}|<1$, 收敛.

(2) 矩阵$A$的特征值为$1,3$.则其谱半径为$\rho = \ff{2}$. 渐进迭代收敛速度$R(A)=-\ln \rho = \ln 2$.
\se{7}
设$B$的特征值为$\l$, $(1-\o)E+\o B$的特征值为$\l'$. $\l'=1+(\l-1)\o$. 由题设, $-1<\l<1$, 因此:
$$-1<1+(-1-1)\o<\l'<1+(1-1)\o=1$$
即$-1<\l'<1$, $\rho=|\l'|_{max}<1$, 该迭代收敛. 
将$x^{ (k+1)}=Bx^{(k)}+f$代入该迭代, 
\def({\lparen}\def){\rparen}
$$\ar{
    x^{(k+1)} =& (1-\o )Ex^{(k)}+\o Bx^{(k)}+\o f\\
    =& (1-\o)Ex^{(k)}+\o x^{(k+1)}\\
    ( 1-\o )x^{(k+1)} =& (1-\o)Ex^{(k)}\\
    x^{(k+1)} =& x^{(k)}
}$$
\def({\ifmmode \left\lparen \else\lparen\fi} \def){\ifmmode \right\rparen \else\rparen\fi}
说明该迭代是$x^{(k+1)}=Bx^{(k)}+f$的解, 也是$Ax=b$的解.
\se{8}
(a)
$$Az_1^{(m+1)}=b_1-B(A^{-1}b_2-A^{-1}Bz_1^{(m-1)}) $$
$$\ip z^{(m+1)} = A^{-1}b_1-A^{-1}BA^{-1}b_2+A^{-1}BA^{-1}Bz_1^{(m-1)}$$
同理, 
$$z_2^{(m+1)} = A^{-1}b_2-A^{-1}BA^{-1}b_1+A^{-1}BA^{-1}Bz_2^{(m-1)}$$
收敛的充要条件为 
$$\sqrt{\rho(A^{-1}BA^{-1}B)} = \left|\f{\l_B}{\l_A}\right|_{max}<1$$

(b)
$$Az_1^{(m+1)}=b_1-B(A^{-1}b_2-A^{-1}Bz_1^{(m)}) \ip z_1^{(m+1)}=A^{-1}b_1-A^{-1}BA^{-1}b_2-A^{-1}BA^{-1}Bz_1^{(m)}$$
同理, 
$$z_2^{(m+1)}=A^{-1}b_2-A^{-1}BA^{-1}b_1-A^{-1}BA^{-1}Bz_1^{(m)}$$
收敛的充要条件为 
$$\rho(A^{-1}BA^{-1}B) = (\f{\l_B^2}{\l_A^2})_{max}<1$$

(3)\\
(a) 中收敛速度为$R_1=-\ln \rho_1 = |\l_A|-|\l_B|,\ R_2=-\ln\rho_2=2(|\l_A|-|\l_B|)=2R_1$, (b)的收敛速度是(a)的2倍.
\end{document}