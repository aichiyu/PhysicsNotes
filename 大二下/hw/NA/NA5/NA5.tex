
\documentclass[UTF8,9pt]{ctexart}
\usepackage{../../template/homeworkTEMP/hw}
\usepackage{listings}
\lstset{language=Matlab}
\setcounter{secnumdepth}{0}
\title{数值分析第五次作业} 
\begin{document} 
\maketitle
\se{3}
其差分为:
\begin{center}
\begin{tabular}{|c|cccc|}
        \hline
        阶数  & 0    & 1 & 2 & 3 \\
        \hline
        1   & -1   &   &   &   \\
        1.5 & 0.5  & 3 &   &   \\
        2   & 2.5  & 4 & 1 &   \\
        2.5 & 5    & 5 & 1 & 0 \\
        3   & 8    & 6 & 1 & 0 \\
        3.5 & 11.5 & 7 & 1 & 0\\
        \hline
\end{tabular}\end{center}
则Newton插值多项式为:
$$N(x)=-1+3(x-1)+(x-1.5)$$
\se{4}
设存在一点$x_{n+1},f(x_{n+1})$. 对这$n+1$个点进行插值, 
$$
\begin{array}{|c|c|c|c|c|c|}
        \hline
        \text{阶数} & 0          & 1                           & 2                                              & \cdots & n                                      \\ \hline
        x_1       & 0          &                             &                                                &        &                                        \\ \hline
        x_2       & 0          & 0                           &                                                &        &                                        \\ \hline
        \vdots    & \vdots     & \vdots                      & \ddots                                         &        &                                        \\ \hline
        x_n       & 0          & 0                           & 0                                              & \cdots &                                        \\ \hline
        x_{n+1}   & f(x_{n+1}) & \f{f(x_{n+1})}{x_{n+1}-x_n} & \f{f(x_{n+1})}{(x_{n+1}-x_n)(x_{n+1}-x_{n-1})} & \cdots & \f{f(x_{n+1})}{\Pi_1^n(x_{n+1}-x_{i})} \\ \hline
\end{array}
$$
$n+1$个点插值$n$次多项式可得到精确解, 因此
$$f(x)=N(x)= \f{f(x_{n+1})}{\prod_1^n(x_{n+1}-x_{i})}\prod_{i=1}^{n}(x-x_i)$$
求导:
$$f'(x_j)= \f{f(x_{n+1})}{\prod_1^n(x_{n+1}-x_{i})}\prod_{i=1,i\neq j}^{n}(x_j-x_i)$$
因此
$$\f{x_j^{n-1}}{f'(x_j)}=\f{\prod_1^n(x_{n+1}-x_{i})}{f(x_{n+1})}\cdot\ff{\prod_{i=1,i\neq j}^{n}(1-\f{x_i}{x_j})}$$
通过$f(x)$表达式可以发现$\f{\prod_1^n(x_{n+1}-x_{i})}{f(x_{n+1})}$正是$x$最高次项系数的倒数, 即$a_n^{-1}$. 即
$$\f{x_j^{n-1}}{f'(x_j)}=a_n^{-1}\cdot\ff{\prod_{i=1,i\neq j}^{n}(1-\f{x_i}{x_j})}$$
又由于
$$\left\{\ar{\dis\sum^n_{j=1} \ff{\prod_{i=1,i\neq j}^{n}(1-\f{x_i}{x_j})}=\f{\prod_{i,j=1}^{n}(1-\f{x_i}{x_j})}{\prod_{i,j=1}^{n}(1-\f{x_i}{x_j})}=1\\
\dis\sum^n_{j=1} \ff{x_j\prod_{i=1,i\neq j}^{n}(1-\f{x_i}{x_j})}=0
}\right.
$$
即$$\left\{\ar[l]{\dis\sum^n_{j=1} \f{x_j^{n-1}}{f'(x_j)}=a_n^{-1}\\
\dis\sum^n_{j=1} \f{x_j^{k}}{f'(x_j)}=0,\quad(k<n-1)}\right.$$
\se{5}
(1) $$P(x)=\a_0(x)y_0+\a_1(x)y_1+\a_2(x)y_2+\b_2(x)y'_0 = \a_0(x)-\a_2(x)$$
根据$\a(x_0)=1,\a(x_{1,2})=0$, 设$\a_0=(Ax+B)l_0$, 满足
$$\ar{
        Ax_0+B=1\\
        ((Ax+B)l_0 \left.(x) \, )' \big|_{x_0}=0
}$$
$$A=\f{3}{2},B=\f{5}{2}$$
$$\ip\a_0=\f{1}{4}x(3x+5)(x-1)$$
同理, 设$\a_2=\f{(x-x_0)^2(x-x_1)}{(x_2-x_0)^2(x_2-x_1)} = \ff{4}x(x+1)^2$.
则$$P(x)=\a_0(x)-\a_2(x)=\f{1}{4}x(3x+5)(x-1)-\ff{4}x(x+1)^2=-\f{3}{2}x+\f{x^3}{2}$$
(2) $$P(0.5)=-\f{11}{16},f(0)-P(0)=f(1)-P(1)=0$$
$$R_4(0.5)=\abs{\f{f^{(4)}(x)}{4!}(x-0)(x-1)(x+1)^2}_{x=0.5}<\ff{4!}\ff{2}\ff{2}(\f{3}{2})^2=\f{3}{128}$$
\se{6}
(1) 设$$ 
\begin{aligned} p(x)=& f(x_{0})+f\left[x_{0}, x_{1}\right](x-x_{0}) \\ &+f\left[x_{0}, x_{1}, x_{2}\right](x-x_{0})(x-x_{1}) \\ &+A(x-x_{0})(x-x_{1})(x-x_{2}) \end{aligned}
 $$
则$P(x)=-\f{3}{4}+(x-x_0)(x-x_1)+A(x-x_0)(x-x_1)(x-x_2)$.\\
代入$P'(x_1)=f'(x_1)$:
$$A=\f{f'(x_1)-(x_1-x_0)-(x_1-x_1)}{(x_1-x_0)(x_1-x_2)} = 0$$
即$$P(x)=-\f{3}{4}+(x+\ff{2})(x-\ff{2}) = x^2-1$$
(2) $$R(1) = \abs{f(1)-P(1)}=\abs{\f{f^{(4)}(x)}{4!}(x-2)(x-1/2)^2(x+1/2)}_{x=1}<\ff{64}$$
又由于$P(1)=0$
即$\abs{f(1)}<\ff{64}$.
\se{7}
(1) 给定节点$a\leq x_0<x_1<\cdots<x_n\leq b$, 若函数$S(x)\in C^3[a,b]$, 且每个小区间$[x_j,x_{j+1}]$上是4次多项式, 则称$S(x)$是节点$a\leq x_0<x_1<\cdots<x_n\leq b$上的4次样条函数. 若在节点上给定函数数值$f(x_j)=y_j$, 且满足$$S(x_j)=y_j,\quad j=0,1,\cdots,n.$$则称$S(x)$为4次样条插值函数. 

(2)
$$ 
S(x)=\left\{\begin{array}{ll}{x^{4}+2 x+1} & {0 \leq x \leq 1} \\ {(x-1)^{4}+a(x-1)^{3}+b(x-1)^{2}+c(x-1)+d} & {1 \leq x \leq 3}\end{array}\right.
 $$
$$S(1_-)=S(1_+)\ip 3=d    $$
$$S'(1_-)=S'(1_+)\ip 6=c    $$
$$S''(1_-)=S''(1_+)\ip 12=2b \ip b=6    $$
$$S'''(1_-)=S'''(1_+)\ip 24=6a\ip a=4    $$

\end{document}