\documentclass[UTF8,9pt]{ctexart}
\usepackage{../../template/homeworkTEMP/hw}
\setcounter{secnumdepth}{0}
\title{数值分析第九次作业} 
\begin{document} 
\maketitle
\se{1}
(1)\\
区间长度为2, 误差不超过1. 每次迭代误差减半, $1/2^{14}<10^{-4}$. 因此需要14次迭代. 

(2)\\
令$f(x)=s\sin x-1$. $f(1)\approx -0.16<0$. $f(0) =-1<0$. $f(1.5)\approx 0.5>0$. $f(1.25)\approx 0.2>0$. 因此3等分后近似根在$[1,1.25]$区间. 取$x=1.125$.
\se{2}
设$\vp(x) = x-\l f(x)$. $-1<\vp'(x) =1 - \l f'(x) <1$. 因此对包含根的一个区间$x\in[a,b]$, 有$\vp(x)\in [a,b]$. 又由于$\vp(x)<1$, 因此对$x\in [a,b]$, 迭代收敛. 
\se{3}
当$\vp'(x)>1$, 对于$x_k>x^*$, 令$x_k=x^*+\d x_k $. \\
此时有$x_{k+1}= \vp(x_k) = \vp(x^*)+\vp'(x^*)\d x_k>x^*+\d x_k=x_k$. 即$x_{k+1}>x_k>x^*$. 不收敛, \\
当$\vp'(x)<1$, 对于$x_k<x^*$, 令$x_k=x^*-\d x_k $. \\
此时有$x_{k+1}= \vp(x_k) = \vp(x^*)+\vp'(x^*)\d x_k<x^*-\d x_k=x_k$. 即$x_{k+1}<x_k<x^*$. 不收敛, \\
\se{4}
$$\vp'(x) = \ff{3}(2x-e^x), \vp''(x)=\ff{3}(2-e^x)$$
$\vp'(x)$在$x=\ln 2$取到极大值. 因此在[0,0.5]单调递增. 又由于$\vp'(0) = -\ff{3},\ \vp'(0.5)<0$. 因此$\vp(x)$在$[0,0.5]$单调递减. 且区间内$|\vp'(x)|<\ff{3}$.\\
又由于$\vp(0) = \ff{3}, \vp(0.5) = \f{2.25-\sqrt{e}}{3} \in[0,0.5]$.
因此迭代收敛. 
\se{5}
(1) \\
格式一\\
$\vp(x) = -\ln(x)$, $\vp'(x)=-\ff{x}$, 在$x_0>0$区间内, $\vp'(x)=-\ff{x}\in(-2,-1.67)$. 根据3题中的证明, 不收敛. \\
格式二\\
$\vp(x) = e^{-x}$, $\vp'(x) =-e^{-x}$. 在$x_0>0$区间内, $\vp'(x)=\in(-1,0)$. 且$\vp(x)\in[\vp(0.6),\vp(0.5)] = [0.549,0.606]$,在$x$区间内, 因此收敛. \\

(2) 
$$\ar{
        l_0(y) \x \f{y-0.02695}{0.10653-0.02695}\f{y+0.05119}{0.10653+0.05119}\\
        l_1(y) \x \f{y-0.10653}{0.02695-0.10653}\f{y+0.05119}{0.02695+0.05119}\\
        l_2(y) \x \f{y-0.10653}{-0.05119-0.10653}\f{y-0.02695}{-0.05119-0.02695}
}$$
$$L(y) = l_0x_0+l_1x_1+l_2x_2 = 0.589335 - 1.02803 y - 16.0127 y^2$$
$L(0)\approx0.59$为近似根.
\se{6}
(1)  将方程变换为: $$\ff{6}\sin x+\f{2}{3} = x$$
并令$\vp(x) = \ff{6}\sin x+\f{2}{3}$. 显然$\vp'(x)=\ff{6}\cos x<\ff{6}$. 又由于$\vp(0)=\ff{2}{3}>0,\vp(2\pi)=\f{2}{3}<2\pi$. 这意味着方程在$(0,2\pi)$之间有且只有一个根. 

(2)  $\vp(x)$如上一小问定义. 有$|\vp'(x)|<\ff{6}$. 因此在任意一个包含根的区间(a,b)上, $\vp(a),\vp(b)\in(a,b)$. 迭代在(a,b)上收敛. 

(3)  取初始点$x_0$. 
$$ 
x_{n+1}=x_{n}-\frac{f\left(x_{n}\right)}{f^{\prime}\left(x_{n}\right)}
 $$
其中
$$\hua{
        f(x) \x \frac{1}{4} \sin x-\frac{3}{2} x+1\\
        f'(x) \x \ff{4}\cos x-\f{3}{2}
}$$
$$\ip x_{n+1} = x_n- \f{\frac{1}{4} \sin x_n-\frac{3}{2} x_n+1}{\ff{4}\cos x_n-\f{3}{2}}$$
\se{7}
对于方程$f(x)=x^{n}-a=0$: 
$$x_{k+1} = x_k-\f{x_k^n-a}{nx_k^{n-1}} = \f{n-1}{n}x_k+\f{a}{nx_k^{n-1}}$$
$$\lim_{k\to\infty}\f{\sqrt[n]{a}-x_{k+1}}{(\sqrt[n]{a}-x_k)^2} = \f{f''(\sqrt[n]{a})}{2f'(\sqrt[n]{a})} = \f{n(n-1)a^{\f{n-2}{n}}}{2na^{\f{n-1}{n}}} = \f{(n-1)a^{-\ff{n}}}{2}$$
对于方程$f(x)=1-\f{a}{x^n}=0$:
$$x_{k+1} = x_k - \f{1-\f{a}{x_k^n}}{anx^{-n-1}} = \f{n+1}{n}x_k-\ff{anx_k^{-n-1}}$$
$$\lim_{k\to\infty}\f{\sqrt[n]{a}-x_{k+1}}{(\sqrt[n]{a}-x_k)^2} = \f{f''(\sqrt[n]{a})}{2f'(\sqrt[n]{a})} = \f{-an(n+1)a^{-\f{n+2}{n}}}{2ana^{-\f{n+1}{n}}} = \f{n+1}{2}a^{-\ff{n}}$$
\se{8}
$$f'(x) = (1+\a) x^\a -1$$
$$'f(0) = -1,\quad f'(1) = \a$$.
因此两个根均为单根, 又由于$f(x)$无穷阶可导, $p_1=p_2=2$.
\end{document}