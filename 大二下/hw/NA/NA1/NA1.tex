
\documentclass[UTF8,9pt]{ctexart}
\usepackage{../../template/homeworkTEMP/hw}
\setcounter{secnumdepth}{0}
\title{数值分析第一次作业} 
\begin{document} 
\maketitle
1. 误差限为$\epsilon = \ff{2}\e{-4}>I^*-I=0.000012$, 则有效数字为4.

2. 
$$\epsilon_{rV}=\f{\dd{V}{R}e_R}{V}=3\f{e_R}{R} \ip \f{e_R}{R}=0.33\%$$

3.
$$x_1=28+\sqrt{783}\approx 28+27.982 = 55.982$$
$$x_2=28-\sqrt{783}=\ff{28+\sqrt{783}}\approx \ff{28+27.982}\approx 0.017863$$

4. \\
(1) 都为6.\\
(2) 真实值为$\sqrt{2018}-\sqrt{2017}\approx0.0111317$. $x-y=0.0112\approx0.011$, 2位有效数字.\\
(3) $\ff{x+y}=1/89.8332=0.0111317$. 6位有效数字.

 5. 记$\de=\sqrt{783}-27.982\approx0.000137$, 每次计算产生误差$\de/100$. 100次后误差为$\de$, 则误差限为$\epsilon=0.5\e{-3}<\de$. 有3位有效数字.

6. 设$x=\sqrt{2},\ e_x=\sqrt{2}-1.4$. 
$$\begin{array}{cc}
    \ar{
        e_1&=\left|\pp{(1+x)^{-6}}{x}\right|e_x\\
        &=6(1+x)^{-7}e_x\\
        \approx& 1.3\e{-2}e_x\\
        e_2&=\left|\pp{(3-2x)^3}{x}\right|e_x\\
        &=6(3-2x)^{2}e_x\\
        \approx& 1.2e_x
    }
    &
    \ar{
        e_3&=\left|\pp{(3+2x)^{-3}}{x}\right|e_x\\
        &=3(3+2x)^{-4}e_x\\
        \approx& 2.6\e{-3}e_x\\
        e_4&=\left|\pp{99-70x}{x}\right|e_x\\
        &\\
        =&70e_x
    }
\end{array}$$

可知$\ff{(3+2\sqrt{2})^3}$式结果最好.

\end{document}