
\documentclass[UTF8,9pt]{ctexart}
\usepackage{../../template/homeworkTEMP/hw}
\setcounter{secnumdepth}{0}
\title{数值分析第二次作业} 
\begin{document} 
\maketitle
\se{1}
$$\begin{pmatrix}
    1 & -1 & 1 \\
    5 & -4 & 3 \\
    2 & 1  & 1
\end{pmatrix}
\begin{pmatrix}
    x_1 \\
    x_2 \\
    x_3
\end{pmatrix}
=
\begin{pmatrix}
    -4 \\
    -12 \\
    11
\end{pmatrix}
$$
$$\begin{pmatrix}
    1 & -1 & 1 & -4 \\
    5 & -4 & 3 & -12 \\
    2 & 1  & 1 & 11
\end{pmatrix}
\ip
\begin{pmatrix}
    1 & -1 & 1 & -4 \\
    0 & 1 & -2 & 8 \\
    0 & 3  & -1 & 19
\end{pmatrix}
\ip
\begin{pmatrix}
    1 & -1 & 1 & -4 \\
    0 & 1 & -2 & 8 \\
    0 & 0  & 5 & -5
\end{pmatrix}
$$
$$\ip \left\{\ar{
x_1 =& 5\\  
x_2 =& 10\\
x_3 =& 1
}\right.$$
\se{Q2}
对$A_2$的各个元素: $a_{ij}^{(2)}=a_{ij}-\f{a_{1j}}{a_{11}}a_{i1} = a_{ji}^{(2)}$, 因此$A_2$为对称阵. 又由于$A_2$的主元均为$A$的主元, $A$是正定阵所以所有主元大于0, 因此$A_2$的主元也大于0, 因此$A_2$正定.
\se{Q3}
方程对应的增广矩阵为:
$$\begin{pmatrix}
    -2 & 1 & 4  & 16 \\
    6  & 0 & -2 & -5 \\
    0  & 8 & -1 & 4  \\
\end{pmatrix}
\ip
\begin{pmatrix}
    6  & 0 & -2 & -5 \\
    0  & 8 & -1 & 4  \\
    0 & 1 & \f{10}{3} & \f{43}{3}    
\end{pmatrix}
\ip
\begin{pmatrix}
    6  & 0 & -2 & -5 \\
    0  & 8 & -1 & 4  \\
    0 & 0 & \f{83}{24} & \f{83}{6}    
\end{pmatrix}
$$
$$\ip \left\{\ar{
    x_1 =& 1/2\\  
    x_2 =& 1\\
    x_3 =& 4
    }\right.$$
\se{Q4}
$$A=
\begin{pmatrix}
    1    & 0    & 0 \\
    -1/2 & 1    & 0 \\
    1/2  & -7/5 & 1
\end{pmatrix}
\begin{pmatrix}
    2 & -1 & 1    \\
    0 & -5/2 & 7/2    \\
    0 & 0  & 27/5  
\end{pmatrix}$$
令$Ux=y,\ Ly=b$ $$\ip y=
\begin{pmatrix}
    4    \\
    7    \\
    69/5
\end{pmatrix},\quad x=
\begin{pmatrix}
    10/9 \\
    7/9  \\
    23/9  
\end{pmatrix},
$$
\se{Q5}
设$G=
\begin{pmatrix}
    G_{11} & G_{1j}^T \\
    G_{1j}      & G'
\end{pmatrix}$
$G_{11}=A_{11}=1, G_{1j}=A_{1j}, G'G'^T=
\begin{pmatrix}
    4 & 4 \\
    4 & 13
\end{pmatrix}$. 于是有:
$$G=
\begin{pmatrix}
    1  & 0 & 0 \\
    -1 & 2 & 0 \\
    2  & 2 & 3
\end{pmatrix}
$$
\se{Q6}
$$(Ax,x)^{1/2}=[(Ax)^Tx]^{1/2}=(x^TA^Tx)^{1/2}=(x^TAx)^{1/2}$$
1. 非负性: 由于$A$正定,因此对任意$x \neq 0$, $x^TAx>0,\ (x^TAx)^{1/2}>0$, 当且仅当$x=0,\ \left\|x\right\|_A=0$.\\
2. 正齐次性: $\left\|kx\right\|_A=\sqrt{(kx)^TA(kx)}=|k|\ \sqrt{x^TAx}=|k|\ \left\|x\right\|_A$.\\
3. 三角不等式: $y^TAx=(x^TAy)^T$, 由于$x^TAy$是一个标量, 标量的转置等于它自身, 于是 $y^TAx=x^TAy$. \\
$\left\|x+y\right\|^2_A=(x^T+y^T)A(x+y)=x^TAx+y^TAy+2x^TAy$.\\
$(\left\|x\right\|_A+\left\|y\right\|_A)^2=x^TAx+y^TAy+2\sqrt{x^TAxy^TAy}$.\\
要证$\left\|x+y\right\|^2_A \leq (\left\|x\right\|_A+\left\|y\right\|_A)^2$, 即证$x^TAy \leq \sqrt{x^TAxy^TAy}$. 当$x^TAy \leq 0$时显然成立. \\
当$x^TAy>0$, 即证$x^TAyx^TAy \leq x^TAxy^TAy$. 由于$x,y$是大小相等的矢量, $xy^T=yx^T$, 因此$x^TAyx^TAy = x^TAxy^TAy$.\\

该定义满足以上三个性质, 因此$\left\|x\right\|_A$是范数.
\se{Q7}
(1) $\left\|A+B\right\|=\left\|A(I+A^{-1}B)\right\| \leq \left\|A\right\|\ \left\|I+A^{-1}B\right\| \leq \left\|A\right\|(\left\|I\right\|+\left\|A^{-1}B\right\|)$.\\
(2) $\left\|I-(I+A^{-1} B)^{-1}\right\| \leq 1-\left\|(I+A^{-1} B)^{-1}\right\| \leq 1-\ff{\left\|I+A^{-1} B\right\|} \leq 1-\ff{1+\left\|A^{-1} B\right\|}=\f{\left\|A^{-1} B\right\|}{1-\left\|A^{-1} B\right\|} $\\
(3)$\frac{\left\|A^{-1}-(A+B)^{-1}\right\|}{\left\|A^{-1}\right\|} \leq \f{\left\|A^{-1}\right\|-\left\|(I+A^{-1}B)^{-1}A^{-1}\right\|}{\left\|A^{-1}\right\|} \leq \f{\left\|A^{-1}\right\|-\left\|(I+A^{-1}B)^{-1}\right\|\ \left\|A^{-1}\right\|}{\left\|A^{-1}\right\|}=1-\left\|(I+A^{-1}B)^{-1}\right\|$, 代入(2)中结论, $\frac{\left\|A^{-1}-(A+B)^{-1}\right\|}{\left\|A^{-1}\right\|} \leq 1-\ff{1+\left\|A^{-1} B\right\|}=\f{\left\|A^{-1} B\right\|}{1-\left\|A^{-1} B\right\|}$.\\
\qqed
\se{Q8}
$A^TA=\begin{pmatrix}
    3  & 0 & 0 \\
    0 & 6 & 0 \\
    0  & 0 & 2
\end{pmatrix}$, 则$A$有最大特征值$\l_m=6$, $A^{-1}$有最大特征值$\l'_m=1/2$
$$cond(A)_2=\sqrt{\l_m\cdot\l'_m}=\sqrt{3}$$
\end{document}