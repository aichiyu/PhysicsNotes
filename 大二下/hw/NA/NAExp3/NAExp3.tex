\documentclass[UTF8,9pt]{ctexart}
\usepackage{../../template/homeworkTEMP/hw}
\usepackage{listings}
\setcounter{secnumdepth}{0}
\usepackage{color}
\definecolor{dkgreen}{rgb}{0,0.6,0}
\definecolor{gray}{rgb}{0.5,0.5,0.5}
\definecolor{mauve}{rgb}{0.58,0,0.82}
\lstset{frame=tb,
        language=Python,
        aboveskip=3mm,
        belowskip=3mm,
        showstringspaces=false,
        columns=flexible,
        basicstyle={\small\ttfamily},
        numbers=none,
        numberstyle=\tiny\color{gray},
        keywordstyle=\color{blue},
        commentstyle=\color{dkgreen},
        stringstyle=\color{mauve},
        breaklines=true,
        breakatwhitespace=true,
        tabsize=3
}
\title{数值分析实验三} 
\begin{document} 
\maketitle
\se{一}
\sub{1}
显式Euler公式为$$ 
y_{n+1}=y_{n}+h f\left(x_{n}, y_{n}\right)
 $$
其中$f(x_n,y_n) = -50y$
该微分方程的解析解为$y(x)=100e^{-50x}$. \\
不同步长的解与精确解的关系如下图所示
\putfig[0.8]{1.png}
可以看到h=0.03时的解接近精确解, h=0.04时在精确解上下波动, 不收敛也不继续发散, 而当h=0.05, 出现数值不稳定, 可以看到$h$变动极大且$|h|$趋于无穷大. \\
实际上, 利用Euler方法的稳定性条件$$ 
|1+\lambda h| \leq 1, \l = -50
 $$可以得到方法稳定的条件为$0\leq h \leq 0.04$. 与实验相符.
\sub{2}
隐式Euler公式为$$ 
y_{n+1}=y_{n}+h f\left(x_{n+1}, y_{n+1}\right)
 $$
在此题中可显式地化为$$y_{n+1}=\ff{1+50h}y_n$$
不同步长的解与精确解的关系如下图所示
\putfig[0.8]{2.png}
此时所有h均不会出现数值不稳定. \\
对于隐式Euler公式, 稳定性条件为$$ 
\left|\frac{1}{1-\lambda h}\right| \leq 1
 $$当$\l<0$, 对一切$h$均稳定, 且$h$越小, 与精确解符合得越好. 这与实验相符.
\se{二}
方程组为$$ 
\left\{\begin{array}{l}{y^{\prime}(t)=-2 x(t)} \\ {x^{\prime}(t)=\frac{9}{2} y(t)} \\ {x(0)=3, y(0)=2}\end{array}\right.
 $$
定义$r(t) = \begin{pmatrix}x\\y \end{pmatrix}$, 则
$$r'(t) = Ar(t) = \begin{pmatrix}0&\f{9}{2}\\-2&0 \end{pmatrix}\begin{pmatrix}x\\y \end{pmatrix},\quad r(0) = \begin{pmatrix}3\\2 \end{pmatrix}$$
梯形公式为$$ 
r_{n+1}=r_{n}+\frac{h}{2}\left[r'\left(t_{n}\right)+r'\left(t_{n+1})\right]\right. \ip r(t_{n+1})=(E-\f{h}{2}A)^{-1}(E+\f{h}{2}A)r(t_n)
 $$
显式Euler法为
$$r(t_{n+1}) = r(t_n)+hr'(t_n) \ip r(t_{n+1}) = (E+hA)r(t_n)$$
隐式Euler法为$$ 
r_{n+1}=r_{n}+hAr_{n+1}
 $$
在此题中可显式地化为$$r_{n+1}=(E-hA)^{-1}r_n$$
按照Runge-Kutta法,
$$r_{{n+1}}=r_{n}+{h \over 6}(k_{1}+2k_{2}+2k_{3}+k_{4})$$
其中
$$ \left\{
\begin{aligned} 
k_{1} &=f\left(t_{n}, r_{n}\right) \\ k_{2} &=f\left(t_{n}+\frac{h}{2}, r_{n}+\frac{h}{2} k_{1}\right) \\ k_{3} &=f\left(t_{n}+\frac{h}{2}, r_{n}+\frac{h}{2} k_{2}\right) \\ k_{4} &=f\left(t_{n}+h, r_{n}+h k_{3}\right) \end{aligned}\right.
 $$
取步长为0.1, 绘制$t\in[0,10]$的图像, 不同方法的解如下图所示:
\putfig[0.8]{3.png}
三个图分别为$x-t,y-t,x-y$的关系图. 可以非常清楚地看到仅有梯形法保持了相平面轨迹不变. 显式Euler法和Runge-Kutta法轨迹逐渐远离原点, 隐式Euler法轨迹逐渐收缩到原点.
\sub{原因解释}
在此题中, 微分方程的解为
$$\f{x^2}{9}+\f{y^2}{4}=2$$
做变量代换$r' = \begin{pmatrix}1/3&0\\0&1/2 \end{pmatrix}r = Qr$. 则方程曲线变为圆. 相平面稳定要求半径不变, 即$r^{'2}_n = 2$. \\
设$r_{n+1} = Br_n$. 变量代换后矩阵$B$也要做变换, $B' = QBQ^{-1}$. $r^{'2}_n = 2$要求$\norm{B'}=1$. \\
对于梯形公式, 代入$r(t_{n+1}) = [(E-\f{h}{2}A)^{-1}(E+\f{h}{2}A)]r(t_n) = Br(t_n)$. 当$h=0.1$, 
$$B' = \begin{pmatrix}
0.9560   & 0.2934\\
-0.2934  &  0.9560\\
\end{pmatrix},
\quad
\norm{B'}_2 = 1$$
因此相平面稳定. \\
对于显式Euler法, $B= E+hA$, 当$h=0.1$, 
$$B' = \begin{pmatrix}
        1   & 0.3\\
        -0.3  &  1\\
        \end{pmatrix},
        \quad
        \norm{B'}_2 = 1.044>1$$
因此相平面不稳定且距离原点越来越远. \\
对于隐式Euler法, $B= (E-hA)^{-1}$, 当$h=0.1$, 
$$B' = \begin{pmatrix}
        0.9174   & 0.2752\\
        -0.2752  &  0.9174\\
        \end{pmatrix},
        \quad
        \norm{B'}_2 = 0.9578<1$$
因此相平面不稳定. 且距离原点越来越近. \\
对于Runge-Kutta法, 较难用$r_{n+1} = Br_n$形式表达, 但可以验证其并不满足$\norm{B'}_2=1$:\\
在本实验中$|r'_0| = \sqrt{2}$, 而计算可得$r_1 = [3.7156625, 1.46655833]^T$, 
$$r'_1 = Qr_1 = \begin{pmatrix}
1.2386\\
0.7333\end{pmatrix},\quad |r'_1| = 1.4393\neq|r'_0|
$$
因此不具有相平面稳定性. 
\se{三}
待解的方程为$$ 
\left\{\begin{array}{l}{\frac{\mathrm{d}^{2} \theta(t)}{\mathrm{d} t^{2}}+\sin \theta=0, \quad 0<t \leq 10} \\ {\theta(0)=\frac{\pi}{3}} \\ {\left.\frac{\mathrm{d} \theta(t)}{\mathrm{d} t}\right|_{t=0}=-\frac{1}{2}}\end{array}\right.
 $$
设$x=\t,y=\t'$, 则$x'=y, y' = -\sin x$.\\
隐式Euler公式为$$x_{n+1} = x_n+hy_{n+1}$$
$$y_{n+1} = y_n-h\sin x_{n+1}$$
将$x_n,y_n$看做常数, 可写出迭代格式
$$x_{n+1}=x_n+hy_n-h^2\sin x_{n+1}$$
$$y_{n+1} = y_n - h\sin(x_n+hy_{n+1})$$
当$x_{n+1}^{(k)}-x_{n+1}^{(k-1)}<\ep$, $y_{n+1}^{(k)}-y_{n+1}^{(k-1)}<\ep$, 可解出$x_{n+1}=x_{n+1}^{(k)}, y_{n+1}=y_{n+1}^{(k)}$.\\
梯形公式为$$ 
 r_{n+1}=r_{n}+\frac{h}{2}\left[r'\left(t_{n}\right)+r'\left(t_{n+1}\right)\right]$$
 可以解得
 $$ x_{n+1}=x_n+\f{h}{2}\of{y_n+y_{n+1}}$$
 $$y_{n+1} = y_n+\f{h}{2}\of{-\sin x_n-\sin x_{n+1}}$$
 将$x_n,y_n$看做常数, 可写出迭代格式
$$x_{n+1}=x_n+\f{h}{2}\of{2y_n+\f{h}{2}\of{-\sin x_n-\sin x_{n+1}}}$$
$$y_{n+1} = y_n+\f{h}{2}\of{-\sin x_n-\sin \of{x_n+\f{h}{2}\of{y_n+y_{n+1}}}}$$
当$x_{n+1}^{(k)}-x_{n+1}^{(k-1)}<\ep$, $y_{n+1}^{(k)}-y_{n+1}^{(k-1)}<\ep$, 可解出$x_{n+1}=x_{n+1}^{(k)}, y_{n+1}=y_{n+1}^{(k)}$.\\
绘制$t\in[0,10]$的图像, $t-\t$图如下图所示:
 \putfig[0.8]{4.png}
 本题中单摆的能量为
 $$\ep = \ff{2}\of{\dt{\t}}^2+\cos\t = \ff{2}y^2+\cos x$$
两种方法计算下能量如下图所示
\putfig[0.8]{5.png}
可以看到梯形公式满足能量守恒, 但隐式Euler法能量逐渐降低. 这与上一题的计算相符. 
\newpage
\se{Code(Python)}
 \begin{lstlisting}
import numpy as np
import matplotlib.pyplot as plt
from scipy.optimize import fsolve
plt.rcParams['font.sans-serif']=['SimHei'] #用来正常显示中文标签  
plt.rcParams['axes.unicode_minus']=False #用来正常显示负号
############   第一题 1   ##################
def f(y): return -50*y
for h in [0.05,0.04,0.03]:
    x,y = np.arange(0,1,h),np.arange(0,1,h)
    y[0]=100
    for i in range(0,x.size-1): y[i+1] = y[i] +h*f(y[i])
    plt.plot(x,y)
plt.plot(x,100*np.exp(-50*x))
plt.legend(labels = ['h=0.05', 'h=0.04', 'h=0.03', 'Exact'], loc = 'best')
plt.show()
############   第一题 2   ##################
for h in [0.05,0.04,0.03]:
    x,y = np.arange(0,1,h),np.arange(0,1,h)
    y[0]=100
    for i in range(0,x.size-1): y[i+1] = y[i]/(1+50*h)
    plt.plot(x,y)
plt.plot(x,100*np.exp(-50*x))
plt.legend(labels = ['h=0.05', 'h=0.04', 'h=0.03', 'Exact'], loc = 'best')
plt.show()
############   第二题   ##################
h=0.1
A = np.matrix([[0,9/2],[-2,0]])
def main(B):
    r = list(np.arange(0,10,h))
    r[0] = np.matrix([[3],[2]])
    for i in range(0,len(r)-1): r[i+1] = B*r[i]
    r = np.array(r).T
    plt.subplot(221)
    plt.plot(np.arange(0,10,h),r[0][0])
    plt.subplot(222)
    plt.plot(np.arange(0,10,h),r[0][1])
    plt.subplot(223)
    plt.plot(r[0][0],r[0][1])
main((np.eye(2)-A*h/2)**(-1)*(np.eye(2)+A*h/2)) #梯形法
main(np.eye(2)+h*A) #显式Euler法
main((np.eye(2)-h*A)**(-1)) #隐式Euler法
def RK4(rn): #Runge-Kutta法
    k1 = A*rn
    k2 = A*(rn+h*k1/2)
    k3 = A*(rn+h*k2/2)
    k4 = rn + h*k3
    return rn+h*(k1+2*k2+2*k3+k4)/6
r = list(np.arange(0,10,h))
r[0] = np.matrix([[3],[2]])
for i in range(0,len(r)-1): r[i+1] = RK4(r[i])
r = np.array(r).T
plt.subplot(221)
plt.title('x-t')
plt.plot(np.arange(0,10,h),r[0][0])
plt.legend(labels = ['梯形法','显式Euler法','隐式Euler法','Runge-Kutta法'], loc = 'best')
plt.subplot(222)
plt.title('y-t')
plt.plot(np.arange(0,10,h),r[0][1])
plt.subplot(223)
plt.title('x-y')
plt.plot(r[0][0],r[0][1])
plt.show()
############   第三题   ##################
h,ep,t = 0.02,1.e-10,100
def euler(xn,yn):# Euler 法
    global h,ep
    x1,y1,x2,y2 = xn,yn,-1,-1
    while abs(x2-x1)>ep or abs(y2-y1)>ep:
        x1,y1 = x2,y2
        x2,y2 = xn + h*yn-h*h*np.sin(x1),yn -h*np.sin(xn+h*y1)
    return (x2,y2)
def ech(xn,yn):# 梯形法
    global h,ep
    x1,y1,x2,y2 = xn,yn,-1,-1
    while abs(x2-x1)>ep or abs(y2-y1)>ep:
        x1,y1 = x2,y2
        x2,y2 = xn + 0.5*h*(2*yn-0.5*h*(np.sin(xn)+np.sin(x1))),yn - 0.5*h*(np.sin(xn)+np.sin(xn+0.5*h*(yn+y1)))
    return (x2,y2)
for func in [euler, ech]:
    r = list(np.arange(0,t,h))
    r[0] = [np.pi/3,-0.5]
    for i in range(0,len(r)-1): r[i+1] = func(r[i][0], r[i][1])
    r = np.array(r).T
    plt.subplot(211)
    plt.plot(np.arange(0,t,h),r[0])
    plt.subplot(212)
    plt.title('能量')
    plt.plot(np.arange(0,t,h),0.5*r[1]*r[1]-np.cos(r[0]))
plt.legend(labels = ['Euler法','梯形法'], loc = 'best')
plt.show()
\end{lstlisting}
\end{document}