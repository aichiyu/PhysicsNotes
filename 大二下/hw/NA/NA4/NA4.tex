
\documentclass[UTF8,9pt]{ctexart}
\usepackage{../../template/homeworkTEMP/hw}
\usepackage{listings}
\lstset{language=Matlab}
\setcounter{secnumdepth}{0}
\title{数值分析第四次作业} 
\begin{document} 
\maketitle
\se{计算题目部分}
\se{1}
$$l_2=\f{x-1}{-2}\f{x-2}{-3}=\f{(x-1)(x-2)}{6}$$
$$l_3=\f{x-1}{1}\f{x+1}{3}=\f{(x-1)(x+1)}{3}$$
$$\ip L = -\f{(x-1)(x-2)}{2}+\f{4(x-1)(x+1)}{3}$$
\se{2}
$e^0=1,\quad e^{0.5}=1.65, \quad e^1-1$. $R_2(x)=\frac{f^{(2)}(\xi)}{6} \omega_{3}(x)\leq \frac{e}{6} \omega_{3}(x)$. 取$x=\f{3-\sqrt{3}}{6}$时$\omega_{3}(x)$最大为$0.48$. 则$R_2(x)<0.22$
\newpage
\se{数值实验部分}
\se{1}
需求解的方程组为
$$ 
L_{h} u^{h}=h^{2} f^{h},\ h=0.1
 $$
此时$N=1/h=10$. 取初始$\bm{u}=\bm{0}$.
\begin{center}
       \begin{tabular}{c|ccc}
               & 是否收敛 & 迭代次数 & 与精确解误差 (两者之差的无穷范数) \\
               \hline
       Jacobi  & 是    & 217  & 0.0082                \\
       Seidel  & 是    & 15   & 0.9759             \\
       SOR 1.2 & 是    & 25   & 0.9759             \\
       SOR 1.3 & 是    & 32   & 0.9759             \\
       SOR 1.9 & 是    & 309  & 0.9759             \\
       SOR 0.9 & 是    & 12   & 0.9759            
       \end{tabular}
\end{center}
\se{2}
\begin{center}
        \begin{tabular}{c|c|c|c}
                N & h                 & 迭代次数 & 与真值误差               \\
                \hline
                3 & 0.333333333333333 & 20   & 0.0724662490584273  \\
                4 & 0.250000000000000 & 38   & 0.0530272790515429  \\
                5 & 0.200000000000000 & 59   & 0.0303503805496695  \\
                6 & 0.166666666666667 & 84   & 0.0231571309052523  \\
                7 & 0.142857142857143 & 111  & 0.0161068074513148  \\
                8 & 0.125000000000000 & 144  & 0.0129394175506303  \\
                9 & 0.111111111111111 & 178  & 0.00989279695986134
                \end{tabular}
\end{center}
\se{3}
第一题中, 所有迭代方法均收敛, Jacobi迭代法收敛较慢, 这是因为Jacobi矩阵谱半径$\rho_J=0.9511$, 接近1. 而Seidel方法的谱半径$\rho_S=0.5$. 收敛速度之比为$\ln(0.9511)/\ln(0.5)=0.07 \approx 15/217$. 在SOR方法中, $\o=0.9$处收敛最快.

第二题中, 所取的几种情况Jacobi迭代法均收敛, 且随着$h$减小迭代次数增加. 这是因为当$h=\f{\sqrt{2}}{\pi}\approx 0.45$时谱半径为0, 收敛的最快, 随着$h$减小, 谱半径增大, 收敛变慢, 但谱半径总是小于1的, 因此总是收敛. 
\end{document}