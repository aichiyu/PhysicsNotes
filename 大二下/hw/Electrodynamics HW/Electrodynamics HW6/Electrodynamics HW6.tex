\documentclass[UTF8,9pt]{ctexart}
\usepackage{../../template/homeworkTEMP/hw}
\setlength{\arraycolsep}{1pt}
\setcounter{secnumdepth}{0}
\title{The 6th HW of Electrodynamics} 
\begin{document} 
\maketitle
\se{Q1}
$n=1,\ a=0,\ \a=\pi/3$
$$ 
\dd{z_1}{z_2}=C_{1} \prod_{i=1}^{n}(z_{2}-a_{i})^{\a_i/\pi-1} = C_1z_2^{-2/3}
 $$
$$\ip z_1=3C_1z_2^{1/3}+C_2$$
$$ 
W=i|\vec{E}| z_{2}=iC_3|\vec{E}|z_1^3=(y^3-3x^2y)+i(x^3-3xy^2)
 $$
 $$ \ip \phi = (y^3-3x^2y),\ \psi = x^3-3xy^2$$
 \se{Q2}
\sub{(a)}\def({\lparen}\def){\rparen}
该分布满足Laplace方程:
$$ 
\frac{1}{\rho} \frac{\partial}{\partial \rho}\left(\rho \frac{\partial \Phi}{\partial \rho}\right)+\frac{1}{\rho^{2}} \frac{\partial^{2} \Phi}{\partial \phi^{2}}=0
 $$
其分离变量解为:
$$ 
\Phi(\rho, \phi)=\left(a_{0}+b_{0} \ln \rho\right)\left(A_{0}+B_{0} \phi\right)+\sum_{\nu \neq 0}\left(a_{\nu} \rho^{\nu}+b_{\nu} \rho^{-\nu}\right)\left(A_{\nu} e^{i \nu \phi}+B_{\nu} e^{-i \nu \phi}\right)
$$
代入边界条件$\Phi(\phi=0)=0$, 
$$ 
0=\left(a_{0}+b_{0} \ln \rho\right)\left(A_{0}\right)+\sum_{\nu \neq 0}\left(a_{\nu} \rho^{\nu }+b_{\nu } \rho^{-\nu }\right)\left(A_{\nu }+B_{\nu }\right)
$$
要使上式对任意$\rho$均成立, 必然有$A_0=0,\ A_\nu =-B_\nu $.则解的形式可以简化为:
$$ 
\Phi(\rho, \phi)=\left(a_{0}+b_{0} \ln \rho\right)\left(B_{0} \phi\right)+\sum_{\nu  \neq 0}\left(a_{\nu } \rho^{\nu }+b_{\nu } \rho^{-\nu }\right) A_{\nu } \sin (\nu  \phi)
$$
代入边界条件$\Phi(\phi=\b)=0$, 
$$ 
0=\left(a_{0}+b_{0} \ln \rho\right)\left(B_{0} \beta\right)+\sum_{\nu  \neq 0}\left(a_{\nu } \rho^{\nu }+b_{\nu } \rho^{-\nu }\right) A_{\nu } \sin (\nu  \beta)
$$
要使上式对任意$\rho$均成立, 必然有$B_0=0,\ \sin(\nu \b)=0\ip \nu \b=n\pi$.则解的形式可以简化为:
$$ 
\Phi(\rho, \phi)=\sum_{n=1}^{\infty}\left(a_{n} \rho^{n \pi / \beta}+b_{n} \rho^{-n \pi / \beta}\right) A_{n} \sin \left(\frac{n \pi \phi}{\beta}\right)
$$
代入边界条件$\Phi(\rho=a)=0$, 
$$ 
0=\sum_{n=1}^{\infty}\left(a_{n} a^{n \pi / \beta}+b_{n} a^{-n \pi / \beta}\right) A_{n} \sin \left(\frac{n \pi \phi}{\beta}\right)
 $$
 要使上式对任意$\phi$均成立, 必然有$b_{n}=-a_{n} a^{2 n \pi / \beta}$.则解的形式可以简化为:
 $$ 
 \Phi(\rho, \phi)=\sum_{n=1}^{\infty} A_{n}\left(\left(\frac{\rho}{a}\right)^{n \pi / \beta}-\left(\frac{\rho}{a}\right)^{-n \pi / \beta}\right) \sin \left(\frac{n \pi \phi}{\beta}\right)
  $$
\sub{(b)}
$$ 
E_{\rho}=-\frac{\partial \Phi}{\partial \rho}=-\frac{\partial}{\partial \rho} \sum_{n=1}^{\infty} A_{n}\left(\left(\frac{\rho}{a}\right)^{n \pi / \beta}-\left(\frac{\rho}{a}\right)^{-n \pi / \beta}\right) \sin \left(\frac{n \pi \phi}{\beta}\right)
 $$
 $$ 
 \ip E_{\rho}=-\sum_{n=1}^{\infty} A_{n} \frac{n \pi}{a \beta}\left(\left(\frac{\rho}{a}\right)^{n \pi / \beta-1}+\left(\frac{\rho}{a}\right)^{-n \pi / \beta-1}\right) \sin \left(\frac{n \pi \Phi}{\beta}\right)
  $$
最低阶项为:
$$ 
E_{\rho}=-A_{1} \frac{\pi}{a \beta}\left(\left(\frac{\rho}{a}\right)^{\pi / \beta-1}+\left(\frac{\rho}{a}\right)^{-\pi / \beta-1}\right) \sin \left(\frac{\pi \phi}{\beta}\right)
 $$
同理对于$E_\phi$,
$$ 
E_{\phi}=-\sum_{n=1}^{\infty} A_{n} \frac{n \pi}{a \beta}\left(\left(\frac{\rho}{a}\right)^{n \pi / \beta-1}-\left(\frac{\rho}{a}\right)^{-n \pi / \beta-1}\right) \cos \left(\frac{n \pi \phi}{\beta}\right)
 $$
 最低阶项为:
 $$ 
E_{\phi}=-A_{1} \frac{\pi}{a \beta}\left(\left(\frac{\rho}{a}\right)^{\pi / \beta-1}-\left(\frac{\rho}{a}\right)^{-\pi / \beta-1}\right) \cos \left(\frac{\pi \phi}{\beta}\right)
$$
$$ 
\sigma(\rho, 0)=\left[\epsilon_{0} E_{\phi}\right]_{\Phi=0}
$$
$$ 
\sigma(\rho, 0)=-A_{1} \frac{\pi \epsilon_{0}}{a \beta}\left(\left(\frac{\rho}{a}\right)^{\pi / \beta-1}-\left(\frac{\rho}{a}\right)^{-\pi / \beta-1}\right)
$$
$$ 
\sigma(\rho, \beta)=\left[-\epsilon_{0} E_{\phi}\right]_{\phi=\beta}
    $$
    $$ 
\sigma(\rho, \beta)=-A_{1} \frac{\pi \epsilon_{0}}{a \beta}\left(\left(\frac{\rho}{a}\right)^{\pi / \beta-1}-\left(\frac{\rho}{a}\right)^{-\pi / \beta-1}\right)
$$
$$ 
\sigma(a, \phi)=\left[\epsilon_{0} E_{\rho}\right]_{\rho=a}
$$
$$ 
\sigma(a, \phi)=-A_{1} \frac{2 \pi \epsilon_{0}}{a \beta} \sin \left(\frac{\pi \phi}{\beta}\right)
$$
\sub{(c)}
由(b)中计算,
$$ 
\mathbf{E}=\sum_{n=1}^{\infty} A_{n} \frac{n \pi}{a \beta}\left[-\hat{\bm{\rho}}\left(\left(\frac{\rho}{a}\right)^{n \pi / \beta-1}+\left(\frac{\rho}{a}\right)^{-n \pi / \beta-1}\right) \sin \left(\frac{n \pi \phi}{\beta}\right)-\hat{\Phi}\left(\left(\frac{\rho}{a}\right)^{n \pi / \beta-1}-\left(\frac{\rho}{a}\right)^{-n \pi / \beta-1}\right) \cos \left(\frac{n \pi \phi}{\beta}\right)\right]
 $$
当$\b=\pi$, 
$$ 
\mathbf{E}=\sum_{n=1}^{\infty} A_{n} \frac{n}{a}\left[-\hat{\boldsymbol{\rho}}\left(\left(\frac{\rho}{a}\right)^{n-1}+\left(\frac{\rho}{a}\right)^{-n-1}\right) \sin (n \Phi)-\hat{\Phi}\left(\left(\frac{\rho}{a}\right)^{n-1}-\left(\frac{\rho}{a}\right)^{-n-1}\right) \cos (n \Phi)\right]
 $$
其最低阶项为:
$$ 
\mathbf{E}=A_{1} \frac{1}{a}\left[-\hat{\boldsymbol{\rho}}\left(1+\left(\frac{a}{\rho}\right)^{2}\right) \sin (\Phi)-\hat{\Phi}\left(1-\left(\frac{a}{\rho}\right)^{2}\right) \cos (\phi)\right]
 $$
当$\rho>>a$:
$$ 
\mathbf{E}=-A_{1} \frac{1}{a}[\hat{\boldsymbol{\rho}} \sin (\phi)+\hat{\mathbf{\phi}} \cos (\phi)] = -\frac{A_{1}}{a} \hat{\mathbf{j}}
 $$
半球面上的面电荷为:
$$ 
\sigma(a, \phi)=\sigma_{0} \sin (\phi),\  \sigma_{0}=-A_{1} \frac{2 \epsilon_{0}}{a}
  $$
在边缘处:
$$ 
\sigma(\rho, 0)=\sigma(\rho, \beta)=\frac{\sigma_{0}}{2}\left(1-\left(\frac{\rho}{a}\right)^{-2}\right)
 $$
半球面上的总电荷为:
$$ 
Q_{\text { half-cyl }}=-A_{1} \frac{2 \epsilon_{0}}{a} \int_{0}^{\pi} \sin (\phi) a \d \phi =-4A_1\ep_0
 $$
边缘处平均面电荷为:
$$ 
\sg_{\text{side}}=\ep_0E=-\f{\ep_0A_1}{a}
 $$
长度为$2a$的一段的总电荷为:
$$Q_{\text{side}}=2a\sg_{\text{side}}=-2\ep_0A_1$$
对比可以发现
$$2Q_{\text{side}} = Q_{\text { half-cyl }}$$
包含了半球的总电荷为:
$$ 
\begin{array}{l}
    {\dis Q_{1}=2 \int_{a}^{l}\left(-A_{1}\right) \frac{\epsilon_{0}}{a}\left(1-\left(\frac{\rho}{a}\right)^{-2}\right) d \rho+Q_{\text { half-cyl }}} \\
     {\dis Q_{1}=-2\left(A_{1}\right) \frac{\epsilon_{0}}{a}(l-a)-2\left(A_{1}\right) a \epsilon_{0}(1 / l-1 / a)+Q_{\text { half-cyl }}} \\
     {\dis Q_{1}=2 A_{1} \epsilon_{0}\left[\frac{-l}{a}-\frac{a}{l}\right]}
\end{array}
 $$
当$l>>a$:
$$ 
Q_{1}=\frac{-2 l \epsilon_{0} A_{1}}{a}
 $$
当没有半球面时, 总电荷为
$$ 
\begin{array}{l}{\dis Q_{2}=\sigma 2 l} \\ 
    {\dis Q_{2}=\left[\epsilon_{0} E_{y}\right]_{y=0} \sigma 2 l} \\ 
    {\dis Q_{2}=\left(-\epsilon_{0} A_{1} / a\right) 2 l} \\ 
    {\dis Q_{2}=\frac{-2 l \epsilon_{0} A_{1}}{a}}
\end{array}
 $$
二式相同. 










\def({\ifmmode \left\lparen \else\lparen\fi} \def){\ifmmode \right\rparen \else\rparen\fi}
\end{document}