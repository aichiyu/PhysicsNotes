\documentclass[UTF8,9pt]{ctexart}
\usepackage{../../template/homeworkTEMP/hw}
\setlength{\arraycolsep}{1pt}
\setcounter{secnumdepth}{0}
\title{The 2nd HW of Electrodynamics} 
\begin{document} 
\maketitle
\se{Q1}
1. Prove that $\nabla r=-\nabla'r=\f{\bm{r}}{r}$:\\
for x direction:
$$(\nabla r)_x=\pp{r}{x}=\ff{2r}2(x-x')=\f{\bm{r}_x}{r}$$
$$(\nabla' r)_x=\pp{r}{x'}=\ff{2r}2(x-x')(-1)=-\f{\bm{r}_x}{r}$$
the same for y and z direcion, therefore $\nabla r=-\nabla'r=\f{\bm{r}}{r}$\\
2. Prove that $\nabla \ff{r}=-\nabla'\ff{r}=-\f{\bm{r}}{r^3}$:\\
for x direction:
$$(\nabla \ff{r})_x=\pp{1/r}{x}=-\ff{2}\ff{r^3}2(x-x')=-\f{\bm{r}_x}{r^3}$$
$$(\nabla' \ff{r})_x=\pp{1/r}{x'}=-\ff{2}\ff{r^3}2(x-x')(-1)=\f{\bm{r}_x}{r^3}$$
the same for y and z direcion, therefore $\nabla \ff{r}=-\nabla'\ff{r}=-\f{\bm{r}}{r^3}$\\
3. Prove that $\nabla \times \f{\bm{r}}{r^3}=0$:\\
from the question above we know $\nabla \ff{r}=-\f{\bm{r}}{r^3}$, and $\nabla \times (\nabla A)=0$ for any scalar $A$:
$$\nabla \times \f{\bm{r}}{r^3}=-\nabla \times (\nabla \ff{r})=0$$
4. Prove that $\nabla \cdot \f{\bm{r}}{r^3}=-\nabla \cdot \f{\bm{r}}{r^3}=0$ for any $r\neq 0$:\\
for x direction:
$$(\nabla \cdot \f{\bm{r}}{r^3})_x=\pp{\bm{r}_x/r^3}{x}=r^{-3}-\f{3}{2}2\bm{r}_x^2\ff{r^5}=\ff{r^3}-3\f{\bm{r}_x^2}{r^5}$$
$$(\nabla' \cdot \f{\bm{r}}{r^3})_x=\pp{\bm{r}_x/r^3}{x'}=-r^{-3}+\f{3}{2}2\bm{r}_x^2\ff{r^5}=-\ff{r^3}+3\f{\bm{r}_x^2}{r^5}$$
so for all directin:
$$\nabla \cdot \f{\bm{r}}{r^3} = \f{3}{r^3}-\f{3}{r^5}(\bm{r}_x^2+\bm{r}_y^2+\bm{r}_z^2)=0$$
$$\nabla' \cdot \f{\bm{r}}{r^3} = -\f{3}{r^3}+\f{3}{r^5}(\bm{r}_x^2+\bm{r}_y^2+\bm{r}_z^2)=0$$
\se{Q2}
Show that the interaction between two fixed current loops obeys Newton’s third law:
$$F_{12}=\int_{C_1} I_1\d \bm{l}_1\times \bm{B}=\f{\mu_0I_1I_2}{4\pi}\iint_{C_1,C_2}\f{\d \bm{l}_1\times(\d\bm{l}_2\times \bm{r_{12}})}{r_{12}^3}$$

$$\f{\d \bm{l}_1\times(\d\bm{l}_2\times \bm{r_{12}})}{r_{12}^3}=\f{(\d\bm{l}_2\cdot\bm{r}_2)\d\bm{l}_1}{r_{12}^3}-\f{(\d\bm{l}_1\cdot\d\bm{l}_2)\bm{r}_{12}}{r_{12}^3}$$
$$\int_{C_2}\f{\d\bm{l}_2\cdot\bm{r}_2}{r_{12}^3}=\iint_{S_2}(\nabla\times \f{\bm{r}_2}{r_{12}^3})\d \bm{S}=0$$

$$\ip F_{12}=-\f{\mu_0I_1I_2}{4\pi}\iint_{C_1,C_2}\f{(\d\bm{l}_1\cdot\d\bm{l}_2)\bm{r}_{12}}{r_{12}^3}$$
And $\bm{r}_{12}=-\bm{r}_{21}$
$$F_{21}=\f{\mu_0I_1I_2}{4\pi}\iint_{C_1,C_2}\f{\d \bm{l}_2\times(\d\bm{l}_1\times \bm{r_{21}})}{r_{21}^3}=\f{\mu_0I_1I_2}{4\pi}\iint_{C_1,C_2}\f{(\d\bm{l}_1\cdot\d\bm{l}_2)\bm{r}_{12}}{r_{12}^3}=-F_{12}$$
\se{Q3}
Use the equation below to find the related equation for the conduction current $\bm{J}=n_fe\bm{v}$. Solve this equation for $\bm{E}(t)=\bm{E}_0\delta(t)$ if $\bm{J}(t<0)=0$ . What is $\bm{J}$ immediately after $t=0$ ? Connect this with the sum rule.
$$\dt{\bm{v}}=-\gamma\bm{v}+\f{e}{m}\bm{E}$$
%When $\dt{v}=0$, we will get $v=\f{eE}{m\gamma}$.\\
%$\ip J=n_fev=n_fe\f{eE}{m\gamma}=\f{n_fe^2}{m\gamma}E$\\
After $t=0, x=\int v\d t=0$ 
$$v=-\gamma\int v\d t+\f{e}{m}\int E\d t = \f{e}{m}E_0$$
$$J=\f{n_fe^2E_0}{m}$$
As $\mathcal{F}(\delta(t))=1$, $\bm{E}(t)=\bm{E}_0\delta(t)$ has all frequencies for the same strength. So that we can use it to determine $n_f$ experimentally using sum rule, 
\end{document}