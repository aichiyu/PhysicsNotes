\documentclass{revtex4}
\usepackage{mathrsfs}
\usepackage{amsthm}
\usepackage{amsmath}
\usepackage{amsfonts}
\usepackage{graphicx} 
\usepackage{color}
  
%%%%%%%%%%%%%%%%%%%%%%%%%%%%%%%%%%%%%%%
%begin{newcommands}
\newcommand\qqed{\rightline{$\square$}}                                 %\square
\newcommand\cm{\mathscr}                                               %dbar
\newcommand\dbar{\text{\dj}}                                            %dbar
\renewcommand\d{\mathrm{d}}                                            %d
\newcommand\sch{Schrödinger}                                           %Schrödinger
\newcommand{\dd}[3][]{\frac{\mathrm{d}^{#1} #2}{\mathrm{d} #3^{#1}}}   %d/d^n
\newcommand{\dt}[2][]{\frac{\mathrm{d}^{#1} #2}{\mathrm{d} t^{#1}}}    %d/dt^n
\newcommand{\pp}[3][]{\frac{\partial^{#1} #2}{\partial #3^{#1}}}       %∂/∂^n
\newcommand{\pt}[2][]{\frac{\partial^{#1} #2}{\partial t^{#1}}}        %∂/∂t^n 
\newcommand{\f}[2]{\frac{#1}{#2}}                                      %A/B
\newcommand{\ff}[1]{\frac{1}{#1}}                                      %1/A
\newcommand\rank{\mathrm{rank}}                                        %rank
\newcommand\tr{\mathrm{tr}}                                            %tr
\newcommand\e[1]{\times 10^{#1}}                                       %×10^
\renewcommand\Re{\mathrm{Re}\ }                                          %Re
\renewcommand\Im{\mathrm{Im}\ }                                          %Im
\renewcommand\ln{\mathrm{ln}\ }                                          %ln
\newcommand\Ln{\mathrm{Ln}\ }                                            %Ln
\renewcommand\arg{\mathrm{arg}}                                        %arg
\newcommand\ip{\implies}                                               %→
\newcommand\tm{\times}                                                 %×
\renewcommand\l{\lambda}                                               %λ
\renewcommand\a{\alpha}                                                %α
\renewcommand\k{\kappa}                                                %κ
\renewcommand\o{\omega}                                                %ω
\newcommand\D{\Delta}                                                  %Δ
\newcommand\de{\delta}                                                 %δ
\renewcommand\t{\theta}                                                %θ
\renewcommand\epsilon{\varepsilon}                                     %ε
\newcommand\abs[1]{\left| #1 \right|}                                  %|A|
\renewcommand\exp{\mathrm{exp}\ }                                        %exp
\newcommand\se{\section}                                               %section
\newcommand\sub{\subsection}                                           %subsection
\newcommand\sumi{\sum_{i=1}^n}                                         %∑ 
\AtBeginDocument{\renewcommand{\bar}{\overline}}                       %bar
\AtBeginDocument{\renewcommand{\hat}{\widehat}}                        %hat
\newcommand\inti{\int_{0}^{infty}}                                     %∮
\renewcommand{\arraystretch}{1.2}                                      %change array stretch
\newcommand{\dis}{\displaystyle}                                       %big equation
\newcommand{\ar}[2][rl]{\begin{array}{#1}                              %begin an array
        #2
    \end{array}}                                                 
\newcommand\bkt[1]{\left< #1 \right>}                                  %<x>
\newcommand\putfig[2]{                                                 %put a figure
    \begin{center}    
        \includegraphics[scale=#1]{#2}
    \end{center}}
%end{newcommands}

%resize the parentheses
\def\lparen{(} 
\def\rparen{)} 
\catcode`(=\active 
\catcode`)=\active
\def({\ifmmode \left\lparen \else\lparen\fi} 
\def){\ifmmode \right\rparen \else\rparen\fi}
%%%%%%%%%%%%%%%%%%%%%%%%%%%%%%%%%%%%%%%
\begin{document}

\pagestyle{empty}

\raisebox{40pt}[0pt][0pt]{\vbox to 0pt{\includegraphics{ShanghaiTech-Logo.png}}}

\raisebox{35pt}[0pt][0pt]{\hbox to \textwidth{\hfil\bfseries \LARGE\color[rgb]{1,0,0} Group Theory\hfil}}

\raisebox{24pt}[0pt][0pt]{\hbox to \textwidth{\hfil\bfseries \LARGE\color[rgb]{0,0,0.9}  Homework Assignment 02\hfil}}

\raisebox{13pt}[0pt][0pt]{\hbox to \textwidth{\hfil\bfseries \Large\color[rgb]{0,0.5,0} Spring, 2019\hfil}}

\noindent\raisebox{13pt}[0pt][0pt]{\rule{\textwidth}{1pt}}

\vspace{-6pt}

\begin{enumerate}
%\item Find the multiplication table for the group $D_{4}$. For the evaluation of the products of the elements of $D_{4}$, provide as many details as possible.

\item Show that the intersection $S$ of two invariant subgroups $S_{1}$ and $S_{2}$ of a group $G$ is an invariant subgroup.

$S=S_1\cap S_2$, for $T \in S$, $X \in G$, $S \in S_1,S_2$, thus $XTX^{-1} \in S_1,S_2 \ip XTX^{-1} \in S$.
\item The multiplication table of a finite group $G$ is given by

\begin{center}
\begin{tabular}{c|cccccccccccc}
        & $E$ & $A$ & $B$ & $C$ & $D$ & $F$ & $I$ & $J$ & $K$ & $L$ & $M$ & $N$\\
\hline
$E$ &  $E$ & $A$ & $B$ & $C$ & $D$ & $F$ & $I$ & $J$ & $K$ & $L$ & $M$ & $N$\\
$A$  & $A$  & $E$  & $F$  & $I$  & $J$  & $B$  & $C$  & $D$  & $M$  & $N$  & $K$  & $L$\\
$B$  & $B$  & $F$  & $A$  & $K$  & $L$  & $E$  & $M$  & $N$  & $I$  & $J$  & $C$  & $D$\\
$C$  & $C$  & $I$  & $L$  & $A$  & $K$  & $N$  & $E$  & $M$  & $J$  & $F$  & $D$  & $B$\\
$D$  & $D$  & $J$  & $K$  & $L$  & $A$  & $M$  & $N$  & $E$  & $F$  & $I$  & $B$  & $C$\\
$F$  & $F$  & $B$  & $E$  & $M$  & $N$  & $A$  & $K$  & $L$  & $C$  & $D$  & $I$  & $J$\\
$I$  & $I$  & $C$  & $N$  & $E$  & $M$  & $L$  & $A$  & $K$  & $D$  & $B$  & $J$  & $F$\\
$J$  & $J$  & $D$  & $M$  & $N$  & $E$  & $K$  & $L$  & $A$  & $B$  & $C$  & $F$  & $I$\\
$K$  & $K$  & $M$  & $J$  & $F$  & $I$  & $D$  & $B$  & $C$  & $N$  & $E$  & $L$  & $A$\\
$L$  & $L$  & $N$  & $I$  & $J$  & $F$  & $C$  & $D$  & $B$  & $E$  & $M$  & $A$  & $K$\\
$M$  & $M$  & $K$  & $D$  & $B$  & $C$  & $J$  & $F$  & $I$  & $L$  & $A$  & $N$  & $E$\\
$N$  & $N$  & $L$  & $C$  & $D$  & $B$  & $I$  & $J$  & $F$  & $A$  & $K$  & $E$  & $M$
\end{tabular}
\end{center}

\begin{enumerate}
\item Find the inverse of each element of $G$.
$$\ar{
    E^{-1}=&E\\
    A^{-1}=&A\\
    B^{-1}=&F\\
    C^{-1}=&I\\
    D^{-1}=&J\\
    F^{-1}=&B\\
    I^{-1}=&C\\
    J^{-1}=&D\\
    K^{-1}=&L\\
    L^{-1}=&K\\
    M^{-1}=&N\\
    N^{-1}=&M
}$$
\item Find the elements in each class of $G$.
$$\{E\}, \  \{A\},\  \{B,C,D\}, \  \{F,I,J\}, \  \{K,L,M\}, \  \{N\}.$$
\item Find all invariant subgroups of $G$.
$$\{E\},\  \{E,A\},\ \{E,A,K,L,M,N\}\,\ \{E,A,B,C,D,F,I,J,K,L,M,N\}$$
\end{enumerate}

\item Consider the group $D_{3}$.
\begin{enumerate}
\item List all the classes of $D_{3}$. 
$$\{E\},\ \{D,F\},\ \{A,B,C\}$$
\item Find the right and left cosets of the subgroup $S=\{E, A\}$ of $D_{3}$.
\end{enumerate}
Right cosets: 
$$\ar{
    SE=SA=&\{E,A\}\\
    SD=SC=&\{D,C\}\\
    SF=SB=&\{F,B\}
}$$
Left cosets:
$$\ar{
    ES=AS=&\{E,A\}\\
    DS=BS=&\{D,B\}\\
    FS=CS=&\{F,C\}
}$$
\item For the group $D_{3}$ and its invariant subgroup $S=\{E, D, F\}$, find the factor group $D_{3}/S$. Construct the multiplication table for the factor group.\\
Right cosets: 
$$\ar{
    SE=SD=SF=S\\
    SA=SB=SC=\{A,B,C\}
}$$
Multiplication table:
$$\ar[c|cc]{
     &SE&SA\\
    \hline
    SE&SE&SA\\
    SA&SA&SE
}$$
\item Consider $C_{6} = \{E, \,  a, \,  a^{2}, \,  a^{3}, \,  a^{4}, \,  a^{5}\}$ and its two subgroups $S_{1}=\{E, \,  a^{3}\}$ and $S_{2}=\{E, \,  a^{2}, \, a^{4}\}$. Show that $C_{6} = S_{1}\otimes S_{2}$.
$$\ar{
    EE=&E\\
    Ea^2=&a^2\\
    Ea^4=&a^4\\
    a^3E=&a^3\\
    a^3a^2=&a^5\\
    a^3a^4=&a
}$$
$$\ip C_{6} = S_{1}\otimes S_{2}.$$
\end{enumerate}
\end{document}
