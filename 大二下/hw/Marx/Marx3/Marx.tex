\documentclass[UTF8,9pt]{ctexart}
\usepackage{../../template/homeworkTEMP/hw}
\title{马克思主义基本原理概论思考题} 
\begin{document} 
\maketitle
\se{社会主义社会是否存在剩余价值?}
剩余价值的生产并非仅存在于资本主义社会, 社会主义社会也广泛存在, 劳动者创造的剩余价值是企业利润和国家财政收入的源泉. 要说明这一点, 首先要理解剩余价值理论的准确含义, 「剩余价值」是相对于「自用价值」而言的, 认为凡是超过了劳动力自身生存需要的生活资料价值、维持劳动力家属子女所需要的生活资料价值、劳动力教育训练费用的, 就是劳动者的剩余价值. 社会主义社会存在剩余价值, 但社会主义剩余价值在性质和构成上与资本主义剩余价值是不同的. 资本主义剩余价值体现的是资本主义的基本经济关系和生产分配方式, 而社会主义剩余价值体现的是社会主义的生产分配关系. 

许多观点认为剩余价值仅在资本主义社会存在, 而在社会主义社会并不存在. 但事实并非如此, 第一, 社会主义社会仍然存在雇人做工的私有企业、合资企业、股份制企业等传统观点所指称的「资本主义经济」, 它们就属于生产剩余价值. 第二, 即使是公有制经济, 同样存在生产剩余价值. 因为公有制企业的职工并没有为满足自己的需要而将自己创造的价值全部占有, 首先企业每赚取一部分利润都需要缴税, 税务形成国家财政收入,主要用于经济建设事业、科教文卫事业、社会福利事业、国防和行政管理费用. 第二每年企业的红利并不会全部作为现金发给员工, 而是拿出一部分为企业继续发展用于扩大再生产. 而这实际就是公有制企业职工创造的剩余价值. 

总之, 社会主义市场经济中的剩余价值,不仅是企业生存和发展的基础,也是国家加强宏观调控的物质基础,国家的税金、社会保障金以及国家的财政储备,都来源于剩余价值. 没有剩余价值的增加,就没有国家财富的增加,就没有国家财政收入的增加,大规模的经济建设和社会公益活动就无法进行,社会主义的生产目的就不可能实现. 
\end{document}