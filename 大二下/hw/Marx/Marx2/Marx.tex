\documentclass[UTF8,9pt]{ctexart}
\usepackage{../../template/homeworkTEMP/hw}
\title{马克思主义基本原理概论思考题} 
\begin{document} 
\maketitle
\se{如何规避科学技术的负面性?}
自21世纪科技革命以来, 科学技术在各个领域都不断出现突破性进展, 这之中又尤其以半导体行业为主. 人类正处于以往任何时代都难以想象的科技洪流中. 日新月异的信息流无疑可以帮助人们以空前的速度学习知识, 同时也给人们带来了一系列高科技的副产物. 但同时也带来了一系列的负面效应. 对于已经深入渗透到人类社会各个角落的科学技术, 我们不可能, 也不应该为了避免这些负面效应去规避它, 而是充分了解它, 去芜存菁. 
    
科学技术在它给人类社会带来了全面改变, 这些改变一些是好的, 一些是不好的, 但绝大多数都是同时具有好与不好两种属性的「双刃剑」, 例如人们常说的, 社交软件帮助人们在网络上交流, 但却让两个人面对面的交流减少了, 搜索引擎给了人们强有力的搜寻信息的手段, 但是网络上良莠不齐, 真假混杂的信息却会让不善分辨的人受害, 高楼大厦带来了更大的人口容纳量, 但却让现代城市人永远失去了像以前那样邻里老少互相串门的机会与体验. 那么这些负面效应既然存在, 我们又要怎么规避这些负面效应呢? 
    
这些负面效应大致可以归结为两个来源: 内在发生和外界影响. 公民受科学技术影响而产生的负面效应即使内在发生类的. 例如人与人之间的交流变少, 隔阂增大, 从「人情社会」进入「冷漠社会」, 这样的影响必须通过文化熏陶感染人们内心来改变, 而不能通过外在的强制性手段加以约束. 落实到行动上, 即是对社会「价值观」的建设. 价值观建设又可以从主动和被动两方面实施, 主动的建设是预防性的, 例如社会主义核心价值观, 我们不是说社会缺乏这些价值观, 而是起引导作用, 鼓励人们拥有这些价值观. 而被动的建设是当社会出现某种不良的道德滑坡时, 及时引导制止. 要让社会价值观朝着希望的方向变化, 两种方式缺一不可. 

另一种负面效应的来源是外在影响, 即是由于某种或某些科学技术的发展, 公民被迫地受到某种负面影响. 例如空气污染, 雾霾. 人们不是主观选择了雾霾, 而是被动地成为了受害者. 经济学上把这种现象称为事件的「外部性」. 对于存在外部性的事件, 不能只依靠人们自己调节, 而需要政府这样具有强制手段的机构加以限制. 例如对重污染的企业收税, 建立排放标准并严格监督审查. 

通过一内一外两个角度对科学技术加以限制, 其负面性也会明显降低. 而如何通过尽量少的强制手段减少负面性, 如何在降低负面性的同时又维持其正面性以保持社会高速发展是一个复杂的问题, 而让这个问题达到最优解就需要一个高效的政府了. 
\end{document}