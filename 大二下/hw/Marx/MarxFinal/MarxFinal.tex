\documentclass[UTF8,9pt]{ctexart}
\usepackage{../../template/homeworkTEMP/hw}
\usepackage[colorlinks,linkcolor=red]{hyperref}
\usepackage{fancyhdr}
\pagestyle{fancy}
\lhead{上海科技大学}
\rhead{马克思主义基本原理概论读书报告}
\title{%
资本主义经济关系的本质:剩余价值理论 \\
  \large ——《资本论》第一卷, 第十四章《绝对剩余价值和相对剩余价值》\cite{title}读书报告}
\begin{document} 
\maketitle
\begin{abstract}
    剩余价值理论是《资本论》的核心概念, 也是马克思主义政治经济学的核心概念. 本文简要介绍了剩余价值理论及其原理和表现, 从剩余价值的绝对和相对两方面将其拆解分析, 并解释了马克思``相对剩余价值是绝对的, 绝对剩余价值是相对的''的含义与其所揭示的现象. 最后从剩余价值的分配角度出发, 分析剩余价值分配去向问题, 得到了``要根除剥削就必须改变现有的生产模式''这一结论. 
\end{abstract}
\textbf{关键字:}剩余价值, 《资本论》, 马克思
\newpage%另起一页
\tableofcontents%命令输出论文目录
\newpage%另起一页
\se{剩余价值理论}
马克思的《资本论》共有四卷, 前三卷均为理论部分, 分别研究``资本的生产过程'',``资本的流通过程'',``资本主义生产的总过程'', 而第四卷, 也就是最后一卷就从剩余价值理论出发, 分析整个资本主义的发展历史进程, 在对剩余价值的讨论中的一个关键出发点便是资本的生产过程实质是资本家剥削雇佣工人的剩余价值, 所以资本的生产过程的核心是剩余价值生产. 

在《资本论》中, 马克思认为商品流通是资本主义经济的基础, 而商品流通具有两个基础的表现形式——为要买而卖, 以及为要卖而买. 马克思提出, 这两种流通形态的主要不同在于, 前者是为了换取商品而出卖已有的等价商品, 后者则是为了得到增额的货币而进行商品买卖\cite{1}. 在前一种的流通过程中, 虽然前后商品的使用价值会有不同, 但是价值量是基本相等的, 即遵循等价交换的原则, 在这个过程中货币在其中只是充当了一个媒介. 而在后一种过程则大不相同, 在一轮交换过程结束后, 商品买卖者可以从中取回一笔增加的货币. 

例如, 用$w_1$元购买棉花, 再将此棉花贩卖出去而得到$w_2$元, 最终所得的货币比开始时增加了$\D w=w_2-w_1$元. 此时$\D w$被就称作剩余价值, 即流通过程中的价值增值额. 

在劳工参与的情况下, 根据劳动价值论``价值是凝聚在商品中的无差别的人类劳动''的观点, 在生产过程中, 原材料的价值被转移到了新的产品中, 因此也产生了新的价值, 于是生产过程隐盖了工人创造剩余价值的过程. 

通过剩余价值理论, 我们很容易可以看出, 劳动者所得的收入总是比他们所生产出来的产品的价值要少(否则剩余价值小于零, 而资本家不会做剩余价值小于零的事). 也就是说, 在资本主义社会, 社会的购买力是低于社会所生产的物质财富(``供过于求''), 因此, 这也是一种可以解释为什么资本主义社会经济危机不可避免, 会周期性发生的方法. 
\se{绝对剩余价值和相对剩余价值的辩证关系}
马克思基于剩余价值理论, 将剩余价值又细分为两类——绝对剩余价值、相对剩余价值. 如果要对两种剩余价值做一个简洁的区分, 那么可以认为, 通过延长工作日而生产的剩余价值, 叫做绝对剩余价值. 而通过缩短必要劳动时间、相应地改变工作日的两个组成部分——必要劳动时间、剩余劳动时间的量的比例而产生的剩余价值, 叫做相对剩余价值. 和绝对剩余价值不同, 它的增加不是由于创造出来的价值绝对地增加了, 而是通过缩短必要劳动时间, 把一部分原来的必要劳动时间转化为剩余劳动时间, 从而使剩余价值相对地增加. 它们是资本家提高剥削程度的两种基本方法之一. 每个资本家提高劳动生产率的直接目的是追逐超额剩余价值, 社会劳动生产率的提高是个别资本家为了这种目的而竞争的必然结果. 

一般而言, 对于绝对剩余价值和相对剩余价值的讨论和分析, 是在考虑资本家对于提高工人压迫的时候的两种方式, 因此剩余价值无论绝对相对, 是要与提高压迫前的剩余价值作比较才能得出的. 

\sub{绝对剩余价值的生产构成资本主义体系的一般基础}
绝对剩余价值生产和相对剩余价值生产作为提高剥削程度的两种基本方法, 二者既有联系又有区别. 

第一, 两种剩余价值生产的手段虽然有所不同, 但其结果和本质是一致的, 即它们都延长了工人的剩余劳动时间, 增加了剩余价值量, 提高了剩余价值率. 

第二, 绝对剩余价值生产是资本主义生产的一般基础, 也是相对剩余价值生产的起点. 因为任何资本主义生产, 都必须把工作回延长到必要劳动时间以上, 否则就不能产生剩余价值. 同时, 只有工作日分割为必要劳动时间和剩余劳动时间, 才能以此为基础, 进一步缩短必要劳动时间, 生产相对剩余价值. 

第三, 生产剩余价值的两种基本方法的物质基础不同, 在资本主义发展的各个历史阶段起着不同的作用. 在资本主义的初期, 由于生产技术没有重大变化, 绝对剩余价值生产是资本家增加剩余价值的主要方法. 随着资本主义的发展, 科学技术在生产中的广泛应用, 以及生产力不断发展, 相对剩余价值生产的作用也就日益突出了. 
\sub{剩余价值的绝对性与相对性}
对于不同时期的相对剩余价值和绝对剩余价值可做如下理解:
\rk{
\item 在早期阶段, 总的价值为$w_1 = w_{\text{劳动力价值}}+w_{\text{剩余价值}}$.
\item 而当资本家提高了绝对剩余价值后, 总的价值为$w_2 = w_{\text{劳动力价值}}+w_{\text{剩余价值}}+w_{\text{绝对剩余价值}}=w_1+w_{\text{绝对剩余价值}}$. 这通常发生在资本主义生产初期, 由于生产能力低下, 绝对剩余价值生产是资本家加强对工人剥削的主要方法. 容易看到, 这时提高的剩余价值是在提高总剩余价值, 是有利于社会总福利提高的. 
\item 当资本家提高了相对剩余价值后, 总的价值为$w_3 = w'_{\text{劳动力价值}}+w_{\text{相对剩余价值}}+w_{\text{剩余价值}}+w_{\text{绝对剩余价值}}$, 其中$w'_{\text{劳动力价值}}+w_{\text{相对剩余价值}} = w_{\text{劳动力价值}}$, 也即$w_2=w_3$, 总的价值并没有提高, 随着资本主义的发展、科学技术的进步和劳动生产率的提高, 工人为了缩短工作日的斗争加强, 于是相对剩余价值法逐渐成为了主要的剥削方法. 它是通过在保证工作日长度不变的情况下, 通过科技进步, 减少劳动者的必要劳动时间, 从而使剩余劳动时间相对增加.
}

在《绝对剩余价值和相对剩余价值》中, 马克思写道:\\
``从一定观点看来, 绝对剩余价值和相对剩余价值之间的区别似乎完全是幻想的. 相对剩余价值是绝对的, 因为它以工作日的绝对延长超过工人本身生存所必需的劳动时间以上为前提. 绝对剩余价值是相对的, 因为它以劳动生产率发展到能够把必要劳动时间限制为工作日的一个部分为前提. 但是, 如果注意一下剩余价值的运动, 这种表面上的同一性就消失了\cite{title}. ''\\
这提到了剩余价值的绝对性与相对性的关系——相对剩余价值是绝对的, 而绝对剩余价值是相对的. 这个看似矛盾而难以理解的句子其实深刻地说明了两种剩余价值的产生机理. 相对剩余价值的产生来自于前式中$w_{\text{劳动力价值}}$的降低, 而资本家可以通过调节工作时间达到这一目的, 因此是绝对可行的. 但绝对剩余价值来自于生产力的提高, 提高绝对剩余价值必须提高劳动生产率, 而这依赖于科学技术的进步, 不是靠人为因素就能做到的. 因此是相对的. 当然, 在科技进步时同样可以降低工作时间来提高相对剩余价值, 但这并不矛盾, 反而更加印证了相对剩余价值绝对可行这一事实. 
\sub{绝对剩余价值生产和相对剩余价值生产的联系与区别}
绝对剩余价值生产和相对剩余价值生产是资本家提高生产收益利润的两个主要方法. 而这两种生产模式既相联系又相区别:
\rk{
    \item 两者的本质和结果是一致的, 它们都延长了工人的剩余劳动时间, 增加了剩余价值量, 提高了剥削程度. 
    \item 绝对剩余价值的生产只同工作日的长度有关; 相对剩余价值的生产使劳动的技术过程和社会组织发生根本的革命. 
    \item 绝对剩余价值的生产以劳动对资本的形式隶属为前提, 而相对剩余价值的生产以劳动对资本的实际隶属为前提. 
    \item 两者的物质基础不同, 在资本主义发展的各个历史阶段起着不同的作用. 在资本主义初期, 生产技术水平低, 绝对剩余价值生产是增加剩余价值的主要方法. 随着资本主义的发展, 生产技术水平不断提高, 相对剩余价值生产日益成为增加剩余价值的主要方法. 
}

\se{对剩余价值分配方法的思考}
马克思指出, 资本主义剩余价值是由雇佣工人的剩余劳动创造的、被资本家无偿占有的价值. 资本主义生产最终目的是尽可能多地榨取劳动者的剩余价值. 资本主义发展过程中, 形成了三种资本:产业资本、商业资本、借贷资本, 资本家也分成三个集团:产业资本家、商业资本家和借贷资本家. 除此以外, 资本主义社会还有存在着大土地所有者, 这几个剥削集团都是以剩余价值作为生存的基础的. 剩余价值分配为产业资本家取得产业利润, 商业资本家取得商业利润, 借贷资本家取得利息\cite{2}. 那么这里就涉及到了剩余价值的分配问题------既然现在资本家无偿占有了劳动者生产的价值, 那这种分配必然是不合理的, 但对于如何分配, 却有又多种不同的看法.
\sub{剩余价值均归劳动者所有}
    这是一种较初级的看法, 这种分配看似最大化了劳动者的利益, 但历史告诉我们这种制度无法稳定存在, 就如之前所说, 资本主义生产最终目的是尽可能多地榨取劳动者的剩余价值. 当资本得不到剩余价值, 便无法正常运行, 不能履行提供生产资料的职责, 这时候就需要政府无偿提供生产资料, 但政府的运行离不开税收, 而对劳动者收税同样是变相地再分配剩余价值. 
\sub{剩余价值在劳动者与资本间按比例分配}
    既然剩余价值不能完全归劳动者所有, 那么将其同时分配给资本和劳动者便是一个自然的想法了, 但这其实隐含着一个矛盾. 如果承认生产资料归资本家占有是合理的, 那么剩余价值就是生产资料在生产中产生的产物, 资本理所应当应该取得剩余价值, 如果不承认其合理性, 那剩余价值就是劳动者劳动的产物, 剩余价值就理所应当应该归劳动者所有. 于是生产资料的分配问题变为关于公共资料所有权和劳动者人权的问题, 但无论如何, 答案只能是承认其合理性或不承认, 不能``部分承认'', 因此就不存在一个按比例分配的问题了. 
\sub{根除``为剩余价值而生产''的生产模式}
    马克思之所以提出剩余价值, 就是为了揭示剥削. 资本家本能的追求利润最大化, 就会想方设法的扩大剩余价值, 绝对不会存在什么合理的分配. 在资本主义中, 虽然单个企业或者商品的剩余价值不等于利润, 但系统整体总剩余价值=总利润, 低于平均利润率的金钱积累速度, 资本就会缩水或者破产, 这必然引资本之间的对抗. 要解决这个问题, 就必须改变为利润生产的模式, 修改为为需求而生产, 按需分配, 这就是共产主义了. 


因此, 其实剩余价值理论并非为了重新分配剩余价值, 而是为了让人们知道``剩余价值''的存在, 明白资本家剥削性的本质. 与资本家商量如何分配剩余价值无异于与虎谋皮, 要根除剥削就必须改变现有的生产模式, 而这就与整本《资本论》的主题------“通过分析资本主义的发展过程, 找出现代社会的运动规律”相契合了. 
\begin{thebibliography}{99}  
    \bibitem{title}马克思. 第十四章 绝对剩余价值和相对剩余价值. Retrieved May 23, 2019, from \url{https://www.marxists.org/chinese/marx/capital/14.htm}
    \bibitem{1}剩余价值. (2019, January 22). Retrieved from 维基百科, 自由的百科全书:\url{https://zh.wikipedia.org/w/index.php?title=%E5%89%A9%E4%BD%99%E4%BB%B7%E5%80%BC&oldid=52888871}
    \bibitem{2}剩余价值. Retrieved from 百度百科:\url{https://baike.baidu.com/item/%E5%89%A9%E4%BD%99%E4%BB%B7%E5%80%BC#3_2}

\end{thebibliography}
\end{document}