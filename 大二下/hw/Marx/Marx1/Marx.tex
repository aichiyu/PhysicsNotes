\documentclass[UTF8,9pt]{ctexart}
\usepackage{../../template/homeworkTEMP/hw}
\title{马克思主义基本原理概论第一次思考题.} 
\begin{document} 
\maketitle
\se{如何运用实践分析法去解释翟天临案例?}
2019年2月,因翟曾在直播中发表言论“知网是什么东西”,被网友发现知网上无法查到其博士学位论文,并质疑其学历可能掺水或造假。不久,翟天临的高考分数也遭到网友质疑。2019年2月14日,翟天临就学术风波发表致歉声明。---维基百科

不久之前, 演员翟天临学术造假的消息在网络上掀起轩然大波. 但很多人不解的是, 翟天临在娱乐圈已经风生水起, 为何还要冒着巨大的风险通过不当手段获得博士学位? 要谈这个问题, 必须联系中国娱乐圈的现状, 当今中国的娱乐圈, 常被人指责是``小鲜肉时代'', 许多人不在意演员的演技如何, 却更加关心演员的``颜值'', ``人设'', 因此, 一个``学霸''的人设比努力提高演技更有吸引力, 更能帮助翟天临提高人气, 因此翟选择铤而走险便不是那么难以理解了. 

但他或许并没有想到事件曝光后, 后果会如此严重. 翟的微博下许多粉丝都评论``谁在学生时代还没做过弊呢?''他和他的一些粉丝作为完全不了解学术圈的人, 可能完全没有想到``学术造假''和``考试作弊''有相当大的差别. 没有正确估计自己行为的后果---这想必也是翟选择铤而走险的原因之一.
\se{如何运用矛盾分析法去解释中美贸易战?}
中国美国都希望积极发展经济, 而两国却展开了妨碍经济发展的贸易战, 这场不见硝烟的战争其实基于一个难以调和的矛盾: 美国对华贸易逆差. 特朗普一直强调, 贸易顺差/逆差是衡量一国在国际金融胜败的关键因素, 而中国长期对美有贸易顺差. 特朗普在竞选总统时, 曾承诺要降低贸易逆差, 特朗普在贸易战中提高关税就是为了减少进口, 减少逆差.  但事与愿违, 提高关税后逆差不降反升, 这又使中美两国经济矛盾更加剧烈, 矛盾一日不得到缓解, 贸易战便一日不会结束, 这正是贸易战一直持续到如今的原因. 

跳出经济矛盾本身, 贸易战或许是美国要维持经济霸权和中国要经济崛起二者矛盾的体现. 贸易战某种程度上是美国对中国经济崛起的一种防卫和控制手段. 对美国来说, 自己世界经济霸主地位绝不能动摇, 面对中国崛起并成为世界第二大经济体, 美国逐渐将以前用于日本的手段用在中国身上也是可以预料到的. 贸易战的战争对象不是中国, 而是中国背后世界第二大经济体的位置. 
\se{如何运用社会关系分析法去解释特朗普执政?}
特朗普和希拉里两人分别代表了两种不同的立场. 特朗普作为一个商人, 他的的执政方针就是发展实体经济, 重振美国梦. 而希拉里的执政方针都更倾向给精英们提供福利. 民众们看到如果让特朗普执政, 他会给予人一个通过努力就能获得回报的相对公平的环境, 这种环境对于广大非精英人群都是极具吸引力的, 因为意味着他们只要努力就有机会上升一个阶层, 而这种机会在阶级固化比中国严重得多的美国是实在难得的. 民主党让美联储和华尔街这些高收入人群得到了最多的利益, 这让硅谷的科技公司CEO一边倒地支持希拉里, 并在社交软件和新闻媒体上投放大量宣传广告. 但希拉里的政策对普通百姓并不友好. 因此作为一个理性的普通人, 选择特朗普更有利于他个人发展. 从投票结果也可以看到, 精英大多把选票给了希拉里, 而人数多得多的普通人选择把选票投给特朗普. 人们常说在民主的环境下, 普通人的话语权会达到前所未有的高度, 民众的力量将空前强大. 特朗普最终当选总统正是这句话的体现. 


\end{document}