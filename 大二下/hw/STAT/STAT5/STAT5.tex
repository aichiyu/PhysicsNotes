\documentclass[UTF8,9pt]{ctexart}
\usepackage{../../template/homeworkTEMP/hw}
\setlength{\arraycolsep}{1pt}
\setcounter{secnumdepth}{0}
\title{统计力学第五次作业} 
\begin{document} 
\maketitle
\se{4.2}
由于$\mu$是强度量, 各组员$T,p$不变且比例不变时, 与$n$无关, 即对任意$\l$, 有
$$\mu_{T,p}(\l n_i) = \mu_{T,p}(n_i)$$
根据Euler定理, 满足:
$$\sum n_i\pp{\mu}{n_i}=0$$
上式对任意$T,p$均成立.
\se{4.3}
(a) 混合前: $G_1=n_1g_1+n_2g_2$. 混合后$G_2=n_1\mu_1+n_2\mu_2$.
$$G_2-G_1 = n_1RT\ln x_1+n_2RT\ln x_2$$

(b) $pV=RT(n_1+n_2)$, 等温等压下$\D V=0$.

(c) $\D S = -\D (\pp{G}{T})_{p,n_i}=-n_1R\ln x_1-n_2R\ln x_2$.

(d) $H (T,p)$是$T,p$的函数, 等温等压下$H$不变.

(e) $\D U =T\D S - P\D V=0$.
\se{4.6}
化学势满足式
$$\mu=\sum RT[\vp_i+\ln(x_ip)]$$
则水的化学势变化为: 
$$\D \mu = RT[-\ln(p)+\ln((1-x)p)] = RT\ln(1-x)$$
又由于$\D\mu = g_1(T,p)-g_1(T,p_0)$, 
$$ g_1(T,p_0)+RT\ln(1-x)=g_1(T,p)$$
\se{4.8}
(1) 
$$p=\f{n}{V}=\f{n_1+n_2}{V_1+V_2}RT$$

(2)\\
由于
$$S=\sum n_i\left[\int\f{c_{pi}}{T}\d T-R\ln(x_ip)+s_{i0}\right]$$
$$\D S= -n_1R\ln\f{n_1}{n_1+n_2}p-n_2R\ln\f{n_1}{n_1+n_2}p+n_1R\ln p_1+n_2R\ln p_2$$
代入(1)中$p$, 
$$\ip \D S= -n_1R\ln\f{n_1RT}{V_1+V_2}-n_2R\ln\f{n_1RT}{V_1+V_2}+n_1R\ln \f{n_1RT}{V_1}+n_2R\ln \f{n_2RT}{V_2}$$
$$\D S = n_1R\ln\f{V_1+V_2}{V_1}+n_2R\ln\f{V_1+V_2}{V_2}$$

(3)\\
利用式$S=nC_{p,m}\ln T-nR\ln p+S_0$,
$$\D S = (n_1+n_2)C\ln T-(n_1+n_2)R\ln\f{V_1+V_2}{n_1+n_2}-n_1C\ln T+n_1R\ln V_1/n_1-n_2C\ln T+n_2R\ln V_2/n_2$$
$$=-(n_1+n_2)R\ln\f{V_1+V_2}{n_1+n_2}+n_1R\ln V_1/n_1+n_2R\ln V_2/n_2$$
\se{4.9}
$$K_1 = \f{c_{NH_3}}{c_{N_2}^{1/2}c_{H_2}^{3/2}},\quad K_2 = \f{c^2_{NH_3}}{c_{N_2}c_{H_2}^{3}}=K_1^2$$
$$\ip K_2=K_1^2 = \f{27}{16}\tm\f{\ep^4}{(1-\ep^2)^2}p^2$$
\se{4.11}
表面张力系数为$\sigma=(\frac{\partial G}{\partial A})_{T, p}$. 热力学第三定律指出$T\rightarrow0$时, $\pp{\D G}{T}=-(\D S)_T = 0$.\\
则$$\pp{}{A}\pp{\D G}{T}=0=\dd{}{T}\pp{\D G}{A} = \dd{}{T}\sg$$
即$\dd{\sg}{T} = 0$.
\se{4.13}
设熵分别为
$$S=S_g(0)+\int_0^{T_0}\f{c_g}{T}+L/T_0\ip \D S = 51.80\text{J/mol}$$
$$S=S_w(0)+\int_0^{T_0}\f{c_w}{T}\ip \D S= 51.54\text{J/mol}$$
二者误差范围内相等, 即$S_g(0)=S_w(0)$.
\se{6.2}
$\d x\d p = \d x \d\sqrt{2m\ep}=\f{\sqrt{m}}{\sqrt{2\ep}}\d x\d \ep=h$. 则$\d p =\f{\sqrt{m}}{\sqrt{2\ep}}\d \ep$下, 量子态数为$\f{L\d p}{h} = \f{L}{h}\sqrt{\f{2\ep}{m}}\d\ep$
\se{6.4}
在$\d p=\d\ep/c$下, 量子态数为$\f{4\pi p^2V\d p}{h^3}$. 代入$\d p=\d\ep/c,\ p=\ep/c$.
$$D\d\ep = \f{4\pi \ep^2V\d\ep}{c^3h^3}$$
\se{6.5}
分布为玻耳兹曼分布. 则两个气体的微观状态数为:
$$\O=\f{N!}{\prod a_l!}\prod \o_l^{a_l}$$
$$\O'=\f{N'!}{\prod a_l'!} \prod {\o'}_l^{a_l'}$$
则总状态数为
$$\ln\O_0 = \ln\O\O' = N\ln N-\sum a_l\ln a_l+\sum a_l\ln \o_l+N'\ln N'-\sum a'_l\ln a'_l+\sum a'_l\ln \o'_l$$
$\O_0$为极大时, 变分为0:
$$\de\ln\O_0 = -\sum\ln\f{a_l}{\o_l}\de a_l-\sum\ln\f{a'_l}{\o'_l}\de a'_l=0$$
又根据$\sum a_l=0, \sum a'_l=0, \sum\ep_l\de a_l+\sum\ep'_l\de a'_l=0$. 用拉氏乘子得
$$\de\ln\O_0-\a\de N-\a'\de N'-\b\de E = -\sum(\ln\f{a_l}{\o_l}+\a+\b\ep_l)\de a_l-\sum(\ln\f{a'_l}{\o'_l}+\a'+\b\ep'_l)\de a'_l=0$$
各个变分项独立: 
$$\ln\f{a_l}{\o_l}+\a+\b\ep_l=0$$
$$\ln\f{a'_l}{\o'_l}+\a'+\b\ep'_l$$
即
$$a_l=\o_le^{-\a-\b\ep_l}$$
$$a'_l=\o'_le^{-\a'-\b\ep'_l}$$
\se{6.6}
约束条件不变. $\O$为玻色子, $\O'$为费米子, 微观状态数变为
$$\O=\prod \f{(\o_l+a_l-1)!}{a_l!(\o_l-1)!}$$
$$\O'=\prod \f{\o'_l}{a'_l!(\o'_l-a_l')!}$$
$$\ln\O_0 = \sum[(\o_l+a_l)\ln(\o_l+a_l)-a_l\ln a_l-\o_l\ln\o_l]+\sum[-(\o'_l-a'_l)\ln(\o'_l-a'_l)+a'_l\ln a'_l+\o'_l\ln\o'_l]$$
$$\de\ln\O_0 = \sum \f{\ln(\o_l+a_l)}{a_l}\de a_l+\sum\ln\f{\o_l'-a'_l}{a'_l}\de a_l'=0$$
利用乘子法:
$$=\sum (\f{\ln(\o_l+a_l)}{a_l}-\a-\b\ep_l)\de a_l+\sum (\ln\f{\o_l'-a'_l}{a'_l}-\a'-\b\ep_l')\de a_l'=0$$
即
$$a_l=\f{\o_l}{e^{\a+\b\ep_l}-1}$$
$$a'_l=\f{\o'_l}{e^{\a'+\b\ep'_l}-1}$$

\end{document}