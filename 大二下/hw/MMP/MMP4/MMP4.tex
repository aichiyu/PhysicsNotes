\documentclass[UTF8,9pt]{ctexart}
\usepackage{../../template/homeworkTEMP/hw}
\setcounter{secnumdepth}{0}
\title{数理方法II第四次作业} 
\begin{document} 
\maketitle
\se{1}
两边fourier变换后:
$$-(k_0^2+k^2)\wv{ G } = -\f{1}{\ep_0}$$
因此
$$\wv{ G } = \f{\f{1}{\ep_0}}{k_0^2+k^2}$$
求逆变换:
$$ G  = \f{1}{(2\pi)^3\ep_0}\iiint \f{e^{ikr \cos\t}}{k_0^2+k^2}\d\bm{k_0}$$
化简: 
$$\ar{
G  \x \f{1}{(2\pi)^3\ep_0}\iiint \f{e^{ikr\cos\t}}{k_0^2+k^2}\d\bm{k_0}\\
\x \f{1}{(2\pi)^3\ep_0}\int_0^{\pi}\d\t\int_0^{2\pi} \d \p\intzi k_0^2\sin\t\f{e^{ikr\cos\t}}{k_0^2+k^2}\d k_0\\
\x \f{1}{(2\pi)^2\ep_0}\int_0^{\pi}\d\t\intzi k_0^2\sin\t\f{e^{ikr\cos\t}}{k_0^2+k^2}\d k_0\\
\x \f{1}{i(2\pi)^2r\ep_0} \intzi \f{2k\sin kr}{k_0^2+k^2} \d k_0\\
\x \f{1}{i(2\pi)^2r\ep_0} \intii \f{ke^{ikr}}{k_0^2+k^2} \d k_0\\
\x \f{1}{i(2\pi)^2r\ep_0} \intii \f{ke^{ikr}}{k_0^2+k^2} \d k_0\\
\x \f{1}{i(2\pi)^2r\ep_0} 2 \pi i \operatorname{Res}\left[\f{z e^{i z r}}{z^2+k^2},z_0\right]\\
&\text{在}z_0 = ik_0\text{取一阶极点}\\
\x \f{1}{4\pi r\ep_0}e^{-kr}
}$$
\se{2}
根据定义
\begin{equation} 
K_{\alpha}(x)=\frac{\pi}{2} \frac{I_{-\alpha}(x)-I_{\alpha}(x)}{\sin (\alpha \pi)}
\end{equation}
且
\begin{equation} 
I_{\alpha}(x)=i^{-\alpha} J_{\alpha}(i x)
\end{equation}
将(2)代入(1)式, $$
K_{\alpha}(x)=\frac{\pi}{2} \frac{i^{\alpha} J_{-\alpha}(i x)-i^{-\alpha} J_{\alpha}(i x)}{\sin (\alpha \pi)}$$
对比$H_{\alpha }^{(1)}(x)$的定义
\begin{equation} H_{\alpha }^{(1)}(x)={\frac {J_{-\alpha }(x)-e^{-\alpha \pi i}J_{\alpha }(x)}{i\sin(\alpha \pi )}}\end{equation}
可以发现
$$K_{\alpha }(x)={\frac {\pi }{2}}i^{\alpha +1}H_{\alpha }^{(1)}(ix)$$
现将
$$N_{\alpha }(x)={\frac {J_{\alpha }(x)\cos(\alpha \pi )-J_{-\alpha }(x)}{\sin(\alpha \pi )}}$$
与(3)式对比, 可以发现$H_{\alpha }^{(1)}(ix) = J_{\a}(i x)+i N_{\a}(i x)$. 因此有
$$K_{\alpha }(x)={\frac {\pi }{2}}i^{\alpha +1}H_{\alpha }^{(1)}(ix) = J_{\a}(i x)+i N_{\a}(i x)$$
\qqed

\se{3}
(1) $\int x J_2(x)\d x$\\
由递推式得 $$J_2(x)=-J_0(x)+\frac{2}{x}J_1(x)$$ 
又由于 $$\int xJ_0(x)\d x=xJ_1(x) ,\quad \int J_1(x)\d x=-J_0(x) $$
从而$$ \int xJ_2(x)\d x=-xJ_1(x)-2J_0(x)$$ 

(2)$\int x^{4} J_{1}(x) \d x$\\
$$\ar{
\int x^{4} J_{1}(x) \d x \x x^4J_2(x)-2\int x^3J_2(x)\d x\\
\x x^4J_2(x)-2\int \d(x^3J_3(x))\\
\x x^4J_2(x)-2x^3J_3(x)
}$$

(3)$\int_{0}^{R} J_{0}(x) \cos x \d x$
$$\ar{
\int_{0}^{R} J_{0}(x) \cos x \d x \x xJ_{0}(x) \cos x\big|_0^R -\int_0^Rx(-J_1\cos x-J_0\sin x)\d x\\
\x xJ_{0}(x) \cos x\big|_0^R +\int_0^RxJ_1\cos x\d x+\int_0^R\sin x\d(x J_1)\\
\x xJ_{0}(x) \cos x\big|_0^R +\int_0^RxJ_1\cos x\d x+\of{xJ_1\sin x\big|_0^R -\int_0^RxJ_1\cos x\d x}\\
\x xJ_{0}(x) \cos x\big|_0^R + xJ_1\sin x\big|_0^R \\
\x RJ_0(R)\cos R +RJ_1(R)\sin R
}$$

(4)$3 J_{0}^{\prime}(x)+4 J_{0}^{\prime \prime \prime}(x)$
$$J_0' = -J_1$$
$$J_0''' = -J_1'' = -\ff{2}\of{J'_0-J'_2} = \ff{2}J_1+\ff{2}J_2' = \ff{2}J_1+\ff{4}(J_1-J_3) = \f{3}{4}J_1(x)-\ff{4}J_3(x)$$
因此
$$3 J_{0}^{\prime}(x)+4 J_{0}^{\prime \prime \prime}(x) = -3J_1(x) + 3J_1(x) -J_3(x) = J_3(x)$$

\se{4}
$$\ar{
W(J_v,J_{-v}) \x \abs{\begin{array}{ll}J_v & J_{-v} \\ J_v' & J_{-v}'\end{array}}\\
\x \abs{\begin{array}{ll}J_v & J_{-v} \\ \frac{1}{2}\left[J_{\nu-1}(x)-J_{\nu+1}(x)\right] & \frac{1}{2}\left[J_{-\nu-1}(x)-J_{-\nu+1}(x)\right]\end{array}}\\
\x J_vJ_{-v-1}-J_vJ_{-v+1}-J_{-v}J_{v-1}+J_{-v}J_{v+1}
}$$
由Bessel方程,
$$ 
\frac{d}{d x}\left[x \frac{d J_{v}(x)}{d x}\right]+x\left(1-\frac{v^{2}}{x^{2}}\right) J_{v}(x)=0
$$$$ 
\frac{d}{d x}\left[x \frac{d J_{-v}(x)}{d x}\right]+x\left(1-\frac{v^{2}}{x^{2}}\right) J_{-v}(x)=0
$$可得
$$ 
J_{-v}(x) \frac{d}{d x}\left[x \frac{d J_{v}(x)}{d x}\right]-J_{v}(x) \frac{d}{d x}\left[x \frac{d J_{-v}(x)}{d x}\right]=0
$$即$$ 
\frac{d}{d x}\left\{x\left[J_{-v}(x) J_{v}^{\prime}(x)-J_{v}(x) J_{-v}^{\prime}(x)\right]\right\}=0 \ip x\left[J_{-v}(x) J_{v}^{\prime}(x)-J_{v}(x) J_{-v}^{\prime}(x)\right]=C
$$
由Bessel函数表达式可确认C:
$$C=\frac{1}{\Gamma(-v+1)} \frac{v}{\Gamma(v+1)}-\frac{1}{\Gamma(v+1)} \frac{-v}{\Gamma(-v+1)}=\frac{2 v}{\Gamma(v+1) \Gamma(-v+1)}=\frac{2}{\Gamma(v) \Gamma(1-v)}=\frac{2 \sin \pi v}{\pi}$$
因此有
$$ 
W\left(J_{v}, J_{-v}\right)=J_{v}(x) J_{-v}^{\prime}(x)-J_{-v}(x) J_{v}^{\prime}(x)=-\frac{C}{x}=-\frac{2 \sin \pi v}{\pi x}
$$$$ 
W\left(J_{v}, Y_{v}\right)=\left| \begin{array}{ll}{J_{v}} & {Y_{v}} \\ {J_{v}^{\prime}} & {Y_{v}^{\prime}}\end{array}\right|=
\cot \pi v \left| \begin{array}{ll}{J_{v}} & {J_{v}} \\ {J_{v}^{\prime}} & {J_{v}^{\prime}}\end{array}\right|-\frac{1}{\sin \pi \nu} \left| \begin{array}{cc}{J_{v}} & {J_{-v}} \\ {J_{v}^{\prime}} & {J_{-\nu}^{\prime}}\end{array}\right|=\frac{2}{\pi x}$$

(1)\\
将$$ 
W\left(J_{v}, J_{-v}\right)=J_{v}(x) J_{-v}^{\prime}(x)-J_{-v}(x) J_{v}^{\prime}(x)=-\frac{C}{x}=-\frac{2 \sin \pi v}{\pi x}
$$两边同时乘$-\frac{\pi}{2 \sin \pi \nu} \frac{1}{J_{v}^{2}(x)}$,
$$ 
\frac{1}{x J_{v}^{2}(x)}=-\frac{\pi}{2 \sin \pi v} \frac{J_{v}(x) J_{-\nu}^{\prime}(x)-J_{-v}(x) J_{v}^{\prime}(x)}{J_{v}^{2}(x)}=-\frac{\pi}{2 \sin \pi v} \frac{d}{d x} \frac{d_{-v}(x)}{J_{v}(x)}
$$
因此$$ 
\int \frac{d x}{x J_{v}^{2}(x)}=-\frac{\pi}{2 \sin \pi v} \int d \frac{J_{-v}(x)}{J_{v}(x)}=-\frac{\pi}{2 \sin \pi v} \frac{J_{-v}(x)}{J_{v}(x)}+C
=\frac{\pi}{2} \frac{Y_{v}(x)}{J_{v}(x)}+C^{\prime}
$$

(2)\\
由$W\left(J_{v}, Y_{v}\right)$可得 $$ 
\frac{1}{x Y_{v}^{2}(x)}=-\frac{\pi}{2} \frac{d}{d x} \frac{J_{v}(x)}{Y_{v}(x)}
$$
因此$$ 
\int \frac{d x}{x Y_{v}^{2}(x)}=-\frac{\pi}{2} \frac{J_{v}(x)}{Y_{v}(x)}+C
$$

(3)\\
同理$$ 
\frac{1}{x J_{v}(x) Y_{v}(x)}=\frac{\pi}{2}\left[\frac{Y_{v}^{\prime}(x)}{Y_{v}(x)}-\frac{J_{v}^{\prime}(x)}{J_{v}(x)}\right]
$$
积分得
$$ 
\int \frac{d x}{x J_{v}(x) Y_{v}(x)}=\frac{\pi}{2} \int\left[\frac{Y_{v}^{\prime}(x)}{Y_{v}(x)}-\frac{J_{v}^{\prime}(x)}{J_{v}(x)}\right] d x=\frac{\pi}{2} \ln \frac{Y_{v}(x)}{J_{v}(x)}+C
$$
\se{5}
令$x = \b z^\g,u=z^\a y$,
$$ 
\frac{d u}{d z}=\alpha z^{\alpha-1} y+z^{\alpha} \frac{d y}{d z}=\alpha z^{\alpha-1} y+\beta \gamma z^{\alpha+\gamma-1} \frac{d y}{d x}=\alpha z^{\alpha-1} y+\gamma x z^{\alpha-1} \frac{d y}{d x}
$$$$ 
\frac{d^{2} u}{d z^{2}}=\alpha(\alpha-1) z^{\alpha-2} y+\alpha z^{\alpha-1} \frac{d y}{d z}+\beta \gamma(\alpha+\gamma-1) z^{\alpha+\gamma-2} \frac{d y}{d x}+\beta \gamma z^{\alpha+\gamma-1} \frac{d}{d z} \frac{d y}{d x}
$$$$ 
=\alpha(\alpha-1) z^{\alpha-2} y+\gamma(2 \alpha+\gamma-1) x z^{\alpha-2} \frac{d y}{d x}+\gamma^{2} x^{2} z^{\alpha-2} \frac{d^{2} y}{d x^{2}}
$$
代入方程可化简为
$$ 
x^{2} \frac{d^{2} y}{d x^{2}}+x \frac{d y}{d x}+\left(x^{2}-v^{2}\right) y=0
$$
这是$v$阶Bessel方程, 通解为
$$ 
y=c_{1} J_{v}(x)+c_{2} Y_{v}(x)
$$因此$$ 
u=z^{\alpha} y=c_{1} z^{\alpha} J_{v}(x)+c_{2} z^{\alpha} Y_{v}(x)=c_{1} z^{\alpha} J_{v}\left(\beta z^{\gamma}\right)+c_{2} z^{\alpha} Y_{v}\left(\beta z^{\gamma}\right)
$$

1.\\
令$\alpha=\frac{1}{2}, \quad \beta=\frac{2 \sqrt{a}}{b+2}, \gamma=\frac{b}{2}+1, \quad v=\frac{1}{b+2}$, 变为上述通解, 因此其解为
$$u=c_{1} \sqrt{z} J_{\frac{1}{b+2}}\left(\frac{2 \sqrt{a}}{b+2} z^{\frac{b}{2}+1}\right)+c_{2} \sqrt{z} Y_{\frac{1}{b+2}}\left(\frac{2 \sqrt{a}}{b+2} z^{\frac{b}{2}+1}\right)$$


2.\\令$\alpha=2, \quad \beta=1, \quad \gamma=1, \quad v=2$
,解为
$$ 
u=c_{1} z J_{1 / 2}\left(z^{2}\right)+c_{2} z Y_{1 / 2}\left(z^{2}\right)
$$
\se{6}
问题为$$ 
\left\{\begin{array}{c}
{\frac{\partial u}{\partial t}-\kappa \frac{1}{\rho} \frac{\partial}{\partial \rho}\left(\rho \frac{\partial u}{\partial \rho}\right)=0} \\ 
{u\big|_{\rho=0}\text{有界},\ \left.\frac{\partial u}{\partial \rho}\right|_{\rho=a}=0} \\ 
{\left.u\right|_{t=0}=u_{0}\left(1-\frac{\rho^{2}}{a^{2}}\right)}
\end{array}\right.
$$
分离变量得$$ 
\left\{\begin{array}{l}{\frac{1}{\rho} \frac{d}{d \rho}\left(\rho \frac{d \mathrm{P}}{d \rho}\right)+\lambda^{2} \mathrm{P}=0} \\ {\mathrm{P}(0)\text{有界},\ P'(a)=0}\end{array}\right.,\quad T^{\prime}+\lambda^{2} \kappa T=0
$$
解本征值问题得$\lambda_{0}=0, \quad \lambda_{i}=\frac{\mu_{i}^{\prime}}{a}$,
$$\mathrm{P}_{0}(\rho)=A_{0}, \quad \mathrm{P}_{i}(\rho)=J_{0}\left(\frac{\mu_{i}^{\prime}}{a} \rho\right)$$
$$ 
T(t)=A_{i} \exp \left[-\kappa\left(\frac{\mu_{i}}{a}\right)^{2} t\right]
$$
所以$$ 
u(\rho, t)=A_{0}+\sum_{i=1}^{\infty} A_{i} J_{0}\left(\frac{\mu_{i}^{\prime}}{a} \rho\right) \exp \left[-\kappa\left(\frac{\mu_{i}}{a}\right)^{2} t\right]
$$$$ 
\left.u\right|_{t=0}=A_{0}+\sum_{i=1}^{\infty} A_{i} J_{0}\left(\frac{\mu_{i}^{\prime}}{a} \rho\right)=u_{0}\left(1-\frac{\rho^{2}}{a^{2}}\right)
$$$$ 
A_{0}=\frac{2 u_{0}}{a^{2}} \int_{0}^{a}\left(1-\frac{\rho^{2}}{a^{2}}\right) \rho d \rho=\frac{u_{0}}{2}
$$$$ 
A_{i}=-\frac{4 u_{0}}{\mu^{\prime 2} J_{0}\left(\mu_{i}^{\prime}\right)}
$$最后得到$$ 
u(\rho, t)=\frac{u_{0}}{2}-4 u_{0} \sum_{i=1}^{\infty} \frac{1}{\mu^{\prime 2} J_{0}\left(\mu_{i}^{\prime}\right)} J_{0}\left(\frac{\mu_{i}^{\prime}}{a} \rho\right) \exp \left[-\kappa\left(\frac{\mu_{i}}{a}\right)^{2} t\right]
$$
稳态为$$ 
u \rightarrow \frac{u_{0}}{2}
$$
\se{7}
分离变量得本征值问题$$ 
\left\{\begin{array}{l}{Z^{\prime \prime}+k^{2} Z=0} \\ {Z(0)=0, Z(h)=0}\end{array}\right.
$$$$ 
\frac{1}{\rho} \frac{d}{d \rho}\left(\rho \frac{d \mathrm{P}}{d \rho}\right)-k^{2} \mathrm{P}=0
$$
解得本征函数$Z_{n}(z)=\sin \frac{n \pi}{h} z$,$\mathrm{P}_{n}(\rho)=I_{0}\left(\frac{n \pi}{h} \rho\right)$.$$ 
u(\rho, z)=\sum_{n=1}^{\infty} A_{n} I_{0}\left(\frac{n \pi}{h} \rho\right) \sin \frac{n \pi}{h} z
$$
根据边界条件得
$$ 
A_{2}=\frac{u_{0}}{I_{0}\left(\frac{2 \pi a}{h}\right)}, A_{n\neq2}=0
$$即$$ 
u(\rho, z)=\frac{u_{0}}{I_{0}\left(\frac{2 \pi a}{h}\right)} I_{0}\left(\frac{2 \pi}{h} \rho\right) \sin \frac{2 \pi}{h} z
$$
\se{8}
$$ 
\int_{-1}^{1}(1+x)^{k} P_{l}(x) d x=\frac{1}{2^{l} l !} \int_{-1}^{1}(1+x)^{k} \frac{d^{l}}{d x^{l}}\left(x^{2}-1\right)^{l} d x
$$$$ 
=\frac{1}{2^{l} l !}(1+x)^{k} \frac{d^{l-1}}{d x^{l-1}}\left.\left(x^{2}-1\right)^{l}\right|^{1}-\frac{1}{2^{l} l !} \int_{-1}^{1} \frac{d(1+x)^{k}}{d x} \frac{d^{l-1}}{d x^{l-1}}\left(x^{2}-1\right)^{l} d x
$$\begin{equation} 
=-\frac{1}{2^{l} l !} \int_{-1}^{1} \frac{d(1+x)^{k}}{d x} \frac{d^{l-1}}{d x^{l-1}}\left(x^{2}-1\right)^{l} d x=\cdots=\frac{1}{2^{l} l !} \int_{-1}^{1}\left(1-x^{2}\right)^{l} \frac{d^{l}(1+x)^{k}}{d x^{l}} d x
\end{equation}$$ 
=\frac{k !}{2^{l} l !(k-l) !} \int_{-1}^{1}\left(1-x^{2}\right)^{l}(1+x)^{k-l} d x=\frac{k !}{2^{l} l !(k-l) !} \int_{-1}^{1}(1-x)^{l}(1+x)^{k} d x
$$作代换$1+x=2t$$$ 
=\frac{2^{k+1} k !}{l !(k-l) !} \int_{0}^{1}(1-t)^{l} t^{k} d t=\frac{2^{k+1} k !}{l !(k-l) !} \mathrm{B}(l+1, k+1)
$$$$ 
=\frac{2^{k+1} k !}{l !(k-l) !} \frac{\Gamma(l+1) \Gamma(k+1)}{\Gamma(l+k+2)}=\frac{2^{k+1}(k !)^{2}}{(k-l) !(l+k+1) !}
$$
$k<l$时根据(4)式, 积分为0. 
\se{9}
$$u=\sum_{n=0}^\infty(a_nr^n+\f{b_n}{r^{n+1}})P_n(\cos\t)$$
$$ 
\left.u\right|_{r=a}=\sum_{l=0}^{\infty}\left(A_{i} a^{l}+\frac{B_{l}}{a^{l+1}}\right) P_{l}(\cos \theta)=u_{0} P_{0}(\cos \theta)
$$$$ 
\left.u\right|_{r=b}=\sum_{l=0}^{\infty}\left(A_{l} b^{l}+\frac{B_{l}}{b^{l+1}}\right) P_{l}(\cos \theta)=u_{0} \cos ^{2} \theta=\frac{1}{3} u_{0} P_{0}(\cos \theta)+\frac{2}{3} u_{0} P_{2}(\cos \theta
$$比较系数得$$ 
A_{0}=\frac{b-3 a}{3(b-a)} u_{0}, B_{0}=\frac{2 a b}{3(b-a)} u_{0}, A_{2}=\frac{2 b^{3}}{3\left(b^{5}-a^{5}\right)} u_{0}, B_{2}=\frac{2 a^{5} b^{3}}{3\left(a^{5}-b^{5}\right)} u_{0}
$$因此$$ 
u(r, \theta)=\frac{b-3 a}{3(b-a)} u_{0}+\frac{2 b}{3(b-a)} \frac{a}{r} u_{0}+\frac{2 b^{3} a^{2} u_{0}}{3\left(b^{5}-a^{5}\right)}\left[\left(\frac{r}{a}\right)^{2}-\left(\frac{a}{r}\right)^{3}\right] P_{2}(\cos \theta)
$$
\se{10}
(1)\\
$$(\sin \t-2\cos^2\t)\cos^2\vp = \ff{2}(\sqrt{1-\cos^2\t}-2\cos^2\t)+\ff{2}(\sqrt{1-\cos^2\t}-2\cos^2\t)\cos2\vp$$
上式右端第一项需展开为$P^0_l$, 第二项需展开为$P^2_l$
$$\ar{
    \sqrt{1-\cos^2\t} \x \sumzi A_lP_l^0(\cos\t),\quad A_l=\f{2l+1}{2}\int_{-1}^1\sqrt{1-x^2}P_l^0\d x\\
    \x \sumzi B_lP_l^2(\cos\t),\quad B_l=\f{(2l+1)(l+2)!}{2(l-2)!}\int_{-1}^1\sqrt{1-x^2}P_l^2\d x
}$$
$$\ar{
    \cos^2\t \x \f{2}{3}P_2^0(\cos\t)+\ff{3}P_0^0(\cos\t)\\
    \x \sumzi C_lP_l^2(\cos\t),\quad C_l=\f{(2l+1)(l+2)!}{2(l-2)!}\int_{-1}^1x^2P_l^2\d x
}$$
于是有
$$(\sin \t-2\cos^2\t)\cos^2\vp = $$
$$\ff{2}(\sumzi A_lP_l^0(\cos\t)-2(\f{2}{3}P_2^0(\cos\t)+\ff{3}P_0^0(\cos\t)))\Phi^0$$
$$+\ff{2}(\sumzi B_lP_l^2(\cos\t)-2\sumzi C_lP_l^2(\cos\t))\Phi^2$$
$$ = \ff{2}\sumzi A_lY_l^0-\f{2}{3}Y^0_2-\ff{3}Y_0^0+\ff{2}\sumzi B_lY_l^2-\sumzi C_lY_l^2$$
*$A,B,C$的定义在前面几个式子中. \\
(2)\\
$$(1-2\sin\t)\cos\t\cos\vp=(\cos\t-2\sin\t\cos\t)\cos\vp$$
$$\ip = \sumoi A_lY_l^1-\ff{2}{3}Y_2^1 ,\quad A_l=\f{(2l+1)(l+2)!}{2(l-2)!}\int_{-1}^1xP_l^1\d x$$
\se{11}
令$$ 
u(r, \theta, \varphi)=\sum_{l=0}^{\infty} \sum_{m=0}^{l} P_{l}^{m}(\cos \theta)\left[R_{l, m}(r) \sin m \varphi+S_{l, m}(r) \cos m \varphi\right]
 $$代入方程可化简为$$ 
 =A+B r^{2} \sin 2 \theta \cos \varphi=A P_{0}^{0}(\cos \theta)-\frac{2}{3} B r^{2} P_{2}^{1}(\cos \theta) \cos \varphi
  $$因此$$ 
  \frac{d}{d r}\left(r^{2} \frac{d R_{l, m}}{d r}\right)-l(l+1) R_{l, m}=0, \frac{d}{d r}\left(r^{2} \frac{d S_{0,0}}{d r}\right)=A r^{2}, \frac{d}{d r}\left(r^{2} \frac{d S_{2,1}}{d r}\right)-6 S_{2,1}=-\frac{2}{3} B r^{2}
   $$
   由边界条件可得$$ 
   R_{l, m}(r)=0, \quad S_{0,0}(r)=\frac{A}{6}\left(r^{2}-a^{2}\right), \quad S_{2,1}(r)=\frac{1}{21} B r^{2}\left(a^{2}-r^{2}\right)
    $$代入即为$$ 
   \begin{aligned} u(r, \theta, \varphi) &=\frac{A}{6}\left(r^{2}-a^{2}\right)+\frac{B}{21} r^{2}\left(a^{2}-r^{2}\right) P_{2}^{1}(\cos \theta) \cos \varphi \\ &=\frac{A}{6}\left(r^{2}-a^{2}\right)+\frac{B}{14} r^{2}\left(r^{2}-a^{2}\right) \sin 2 \theta \cos \varphi \end{aligned}
    $$
\end{document}
