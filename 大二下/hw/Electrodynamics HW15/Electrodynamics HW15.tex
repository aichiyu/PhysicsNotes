\documentclass[UTF8,9pt]{ctexart}
\usepackage{../../template/homeworkTEMP/hw}
\setcounter{secnumdepth}{0}
\title{The 15th HW of Electrodynamics}
\begin{document} 
\maketitle
\se{1}Prove that the above invariants are indeed invariant under Lorentz transformation.
$$\frac{1}{2} F_{\mu \nu} F_{\mu \nu}=B^{2}-\frac{1}{c^{2}} E^{2}$$
$$\frac{i}{8} \epsilon_{\mu \nu \lambda \tau} F_{\mu \nu} F_{\lambda \tau}=\frac{1}{c} \vec{B} \cdot \vec{E}$$
Proof:\\
\sub{1}对于式
$$\frac{1}{2} F_{\mu \nu} F_{\mu \nu}=B^{2}-\frac{1}{c^{2}} E^{2}$$
定义$$ 
a=\left(\begin{array}{cccc}{\gamma} & {0} & {0} & {i \beta \gamma} \\ {0} & {1} & {0} & {0} \\ {0} & {0} & {1} & {0} \\ {-i \beta \gamma} & {0} & {0} & {\gamma}\end{array}\right)
 $$
$$\vec{B'}=\g(\vec{B}-\f{\vec{v}}{c^2}\tm\vec{E})$$
$$\vec{E'}=\g(\vec{E}-\vec{v}\tm\vec{B})$$
则有
$$ 
F_{\mu \nu}^{\prime}=a_{\mu \lambda} a_{v \tau} F_{\lambda \tau} = aF\wv{a}=
\left(\begin{array}{cccc}
0 & \g(B_3-\f{v}{c^2}E_2) &-\g(B_2+\f{v}{c^2}E_3) &-\f{i}{c}E_1 \\ 
\g(-B_3+\f{v}{c^2}E_2) & 0& B_1 &-\f{i}{c}\g(E_2-vB_3)\\ 
\g(B_2+\f{v}{c^2}E_3) & -B_1 &0 &-\f{i}{c}\g(E_3+vB_2)\\ 
\f{i}{c}E_1 & \f{i}{c}\g(E_2-vB_3) & \f{i}{c}\g(E_3+vB_2) & 0
\end{array}\right)
 $$
$F_{\mu \nu}^{\prime}F_{\mu \nu}^{\prime} = 2*B1^2 - (2*E1^2)/c^2 + 2*B2^2*g^2 + 2*B3^2*g^2 - (2*E2^2*g^2)/c^2 - (2*E3^2*g^2)/c^2 - (2*B2^2*g^2*v^2)/c^2 - (2*B3^2*g^2*v^2)/c^2 + (2*E2^2*g^2*v^2)/c^4 + (2*E3^2*g^2*v^2)/c^4 + (8*B2*E3*g^2*v)/c^2
$\\\\
$B'^{2}-\frac{1}{c^{2}} E'^{2} = B1^2 - E1^2/c^2 + B2^2*g^2 + B3^2*g^2 - (E2^2*g^2)/c^2 - (E3^2*g^2)/c^2 - (B2^2*g^2*v^2)/c^2 - (B3^2*g^2*v^2)/c^2 + (E2^2*g^2*v^2)/c^4 + (E3^2*g^2*v^2)/c^4
$\\
可以发现$$\ff{2}(F_{\mu \nu}^{\prime}F_{\mu \nu}^{\prime}) = B'^{2}-\frac{1}{c^{2}} E'^{2}$$
\sub{2}
对于式
$$\frac{i}{8} \epsilon_{\mu \nu \lambda \tau} F_{\mu \nu} F_{\lambda \tau}=\frac{1}{c} \vec{B} \cdot \vec{E}$$
洛伦兹变换后:
$$\sum_{\mu=1}^4\sum_{\nu=1}^4\sum_{\l=1}^4\sum_{\tau=1}^4(\epsilon_{\mu \nu \lambda \tau} F'_{\mu \nu} F'_{\lambda \tau}) = -\f{8}{ic}\vec{B}\cdot\vec{E}$$
由于$\vec{B}\cdot\vec{E}$是一个洛伦兹不变量, $\vec{B'}\cdot\vec{E'}=\vec{B}\cdot\vec{E}$
因此
$$\sum_{\mu=1}^4\sum_{\nu=1}^4\sum_{\l=1}^4\sum_{\tau=1}^4(\epsilon_{\mu \nu \lambda \tau} F'_{\mu \nu} F'_{\lambda \tau}) = -\f{8}{ic}\vec{B'}\cdot\vec{E'}$$
\se{2}
(1) Lab系中系统的动量为
$$ p = \f{\sqrt{E_1^2-m_1^2c^4}}{c}$$
洛伦兹变换得
$$ 
\begin{array}{l}{p_{1}=\frac{p_{1}^{\prime}+\frac{\beta_{c}}{c^{2}} E_{1}^{\prime}}{\sqrt{1-\beta_{c}^{2} / c^{2}}} ; E_{1}=\frac{E_{1}^{\prime}+\beta_{c} p_{1}^{\prime}}{\sqrt{1-\beta_{c}^{2} / c^{2}}}} \\ {p_{2}=\frac{p_{2}^{\prime}+\frac{\beta_{c}}{c^{2}} E_{2}^{\prime}}{\sqrt{1-\beta_{c}^{2} / c^{2}}} ; \quad E_{2}=\frac{E_{2}^{\prime}+\beta_{c} p_{2}^{\prime}}{\sqrt{1-\beta_{c}^{2} / c^{2}}}}\end{array}
 $$
即
$$ 
\begin{array}{c}{p_{1}=\gamma \frac{\beta_{c}}{c^{2}}\left(E_{1}^{\prime}+E_{2}^{\prime}\right)} \\ {E_{1}+E_{2}=\gamma\left(E_{1}^{\prime}+E_{2}^{\prime}\right)}\end{array}
 $$
因此$$ 
\beta_{c}=\frac{p_{1} c^{2}}{E_{1}+E_{2}}=\frac{\sqrt{E_{1}^{2}+m_{1}^{2} c^{4}}}{E_{1}+m_{2} c^{2}} c
 $$
 (2)$$ 
 \left|\vec{p}_{1}^{\prime}\right|=\frac{m_{2} \sqrt{E_{1}^{2}-m_{1}^{2} c^{4}}}{M c}, \quad\left|\vec{p}_{2}^{\prime}\right|=\left|\vec{p}_{1}^{\prime}\right|
  $$
  $$ 
E_{1}^{\prime}=\sqrt{p_{1}^{\prime 2} c^{2}+m_{1}^{2} c^{4}}=\frac{m_{1}^{2} c^{2}+m_{2} E_{1}}{M}, \quad E_{2}^{\prime}=\sqrt{p_{2}^{\prime 2} c^{2}+m_{2}^{2} c^{4}}=\frac{m_{2} E_{1}+m_{2}^{2} c^{2}}{M}
 $$
因此总能量$$ 
 E^{\prime}=E_{1}^{\prime}+E_{2}^{\prime}=\frac{\left(m_{1}^{2}+m_{2}^{2}\right) c^{2}+2 m_{2} E_{1}}{M}
  $$其中$$ 
  M^{2} c^{4}=m_{1}^{2} c^{4}+m_{2}^{2} c^{4}+2 m_{2} E_{1} c^{2}
   $$
(3) 实验室系中$$ 
p_{\mu}=\left[\vec{p}_{1}+\vec{p}_{2}, \frac{i}{c}\left(E_{1}+E_{2}\right)\right]=\left[]\vec{p}, \frac{i}{c}\left(E_{1}+E_{2}\right)\right]
$$质心系中$$ 
p_{v}^{\prime}=\left[\vec{p}_{1}^{\prime}+\vec{p}_{2}^{\prime}, \frac{i}{c}\left(E_{1}^{\prime}+E_{2}^{\prime}\right)\right]=\left[0, \frac{i}{c} 2 E_{1}^{\prime}\right]
$$
可得$$ 
-2 m_{e} E_{1}=-\frac{1}{c^{2}} 4 E_{1}^{\prime 2}
 $$
 即
 $$ 
E_{1}=\frac{2 E_{1}^{\prime 2}}{m_{e} c^{2}}=1.9 \times 10^{4} G e V
 $$
\end{document}