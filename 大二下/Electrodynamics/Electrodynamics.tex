\documentclass[UTF8,9pt]{ctexart}
\usepackage{../template/Notes/notes}
\title{ElectroDynamics}
\begin{document}
\maketitle
\se{方程} 
真空麦克斯韦方程 
$$\ar{
    \nabla\cdot\bm{E} \x  \f{\rho}{\epsilon_0}&\qquad \oint_S\bm{E}\d \bm{s} \x  \f{Q}{\epsilon_0}&\text{高斯定律}\\
    \nabla\cdot\bm{B} \x  0&\qquad \oint_S\bm{B}\d \bm{s} \x  0&\text{高斯磁定律}\\
    \nabla\times\bm{E} \x  -\pt{\bm{B}}&\qquad \oint_L\bm{E}\d \bm{l} \x  -\dt{\varphi_B}&\text{法拉第电磁感应定律}\\
    \nabla\times\bm{B} \x  \mu_0\bm{J}+\mu_0\epsilon_0 \pt{\bm{E}}&\qquad \oint_L\bm{B}\d \bm{l} \x  \mu_0I+\mu_0\epsilon_0\dt{\varphi_E}\quad &\text{安培定律}
}$$

物质内麦克斯韦方程
$$\ar{
    \nabla\cdot\bm{D} \x  \rho_f&\qquad \oint_S\bm{D}\d \bm{s} \x  Q_f&\text{高斯定律}\\
    \nabla\cdot\bm{B} \x  0&\qquad \oint_S\bm{B}\d \bm{s} \x  0&\text{高斯磁定律}\\
    \nabla\times\bm{E} \x  -\pt{\bm{B}}&\qquad\quad\qquad \oint_L\bm{E}\d \bm{l} \x  -\dt{\varphi_B}&\text{法拉第电磁感应定律}\\
    \nabla\times\bm{H} \x  \bm{J}_f+\pt{\bm{D}}&\qquad \oint_L\bm{H}\d \bm{l} \x  I_f+\dt{\varphi_D}\quad\qquad &\text{安培定律}
}$$

镜像法\\
距半径为$R_0$的球的球心距离为$a$处有一点电荷$q$, 则镜像电荷$-\f{R_0}{a}q$距球心$\f{R_0^2}{a}$远,且在靠近$q$方向.

球谐函数解拉普拉斯方程
$$\vp(r,\t)=\sum_{n=0}^\infty(a_nr^n+\f{b_n}{r^{n+1}})P_n(\cos\t)$$
$$\left\{\ar{
    P_0(\cos\t)  \x  1\\
    P_1(cos\t)  \x  \cos\t\\
    P_2(\cos\t)  \x  \ff{2}(3\cos^2\t-1)
}\right.$$
$$\frac{1}{\left| \bm{R} -\bm{a}' \right|} = \frac{1}{\sqrt{R^2+a^{\prime 2} - 2Ra\cos\t}} = \left\{\ar{
\ff{a}\sum_{n=0}^{\infty} (\f{R}{a})^n P_n(\cos \t),\quad (R<a)\\
\ff{R}\sum_{n=0}^{\infty} (\f{a}{R})^n P_n(\cos \t),\quad (R>a)
}\right.$$

格林函数法
$$\nabla^2G(x,x')=-\ff{\ep}\de^3(x-x')$$
(1) 无界空间中
$$G(x,x')=\ca\ff{\abs{r-r'}}$$
(2) 上半平面中
$$G(x,x')=\ca(\ff{\abs{r-r'}}-\ff{\abs{r+r'}})$$
(3) 球外空间 ($R'$为电荷位置, $\a$为场点与电荷位置夹角,$R_0$为球半径).
$$G(x,x')=\ca(\ff{R^2+R'^2-2RR'\cos\a}-\ff{(\f{RR'}{R_0})^2+R_0^2-2RR'\cos\a})$$
给定$\rho(x')$, 第一类边值问题的解为($G$交换了$x,x'$):
$$\vp(x)=\int_VG(x',x)\rho(x')\d V'+\ep_0\oint_S(G(x',x)\pp{\vp}{n'}-\vp(x')\pp{G(x',x)}{n'}\d S')$$
若给定边界$\vp$, 则应使$G$在边界为0, 若给定边界$\pp{\vp}{n}$, 则应使$\pp{G}{n}$在边界为0.

泊松方程 
$${\nabla}^2 \Phi =\div\bm{E}= - {\rho \over \epsilon_0}$$

电多极矩
$$\varphi(\bm{x})=\frac{1}{4 \pi \epsilon_{0}}(\frac{q}{R}+\f{\bm{p}\cdot\bm{R}}{R^3} + \frac{1}{6} \sum_{i, j} \mathfrak{D}_{i j} \frac{\partial^{2}}{\partial x_{i} \partial x_{j}} \frac{1}{R}+\cdots)$$
$$\bm{E}=-\ca(\f{3(\bm{p}\cdot\bm{R})\bm{R}}{R^5}-\f{\bm{p}}{R^3})$$
$$\bm{p}=\iiint_V\rho(x')x'\d^3x'$$
$$\mathfrak{D}=\iiint_{V} 3 \bm{x}^{\prime} \bm{x}^{\prime} \rho(\bm{x}^{\prime}) d^{3} x^{\prime}$$

磁偶极矩
$$\vp = \f{\bm{m}\cdot\bm{R}}{4\pi R^3}$$
$$\bm{m}=\ff{2}\iiint_V\bm{x}'\tm \bm{J}(\bm{x'})\d^3x'$$

保角变换 ($z_1$为原来的点, $a$为夹角出现的位置的横坐标, $\a$为边界夹角).
$$\frac{\d z_{1}}{\d z_{2}}=C_{1} \prod_{i=1}^{n}(z_{2}-a_{i})^{\frac{\alpha_{i}}{\pi}-1}$$

电荷
$$\ar{
    \text{面电荷}&&\text{电场}&&\text{磁场}\\
    \hline
    \sigma_{polar}   \x   P\quad&\quad \rho_p  \x  -\div\bm{P} \quad&\quad \rho_M  \x  -\mu_0\div\bm{M}\\
    \sigma_{free}   \x   D \quad&\quad \rho_f  \x  \div\bm{D} \quad&\quad \rho_f  \x  0 \\
    \sigma_{total}   \x   \epsilon_0E \quad&\quad \rho_{tot}   \x   \ep_0\div\bm{E} \quad&\quad \rho_{tot} = \rho_p   \x   \mu_0\div\bm{H}\\
    \hline
    &&\ep_0E  \x  D-P&B  \x  \mu_0(H+M)
}$$

边界条件
$$\ar{
    \text{磁场}&&&\text{电场}\\
    \hline
    \varphi_{1} \x  \varphi_{2}&\varphi_{1} \x  \varphi_{2} \\ 
    \mu_{1} \frac{\partial \varphi_{1}}{\partial n}  \x   \mu_{2} \frac{\partial \varphi_{2}}{\partial n} & \varepsilon_{1} \frac{\partial \varphi_{1}}{\partial n}+\sg_f \x  \varepsilon_{2} \frac{\partial \varphi_{2}}{\partial n}\\
    &&\sg_{\text{电导}1}\frac{\partial \varphi_{1}}{\partial n} \x  \sg_{\text{电导}2}\frac{\partial \varphi_{2}}{\partial n}\\
    H_{1\parallel}+\bm{\a}_f \x  H_{2\parallel}&E_{1\parallel} \x  E_{2\parallel}\\  
    B_{1\perp} \x  B_{2\perp}&D_{1\perp}+\sg_f \x  D_{2\perp}\\
    M_{1\perp}+\a_M  \x   M_{2\perp} & \\
    B_{2\perp}-B_{1\perp} \x   \mu_0(\a_f+\a_M)&E_{2\perp}-E_{1\perp} \x  \,(\sg_f+\sg_p)/\ep_0\\
    B_\perp  \x   0 (\text{超导球}) & D_\perp  \x   \sg_f (\text{导体})
}$$

洛伦兹力:$$\ar{
    \bm{F}  \x  q\bm{E}+q\bm{v}\times\bm{B}\\
    \bm{f}  \x  \rho\bm{E}+\bm{J}\times\bm{B}
}$$

磁偶极子:
$$ 
\vec{B}=\frac{\mu_{0}}{4 \pi} \frac{3(\vec{m} \cdot \widehat{R}) \widehat{R}-\vec{m}}{R^{3}}
 $$
$$\vp = \f{\bm{m}\cdot\bm{R}}{4\pi R^3}$$

电磁场:
$$\bm{S}=\bm{E}\times\bm{H}$$
$$w=\ff{2}(\bm{E\cdot D}+\bm{H\cdot B})=\ff{2}(\rho\vp+\bm{J}_f\cdot\bm{A})$$

电流:
$$\nabla\cdot J=-\pt{\rho}$$
$$J=\sigma E$$

毕奥——萨伐尔定律  $\bm{B}=\frac{\mu_0}{4\pi}\int\frac{I\d l\times \bm{e}_r}{r^2}$,若$I$为直线,$B=\f{\mu_0Il}{4\pi r^2}$.


磁矢势:\\
库仑规范:
$$\ar{
    \bm{\nabla}\cdot \bm{A} \x 0\\
    \nabla^2 \varphi \x -\f{\varphi}{\epsilon_0}\\
    \bm{\square A} \x -\mu_0\bm{J}+\ff{c^2}\nabla\pt{\varphi}
}$$
洛伦兹规范:
$$  \bm{\nabla}\cdot\bm{A}+\ff{c^2}\pt{\varphi} = 0$$
达朗贝尔方程:
$$    \ar{\square \varphi \x -\f{\rho}{\epsilon_0}\\
    \bm{\square A} \x -\mu_0\bm{J}}$$
\sub{超导体}
临界磁场: 超过$H_c(T)=H_c(0)\left[1-\of{\f{T}{T_c}}^2\right]$时, 超导电性会被破坏. \\
迈斯纳效应: 超导体内部$B=0$. \\
伦敦第一方程
$$\pp{J_s}{t}=\a E,\a=\f{n_se^2}{m}$$
伦敦第二方程
$$\nabla\tm\bm{J_s}=-\a B$$
\se{电磁波的传播}
\sub{电磁波}
$$\ar{
    \bm{\nabla}\times \bm{B}-\ff{c^2}\pt{\bm{E}} \x  0\quad&\quad \bm{\nabla^2 E}-\ff{c^2}\pt[2]{\bm{E}}=\square \bm{E} \x  0\\
    \bm{\nabla}\times \bm{E}+\pt{\bm{B}} \x  0&\bm{\nabla^2 B}-\ff{c^2}\pt[2]{\bm{B}}=\square \bm{B} \x  0
}$$
$$\abs{\f{E}{B}}=\ff{\sqrt{\mu\ep}}=v$$
$$S=\sqrt{\f{\ep}{\mu}}E^2\bm{e}_k=vw\bm{e}_k$$
群速与相速关系
$$v_g=\dd{\o}{k}=v_p+k\dd{v_p}{k}=\f{c}{n+\o(\d n/\d\o)}$$
导体内波矢量$k=\b+i\a,v\o/\b$. 垂直入射时$\a\approx\b\approx\sqrt{\o\mu\sg/2}, B\approx \sqrt{\mu\sg/\o}e^{i\f{\pi}{4}}e_n\tm E$. B的相位比E滞后1/4. 金属内部主要是磁场能. 电磁波穿透深度为$\de=\ff{\a}=\sqrt{2/\o\mu\sg}$, 此为趋肤效应. 
\sub{反射}
介质界面上的边界条件为
$$\ar{
    e_n\tm(E_2-E_2)\x0\\
    e_n\tm(H_2-H_1)\x\a
}$$
入射波, 反射波和折射波分别为$E,E',E''$. 从介质1射向介质2. 有边界条件$\bm{k\cdot x}=\bm{k'\cdot x}=\bm{k''\cdot x}$\\.
菲涅耳公式:\\
当$E\perp$入射面(s光):
\begin{equation}
    \f{E'}{E} = \f{\sqrt{\ep_1}\cos\t-\sqrt{\ep_2}\cos\t''}{\sqrt{\ep_1}\cos\t+\sqrt{\ep_2}\cos\t''}=-\f{\sin(\t-\t'')}{\sin(\t+\t'')}
\end{equation}
\begin{equation}
    \f{E''}{E} = \f{2\sqrt{\ep_1}\cos\t}{\sqrt{\ep_1}\cos\t+\sqrt{\ep_2}\cos\t''}=\f{2\cos\t\sin\t''}{\sin(\t+\t'')}
\end{equation}
当$E\parallel$入射面(p光):
$$\ar{
    \f{E'}{E}\x \f{\tan(\t-\t'')}{\tan(\t+\t'')}\\
    \f{E''}{E}\x \f{2\cos\t\sin\t''}{\sin(\t+\t'')\cos(\t-\t'')}
}$$
布儒斯特角: 当$\t+\t''=90^\circ$, $E_\parallel'$消失. $\tan\t_B=n_{21}$\\
半波损失: 前一种情况反射波与入射波反相. \\
反射系数和透射系数
$$R=\f{E_0'^2}{E_0^2}, T=\f{n_2\cos\t''}{n_1\cos\t}\f{E_0''^2}{E_0^2}$$
$$R_s=(1)^2,R_p=(2)^2,T_s=\f{\sin2\t\sin2\t''}{\sin^2(\t+\t'')},T_p=\f{4\sin2\t\sin2\t''}{(\sin2\t+\sin2\t'')^2}$$
当$\t=0$, $R_s=R_p=\f{(n_2-n_1)^2}{(n_2+n_1)^2}$. \\
全反射:
$$k_z''=i\kappa,\ \kappa=k\sqrt{\sin^2\t-n_{21}^2}$$
全反射能量守恒为:
$$R+\f{\cos\t''}{\cos\t}T=1$$
垂直入射到良导体($\f{\sg}{\o\ep}>>1$)表面:
$$R=1-2\sqrt{\f{2\o\ep_0}{\sg}}=\f{(n-n_1)^2+\kappa}{(n+n_1)+\kappa^2}$$
$$\ep'=\ep+i\f{\sg}{\o}$$
折射波场沿x轴传播, 场强沿z轴指数衰减:
$$\bm{E}''=\bm{E}''_0e^{-\kappa z}e^{i(k''_xx-\o t)}$$
其厚度$\sim\kappa^{-1}$.
$$\k^{-1}=\ff{\l_1}{2\pi\sqrt{\sin^2\t-n_{21}^2}}$$
\sub{矩形谐振腔}
1代表导体, 2代表真空. 法线由导体指向介质. 满足亥姆霍兹方程$\nabla^2u+k^2u=0$. 边界条件为$E_\parallel=H_\perp=\pp{E_n}{n}=0$.$k=\o\sqrt{\mu\ep}$. \\
满足$k_xA_1+k_yA_2+k_zA_3=0$. 本征频率$\o=\f{\pi}{\sqrt{\mu\ep}}\sqrt{(m/l_1)^2+\cdots}$. 
$$\hua{
    E_x \x A_1\cos k_xx \sin k_y y \sin k_zze^{-i\o t}\\
    E_y \x A_2\sin k_xx \cos k_y y \sin k_zze^{-i\o t}\\
    E_z \x A_3\sin k_xx \sin k_y y \cos k_zze^{-i\o t}
},\ k_x=\f{m\pi}{l_1},k_y=\f{n\pi}{l_2},k_z=\f{p\pi}{l_3}$$
\sub{矩形波导}
$z$方向无穷长的解为$k_x^2+k_y^2+k_z^2=k^2$.$k_xA_1+k_yA_2-ik_zA_3=0$. 
$$\hua{
    E_x \x A_1\cos k_xx \sin k_y y e^{-k_zz}\\
    E_y \x A_2\sin k_xx \cos k_y y e^{-k_zz}\\
    E_z \x A_3\sin k_xx \sin k_y y e^{-k_zz}
},\ k_x=\f{m\pi}{a},k_y=\f{n\pi}{b}$$
$$H=-\f{i}{\o\mu}\nabla\tm E$$
由上式, $E_z=0$则$H_z\neq 0$. 因此波导中的波不能是TEM. 由于$k_z=\sqrt{(\o/c)^2-(k_x^2+k_y^2)}$为实数, 截止频率为$\o=\pi c\sqrt{(m/a)^2+(n/b)^2}$. 相速度<c, 群速度>c.
\sub{等离子体}
振荡频率$\o_p=\sqrt{n_0e^2/m\ep_0}$. $m\ddot{r}=-eE=eE_0e^{i(kx-\o t)}$. $J(\o)=-n_0ev=\sg(\o)E,\sg=i\f{n_0e^2}{m\o}$. 稀薄等离子折射率为$n=\sqrt{1-\o_p^2/\o^2}$. 当$\o>\o_p$,$v_p>c$全反射,可传播电磁波. 
\sub{推迟势}
以$R$表示原点$x'$到场点$x$的距离. $r\approx R-e_R\cdot x'$. 
$$\vp(x,t)=\int_V\f{\r\of{x',t-\f{r}{c}}}{4\pi\ep_0r}\d V'$$
$$A(x,t)=\f{\mu_0}{4\pi}\int_V\f{\bm{J}\of{x',t-\f{r}{c}}}{r}\d V'=\f{\mu_0}{4\pi}\int_V\f{\bm{J}(x')e^{ik(R-e_R\cdot x')}}{R-e_R\cdot x'}\d V'$$
展开第一项为:
$$A(x)=\f{\mu_0e^{ikR}}{4\pi R}\int_VJ(x')\d V'=\f{\mu_0e^{ikR}}{4\pi R}\dot{\bm{p}}$$
可得电偶极辐射:
$$\hua{
    \bm{B}\x\ff{4\pi\ep_0c^3R}\ddot{p}e^{ikR}\sin\t \bm{e}_\vp\\
    \bm{E}\x\ff{4\pi\ep_0c^3R}\ddot{p}e^{ikR}\sin\t \bm{e}_\t\\
    \bar{\bm{S}} \x \f{|\ddot{p}|^2}{32\pi^2\ep_0c^3R^2}\sin^2\t\bm{e}_R\\
    P \x \oint|\bar{\bm{S}}|R^2\d\O = \ff{4\pi\ep_0}\f{|\ddot{p}|^2}{3c^3}
}$$
磁偶极辐射和电四极辐射
展开第二项:($\mathcal{D}=\sum 3qx'_ix'_j-r'^2\de_{ij}$)
$$A(x)=\f{-ik\mu_0e^{ikR}}{4\pi R}\int_V\bm{J}(x')(\bm{e_R\cdot x'}\d V')=\f{-ik\mu_0e^{ikR}}{4\pi R}\of{-\bm{e_R\tm m}+\ff{6}\bm{e_R}\cdot\dot{\mathcal{D}}}$$
先计算磁偶极辐射$A=\f{ik\mu_0e^{ikR}}{4\pi R}\bm{e_R\tm m}$.
$$\hua{
    \bm{B} \x \f{\mu_0e^{ikR}}{4\pi c^2R}(\ddot{m}\tm e_R)e_R\\
    \bm{E} \x -\f{\mu_0e^{ikR}}{4\pi cR}(\ddot{m}\tm e_R)\\
    \bm{\bar{S}} \x \f{\mu_0\o^4|\bm{m}|^2}{32\pi^2c^3R^2}\sin^2\t e_R\\
    P\x \f{\mu_0\o^4|\bm{m}|^2}{12\pi c^3}
}$$
再计算电四极辐射$A=\f{-ik\mu_0e^{ikR}}{24\pi R}\dot{\mathcal{D}}$. 定义$\bm{D}=e_R\cdot \mathcal{D}$. 
$$\hua{
    A(x) \x \f{e^{ikR}}{24\pi\ep_0c^4R}\dddot{\bm{D}}\tm e_R\\
    B \x ik\bm{e_R\tm A}\\
    E \x c\bm{B\tm{e_R}}\\
    S \x \ff{4\pi\ep_0}\ff{288\pi c^5R^2}(\dddot{\bm{D}}\tm e_R)^2e_R
}$$
\sub{衍射}
基尔霍夫公式:$\psi(x)=-\ff{4\pi}\oint_S\f{e^{ikr}}{r}\bm{e}_n\cdot\left[\nabla'\psi+\of{ik-\ff{r}}\f{\bm{r}}{r}\psi\right]\d S'$.\\
夫琅禾费衍射: $x'$为小孔上一点, $x$为空间远处一点, $R$为小孔中心到远处距离. $k_1$沿入射方向, $k_2$沿R方向. $\t_1,\t_2$为入射出射角. 
$$\phi(x)=-\f{i\psi_0e^{ikR}}{4\pi R}\int_{S}e^{i(k_1-k_2)\cdot x'}(\cos\t_1+\cos\t_2)\d S'$$
长宽为$\a,\b$的矩形孔夫琅禾费衍射为:
$$I=I_0\of{\f{1+\cos\t_2}{2}}^2\of{\f{\sin ka\a}{ka\a}}^2\of{\f{\sin kb\b}{kb\b}}^2$$
电磁场动量:\\
动量密度$\bm{g}=\ep_0\bm{E\tm B}=\ff{c^2}\bm{S}=\f{w}{c}\bm{e}_k$. 
\se{狭义相对论}
$$ 
a=\left(\begin{array}{cccc}{\gamma} & {0} & {0} & {i \beta \gamma} \\ {0} & {1} & {0} & {0} \\ {0} & {0} & {1} & {0} \\ {-i \beta \gamma} & {0} & {0} & {\gamma}\end{array}\right),\quad\beta=\frac{v}{c}, \quad \gamma=\frac{1}{\sqrt{1-\beta^{2}}}
 $$
定义固有时$d \tau=\frac{1}{c} d s$和4-速度:$U_{\mu}=\frac{d x_{\mu}}{d \tau}=\gamma_{u}\left(u_{1}, u_{2}, u_{3}, i c\right)$. \\
相对论多普勒效应
$$ 
\omega \approx \frac{\omega_{0}}{1-\frac{v}{c} \cos \theta}
 $$
定义场强张量
$$ 
F_{\mu \nu}=\left[\begin{array}{cccc}{0} & {B_{3}} & {-B_{2}} & {-\frac{1}{c} E_{1}} \\ {-B_{3}} & {0} & {B_{1}} & {-\frac{\mathbf{i}}{c} E_{2}} \\ {B_{2}} & {-B_{1}} & {0} & {-\frac{\mathbf{i}}{c} E_{3}} \\ {\frac{\mathbf{i}}{c} E_{1}} & {\frac{\mathbf{i}}{\boldsymbol{c}} \boldsymbol{E}_{2}} & {\frac{\mathbf{i}}{c} \boldsymbol{E}_{3}} & {0}\end{array}\right]
 $$
 Maxwell方程变为
 $$ 
\frac{\partial F_{\mu v}}{\partial x_{v}}=\mu_{0} J_{\mu}
 $$
 $$ 
\frac{\partial F_{\mu \nu}}{\partial x_{\lambda}}+\frac{\partial F_{v \lambda}}{\partial x_{\mu}}+\frac{\partial F_{\lambda \mu}}{\partial x_{v}}=0
 $$
 且满足
 $$ 
\begin{aligned} \frac{1}{2} F_{\mu \nu} F_{\mu \nu}=B^{2} &-\frac{1}{c^{2}} E^{2} \\ \frac{i}{8} \epsilon_{\mu \nu \lambda \tau} F_{\mu \nu} F_{\lambda \tau}=& \frac{1}{c} \vec{B} \cdot \vec{E} \end{aligned}
 $$
能动量洛伦兹变换
$$ 
\begin{array}{l}{p_{1}=\frac{p_{1}^{\prime}+\frac{\beta_{c}}{c^{2}} E_{1}^{\prime}}{\sqrt{1-\beta_{c}^{2} / c^{2}}} ; E_{1}=\frac{E_{1}^{\prime}+\beta_{c} p_{1}^{\prime}}{\sqrt{1-\beta_{c}^{2} / c^{2}}}} \\ {p_{2}=\frac{p_{2}^{\prime}+\frac{\beta_{c}}{c^{2}} E_{2}^{\prime}}{\sqrt{1-\beta_{c}^{2} / c^{2}}} ;E_{2}=\frac{E_{2}^{\prime}+\beta_{c} p_{2}^{\prime}}{\sqrt{1-\beta_{c}^{2} / c^{2}}}}\end{array}
 $$
\se{数学}
\sub{柱坐标系 $(\rho,\phi,z)$}
$$\nabla \varphi = \hat e_1 \frac { \partial \varphi } { \partial \rho } + \hat e_2 \frac { 1 } { \rho } \frac { \partial \varphi } { \partial \phi } + \hat { e } _ { 3 } \frac { \partial \varphi } { \partial z }$$
$$\nabla \cdot \bm{ A } = \frac { 1 } { \rho } \frac { \partial ( \rho A _ { 1 } ) } { \partial \rho } + \frac { 1 } { \rho } \frac { \partial A _ { 2 } } { \partial \phi } + \frac { \partial A _ { 3 } } { \partial z }$$
$$ { \nabla \times \bm{ A } } =\hat e_1 (\ff{\rho} \frac { \partial A_3} { \partial \phi } - \frac { \partial A _ { 2 } } { \partial z } ) + \hat { e } _ { 2 } ( \frac { \partial A _ { 1 } } { \partial z } - \frac { \partial A _ { 3 } } { \partial \rho } ) + \hat { e } _ { 3 } \frac { 1 } { \rho } ( \frac { \partial ( \rho A _ { 2 } ) } { \partial \rho } - \frac { \partial A _ { 1 } } { \partial \phi } )$$
$$\nabla ^ { 2 } \varphi = \frac { 1 } { \rho } \frac { \partial } { \partial \rho } ( \rho \frac { \partial \varphi } { \partial \rho } ) + \frac { 1 } { \rho ^ { 2 } } \frac { \partial ^ { 2 } \varphi } { \partial \phi ^ { 2 } } + \frac { \partial ^ { 2 } \varphi } { \partial z ^ { 2 } }$$
\sub{球坐标系 $(r,\t,\varphi)$ }
$$\nabla \varphi = \hat { e } _ { 1 } \frac { \partial \varphi } { \partial r } + \hat { e } _ { 2 } \frac { 1 } { r } \frac { \partial \varphi } { \partial \theta } + \hat { e } _ { 3 } \frac { 1 } { r \sin \theta } \frac { \partial \varphi } { \partial \phi }q$$
$$\nabla \cdot \bm{ A } = \ff{r^2} \pp{r^2A_1}{r} + \frac { 1 } { r \sin \theta } \frac { \partial } { \partial \theta } ( \sin \theta A _ { 2 } ) + \frac { 1 } { r \sin \theta } \frac { \partial A _ { 3 } } { \partial \phi }$$
$$\nabla \times \bm{ A }  = 
\hat { e } _ { 1 } \frac { 1 } { r \sin \theta } \left[ \frac { \partial } { \partial \theta } ( \sin \theta A _ { 3 } ) - \frac { \partial A _ { 2 } } { \partial \phi } \right]  
+ \hat { e } _ { 2 } \left[ \frac { 1 } { r \sin \theta } \frac { \partial A _ { 1 } } { \partial \phi } - \frac { 1 } { r } \frac { \partial } { \partial r } ( r A _ { 3 } ) \right] 
+ \hat { e } _ { 3 } \frac { 1 } { r } \left[ \frac { \partial } { \partial r } ( r A _ { 2 } ) - \frac { \partial A _ { 1 } } { \partial \theta } \right] $$
$$\nabla^2\varphi=\ff{r^2\sin\t}\left[\sin\t\pp{}{r}(r^2\pp{\varphi}{r})+\pp{}{\t}(\sin\t\pp{\varphi}{\t})+\ff{\sin\t}\pp[2]{\varphi}{\phi}\right]$$
\sub{矢量变换}
$$\nabla\bm{r}=-\nabla'\bm{r}=\bm{e}_r$$
$$\nabla\ff{\bm{r}}=-\nabla'\ff{\bm{r}}=-\ff{r^2}\bm{e}_r$$
$$\curl\ff{\bm{r}^2}=\div\ff{\bm{r}}=0$$
$$\div\vp\bm{A}=\vp\div\bm{A}+\bm{A}\cdot\grad\vp$$
$$\curl\vp\bm{A}=\vp\curl\bm{A}+\grad\vp\times\bm{A}$$
$$\div(\bm{A}\cdot\bm{B}) = (\bm{A}\cdot\nabla)\bm{B}+(\bm{B}\cdot\nabla)\bm{A}+\bm{A}\tm(\curl\bm{B})+\bm{B}\tm(\curl\bm{A})$$
$$\nabla\cdot(\bm{F} \times \bm{G})=(\nabla\times \bm{F})\cdot \bm{G}-\bm{F}\cdot(\nabla\times \bm{G})$$
$$\bm{A}\times(\bm{B}\times\bm{C})=(\bm{A}\cdot\bm{C})\bm{B}-(\bm{A}\cdot\bm{B})\bm{C}$$
$$( \bm{ A } \times \bm{ B } ) \times ( \bm{ C } \times \bm{ D } ) = [ \bm{ A } \cdot ( \bm{ B } \times \bm{ D } ) ] \bm{ C } - [ \bm{ A } \cdot ( \bm{ B } \times \bm{ C } ) ] \bm{ D }$$
$$( \bm{ A } \times \bm{ B } ) \cdot ( \bm{ C } \times \bm{ D } ) = \abs{\ar{\bm{A}\cdot\bm{C}&\ \ \bm{A}\cdot\bm{D}\\
\bm{B}\cdot\bm{C}&\ \ \bm{B}\cdot\bm{D}}}$$
$$\bm{ A } \times ( \bm{ B } \times \bm{ C } ) + \bm{ B } \times ( \bm{ C } \times \bm{ A } ) + \bm{ C } \times ( \bm{ A } \times \bm{ B } ) = 0$$
\end{document}